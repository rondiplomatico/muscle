%%This is a very basic article template.
%%There is just one section and two subsections.
\documentclass[a4paper,11pt]{article}
%\documentclass[a4paper,12pt]{article}
\usepackage[english]{babel}
\usepackage{amsmath}
\usepackage{amssymb}
\usepackage{amsthm}
\usepackage[makeroom]{cancel}

% Graphics
\usepackage{graphicx}
\graphicspath{{img/}}
\DeclareGraphicsExtensions{.eps,.png,.pdf,.jpg}

% Windows-only
\usepackage[utf8]{inputenc}
\usepackage{epstopdf}
%pdf latex -synctex=1 --shell-escape -interaction=nonstopmode --src-specials

%\usepackage{libertine}
%\usepackage[ttscale=.875]{libertine}

% Display
\usepackage[font={small,it}]{caption}
%\usepackage{multirow}
%\usepackage{fancyvrb}
%\usepackage{listings}

% Illustrations
%\usepackage{tikz}
%usetikzlibrary{arrows,automata}

% Algorithms
%\usepackage{algorithm}
%\usepackage{algpseudocode}

% For bibliography
\usepackage{url}

% \usepackage{setspace}
\usepackage{geometry}
\geometry{%
	top=25mm, bottom=30mm,
	left=20mm, right=20mm, twoside
}

\setlength{\parskip}{8pt}
\setlength{\parindent}{0cm}

\newlength{\myleftmargin}
\setlength{\myleftmargin}{-6ex}
\newlength{\fixboxwidth}
\setlength{\fixboxwidth}{\marginparwidth}
\addtolength{\fixboxwidth}{\myleftmargin}
\newcommand{\fix}[1]{\marginpar{%
    \fbox{\parbox{\fixboxwidth}{\footnotesize \red{#1}}}}}

%%%%%%%%%%%%%%%%%%%%%%%%%% Text commands
\usepackage{color}
\definecolor{emph}{rgb}{.1,.4,.1}
\definecolor{dark-red}{rgb}{0.4,0.15,0.15}
\definecolor{dark-blue}{rgb}{0.15,0.15,0.4}
\definecolor{medium-blue}{rgb}{0,0,0.5}
\newcommand{\ow}[1]{``\emph{#1}''}%\textcolor{emph}{\emph{#1}}
\newcommand{\e}[1]{\emph{#1}}
\newcommand{\red}[1]{\textcolor{red}{#1}}
%\newenvironment{redp}{\par\color{red}}{\par}

%%%%%%%%%%%%%%%%%%%% Figure stuff
\newenvironment{myfig}[2]{%
\def\mystupidworkaround{#1}
\def\mystupidworkaroundtwo{#2}
\begin{center}
}{%
	\captionof{figure}{\mystupidworkaroundtwo}
  	\label{\mystupidworkaround}
\end{center}
}
% \usepackage{float}
% \newenvironment{myfig}[2]{%
% \def\mystupidworkaround{#1}
% \def\mystupidworkaroundtwo{#2}
% \begin{figure}[H]
% 	\centering
% }{%
%  	\caption{\mystupidworkaroundtwo}
%  	\label{\mystupidworkaround}
% \end{figure}
% }

% % MathBBM / MathCal symbols
\newcommand{\R}{\mathbb{R}}
\newcommand{\N}{\mathbb{N}}
\newcommand{\D}{\mathcal{D}}
\renewcommand{\P}{\mathcal{P}}
\renewcommand{\L}{\mathcal{L}}
\newcommand{\Udeim}{\mathcal{U}}
\newcommand{\E}{\mathcal{E}}
\newcommand{\V}{\mathcal{V}}
\newcommand{\W}{\mathcal{W}}
\newcommand{\Rp}{\mathbb{R}_+}

%%%%%%%%%%%%%%%%%%%%%%%% Dimensions
% State space: d
\def\dis{l} % input space
\def\dos{k} % output space
\def\dpar{p} % parameter space

%%%%%%%%%%%%%%%%%%%%%%%% Kernel related 
\newcommand{\K}{K}
\renewcommand{\H}{\mathcal{H}}
\newcommand{\Pxdot}{\K(\vx,\cdot)}
\newcommand{\Pydot}{\K(\vy,\cdot)}
\newcommand{\Pxidot}{\K(\vx_i,\cdot)}
\newcommand{\Pxjdot}{\K(\vx_j,\cdot)}
\newcommand{\Pyidot}{\K(\vy_i,\cdot)}
\newcommand{\Kxix}{\K(\vx_i,\vx)}
\newcommand{\spH}[2]{\sp{#1}{#2}_{\H}}
\newcommand{\spHq}[2]{\sp{#1}{#2}_{\H^q}}
\newcommand{\noH}[1]{\norm{#1}{\H}}
\newcommand{\noHq}[1]{\norm{#1}{\H^q}}
\newcommand{\Kh}{\tilde{\K}}
\newcommand{\vci}{\vc_i}

%%%%%%%%%%%%%%%%%%%%%%%% Vectors and Matrices
\newcommand{\mx}[1]{\ensuremath{\left(\begin{matrix}#1\end{matrix}\right)}}
\newcommand{\sm}[1]{\ensuremath{\left(\begin{smallmatrix}#1\end{smallmatrix}\right)}}
\newcommand{\makebf}[1]{\boldsymbol{#1}}
\newcommand{\vnull}{\makebf{0}}
\newcommand{\vone}{\makebf{1}}
\newcommand{\vA}{\makebf{A}}
\newcommand{\vB}{\makebf{B}}
\newcommand{\vC}{\makebf{C}}
\newcommand{\vD}{\makebf{D}}
\newcommand{\vE}{\makebf{E}}
\newcommand{\vF}{\makebf{F}}
\newcommand{\vG}{\makebf{G}}
\newcommand{\vH}{\makebf{H}}
\newcommand{\vI}{\makebf{I}}
\newcommand{\vJ}{\makebf{J}}
\newcommand{\vK}{\makebf{K}}
\newcommand{\vL}{\makebf{L}}
\newcommand{\vM}{\makebf{M}}
\newcommand{\vN}{\makebf{N}}
\newcommand{\vP}{\makebf{P}}
\newcommand{\vQ}{\makebf{Q}}
\newcommand{\vR}{\makebf{R}}
\newcommand{\vS}{\makebf{S}}
\newcommand{\vU}{\makebf{U}}
\newcommand{\vV}{\makebf{V}}
\newcommand{\vW}{\makebf{W}}
\newcommand{\vX}{\makebf{X}}
\newcommand{\vY}{\makebf{Y}}
\newcommand{\va}{\makebf{a}}
\newcommand{\vb}{\makebf{b}}
\newcommand{\vc}{\makebf{c}}
\newcommand{\vd}{\makebf{d}}
\newcommand{\ve}{\makebf{e}}
\newcommand{\vf}{\makebf{f}}
\newcommand{\vg}{\makebf{g}}
\newcommand{\vj}{\makebf{j}}
\newcommand{\vk}{\makebf{k}}
\newcommand{\vn}{\makebf{n}}
\newcommand{\vp}{\makebf{p}}
\newcommand{\vq}{\makebf{q}}
\newcommand{\vr}{\makebf{r}}
\newcommand{\vs}{\makebf{s}}
\newcommand{\vu}{\makebf{u}}
\newcommand{\vv}{\makebf{v}}
\newcommand{\vw}{\makebf{w}}
\newcommand{\vx}{\makebf{x}}
\newcommand{\vy}{\makebf{y}}
\newcommand{\vz}{\makebf{z}}
\newcommand{\val}{\makebf{\alpha}}
\newcommand{\vmu}{{\makebf{\mu}}}
\newcommand{\vxi}{\makebf{\xi}}
\newcommand{\veta}{\makebf{\eta}}
\newcommand{\vSig}{\makebf{\Sigma}}
\newcommand{\vLam}{\makebf{\Lambda}}

% Tilde: Projected quantities 
\newcommand{\vtA}{\tilde{\vA}}
\newcommand{\vtB}{\tilde{\vB}}
\newcommand{\vtC}{\tilde{\vC}}
\newcommand{\vtJ}{\tilde{\vJ}}
\newcommand{\vtM}{\tilde{\vM}}
\newcommand{\vtQ}{\tilde{\vQ}}
\newcommand{\vtU}{\tilde{\vU}}
\newcommand{\vtV}{\tilde{\vV}}
\newcommand{\vtc}{\tilde{\vc}}
\newcommand{\vtq}{\tilde{\vq}}
\newcommand{\vtx}{\tilde{\vx}}
\newcommand{\vtv}{\tilde{\vv}}
\newcommand{\vtz}{\tilde{\vz}}
\newcommand{\tf}{\tilde{f}}
\newcommand{\tlambda}{\tilde{\lambda}}

% Hat: Approximated quantities
\newcommand{\hf}{\hat{f}}
\newcommand{\vhf}{{\hat{\vf}}}
\newcommand{\vhV}{{\hat{\vV}}}
\newcommand{\vhW}{{\hat{\vW}}}
\newcommand{\vhP}{\hat{\vP}}
\newcommand{\vhR}{{\hat{\vR}}}
\newcommand{\vhU}{\hat{\vU}}

% Tilde hat: Projected approximated quantities
\newcommand{\vthf}{\tilde{\vhf}}

%%%%%%%%%%%%%%%%%%%%%%%% Misc
\renewcommand{\O}[1]{\ensuremath{\mathcal{O}\left(#1\right)}}
\newcommand{\bs}{\backslash}
\newcommand{\fo}{~\forall~}
\newcommand{\ex}{~\exists~}
\newcommand{\exu}{~\exists!~}
\newcommand{\ep}{\epsilon}
\newcommand{\es}{\emptyset}
\newcommand{\norm}[2]{\left|\left|#1\right|\right|_{#2}}
\newcommand{\no}[1]{\norm{#1}{}}
\newcommand{\noG}[1]{\norm{#1}{G}}
\newcommand{\Mmo}{\mathcal{C}}
\newcommand{\mmo}{\mathcal{D}}
\newcommand{\PM}{\P_{\Mmo}}
\newcommand{\Pm}{\P_{\mmo}}
\newcommand{\flux}{f}%
\newcommand{\sat}{s}%s^{\ep,\tau}
\newcommand{\col}{p{2cm}}

\renewcommand{\sp}[2]{\left\langle #1,#2 \right\rangle}
\newcommand{\spG}[2]{\sp{#1}{#2}_{G}}
% \newcommand{\spHq}[2]{\sp{#1}{#2}_{\H^q}}
\newcommand{\Ball}[2]{B_{#1}\left(#2\right)}

%%%%%%%%%%%%%%%%%%%%%%%%% Sizes
\newcommand{\single}{.8\textwidth} % 1 image in row
\newcommand{\third}{.32\textwidth}
\newcommand{\half}{.49\textwidth}
\newcommand{\quarter}{.24\textwidth}

%%%%%%%%%%%%%%%%%%%%%%%%% Sums, integrals, limits 
\newcommand{\suml}[2]{\sum\limits_{#1}^{#2}}
\newcommand{\sumi}{\suml{i=1}{N}}
\newcommand{\sumim}{\suml{i=1}{m}}
\newcommand{\sumj}{\suml{j=1}{N}}
\newcommand{\sumk}{\suml{k=1}{N}}
\newcommand{\sumjq}{\suml{j=1}{q}}
\newcommand{\sumr}{\suml{i=1}{r}}
\newcommand{\sumjk}{\suml{j=1}{k}}
\newcommand{\sumik}{\suml{j=1}{k}}
\newcommand{\sumkM}{\suml{k=1}{M}}
\newcommand{\intl}[2]{\int\limits_{#1}^{#2}}

%%%%%%%%%%%%%%%%%%%%%%%%%% SVR-Commands
\newcommand{\ap}{\val^+}
\newcommand{\am}{\val^-}
\newcommand{\aip}{\alpha_i^+}
\newcommand{\aim}{\alpha_i^-}
\newcommand{\ajp}{\alpha_j^+}
\newcommand{\ajm}{\alpha_j^-}
\newcommand{\apm}{\ap-\am}
\newcommand{\apmb}{\left(\ap-\am\right)}
\newcommand{\dw}{\nabla W}
\newcommand{\dwp}{\dw^+(\ap,\am)}
\newcommand{\dwpi}{\dw_i^+(\ap,\am)}
\newcommand{\dwpj}{\dw_j^+(\ap,\am)}
\newcommand{\dwm}{\dw^-(\ap,\am)}
\newcommand{\dwmi}{\dw_i^-(\ap,\am)}
\newcommand{\dwmj}{\dw_j^-(\ap,\am)}
\newcommand{\vdw}{\nabla \vW}
\newcommand{\clip}[3]{\left[#1\right]^{#2}_{#3}}
\newcommand{\nonp}[1]{\norm{#1}{\Np}}
\newcommand{\ripp}{r_i^{++}}
\newcommand{\sjpp}{s_j^{++}}
\newcommand{\xip}{\xi_i^+}
\newcommand{\xim}{\xi_i^-}
\newcommand{\noep}[1]{\left|#1\right|_\ep}
\newcommand{\wplus}{\nabla\vw^+} 
\newcommand{\wm}{\nabla\vw^-}

%%%%%%%%%%%%%%%%%%%%%%%%%% VKOGA-Commands
\newcommand{\tp}{\tilde{\phi}}
\newcommand{\sfx}{{s_{f,X}}}
\newcommand{\sfxm}{{s_{f,X_m}}}
\newcommand{\Pq}{\P^q}
\newcommand{\PHX}{\P_{\H^X}}
%\newcommand{\PHXq}{\Pq_{\H^X}}
\newcommand{\Kinj}{(\vK^{-1})_j^T}
\newcommand{\OX}{\Omega_X}
\newcommand{\OXm}{\vx\in\Omega\bs\Omega_{X_{m-1}}}
\newcommand{\nKx}{\nabla\vK(\vx)}
\newcommand{\gxf}{G_{X,\vf}} % vectorial gain function
\newcommand{\px}{\phi_{\vx}}
\newcommand{\pxv}{\px^{\nabla v}}
\newcommand{\tpx}{\tp_{\vx}}
\newcommand{\tpv}{\tp^{\nabla v}}
\newcommand{\vtpx}{\makebf{\tp}_{\vx}}
\newcommand{\tpxv}{\tpx^{\nabla v}}
\newcommand{\tN}{\tilde{N}}
\newcommand{\im}{{i_{max}}}

%%%%%%%%%%%%%%%%%%%%%% DEIM commands
\newcommand{\ei}{{\wp_i}}
\newcommand{\DEI}{\DE_{EI}}
\newcommand{\DEIJ}{\DE}
\newcommand{\Prm}{\Pi_m}
\newcommand{\Pmd}{\Pi_{m'}}
\newcommand{\proj}{\bigl(\vI - \vV\vW^T\bigr)}

%%%%%%%%%%%%%%%%%%%%%% Operators
\DeclareMathOperator{\rangeop}{range}
\DeclareMathOperator{\meanop}{mean}
\DeclareMathOperator{\sgn}{sign}
\DeclareMathOperator{\diagmat}{diag}
\DeclareMathOperator{\divop}{div}
\newcommand{\sign}[1]{\sgn\left(#1\right)}
\newcommand{\range}[1]{\rangeop\left(#1\right)}
\newcommand{\diag}[1]{\diagmat\left(#1\right)}
\renewcommand{\d}[2]{\frac{\partial #1}{\partial #2}}
\newcommand{\dx}{\partial_x}%
\newcommand{\dd}[2]{\frac{\partial^2 #1}{\partial {#2}^2}}

%%%%%%%%%%%%%%%%%%%%%% Error estimation
\newcommand{\intt}{\intl{0}{t}}
\newcommand{\exo}{E_0} % initial error
\newcommand{\exomu}{\exo(\vmu)}
\newcommand{\ea}{E_A} % approximation error
\newcommand{\eproj}{\left(\vI-\vV\vW^T\right)} %projection error
\newcommand{\fd}{{h_{X,\Omega}}} % fill distance
\renewcommand{\Ball}[2]{\overline{B_{#1}\left(#2\right)}}
\newcommand{\Bts}{\Ball{\Theta}{s}}
\newcommand{\Bty}{\Ball{\Theta}{y}}
\newcommand{\Bf}{\mathcal{B}} % bell functions
\newcommand{\DE}{\Delta}
\newcommand{\DGLE}{\DE_{GLE}}
\newcommand{\DLSLE}{\DE_{LSLE}} 
\newcommand{\Gno}{\Gamma_1^c}

%%%%%%%%%%%%%%%%%%%% Code
\newcommand{\ML}{\textsc{MatLab}}%{\tiny\texttrademark~}
\newcommand{\KM}{\textit{KerMor}}
\newcommand{\code}[1]{\lstinline$#1$}
\definecolor{c-keywords}{rgb}{.1,.1,.5}
\definecolor{c-identifier}{rgb}{0.3,0.3,0.3}

\usepackage{verbatim}
\lstset{language=Matlab,
	basicstyle=\scriptsize\ttfamily,
	keywordstyle=\bfseries\ttfamily\color{c-keywords},
	identifierstyle=\color{c-identifier},
	commentstyle=\color{green},
	stringstyle=\ttfamily,
	showstringspaces=false,
	fancyvrb=true,
	xleftmargin=5pt,
	morekeywords={doc,help,methods,events,properties,abstract,
                      classdef,double,true,false,this,access,setaccess,getaccess,
		      varargin, varargout}
}
 
% %%%%%%%%%%%%%%%%%%%%%%%%%%%%%%%%%%%%%%%%%%%%%%%%%%%%%%%%%%%%%%%%%%%%%
% %% THEOREM & LEMMA COMMANDS
% %%%%%%%%%%%%%%%%%%%%%%%%%%%%%%%%%%%%%%%%%%%%%%%%%%%%%%%%%%%%%%%%%%%%%
\newtheorem{theorem}{Theorem}[section]
\newtheorem{lemma}[theorem]{Lemma}
\newtheorem{corollary}[theorem]{Corollary}
\newtheorem{proposition}[theorem]{Proposition}
 	
% Option 1: Continuous numbering also for defs & remarks
\theoremstyle{definition}
\newtheorem{definition}[theorem]{Definition}
\theoremstyle{remark}
\newtheorem{remark}[theorem]{Remark}
% 	% Option 2: Own counters for defs & remarks
% 	\theoremstyle{definition}
% 	\newtheorem{definition}{Definition}[section]
% 	\theoremstyle{remark}
% 	\newtheorem{remark}{Remark}[section]
% 	
% 	%\newtheorem{example}[theorem]{Example}
% 	%\newtheorem{xca}[theorem]{Exercise}

%--------------------------------------------------------------------------------------------------
%
% place to define macros, shortcuts, etc. ...
%
%--------------------------------------------------------------------------------------------------
%
 % e.g., i.e., c.f.
%
%--------------------------------------------------------------------------------------------------
%
\newcommand{\eg}{e.g.~}
\newcommand{\ie}{i.e.~}
\newcommand{\iec}{i.e.,}
\newcommand{\cf}{c.f.~}
%
%--------------------------------------------------------------------------------------------------
%
% ENVIRONMENTS
%
%--------------------------------------------------------------------------------------------------
% figure
\newcommand\bfg[1]{
        \begin{figure}[#1]
        \begin{center}
}
\newcommand\efg{
        \end{center}
        \end{figure}
} 
%
% minipage  <--- NOT WORKING ...?
\newcommand\bmp[1]{
        \begin{minipage}{#1}
        \begin{center}
}
\newcommand\emp[1]{
        \end{center}
        \end{minipage}
}                
%
%--------------------------------------------------------------------------------------------------
%
 % REFERENCES
%
%--------------------------------------------------------------------------------------------------
% equation
\newcommand{\eqnrefE}[1]{\textup{Equation~}(\ref{#1})}
\newcommand{\eqnref}[1]{(\ref{#1})}
%
% figure
\newcommand{\figref}[1]{\textup{Figure~}\ref{#1}}
%
% table
\newcommand{\tabref}[1]{\textup{Table~}\ref{#1}}
%
% chapter
\newcommand{\chref}[1]{\textup{Chapter~}\ref{#1}}
%
% section
\newcommand{\secref}[1]{\textup{Section~}\ref{#1}}
%
% subsection
\newcommand{\subsecref}[1]{\textup{Subsection~}\ref{#1}}
%
%--------------------------------------------------------------------------------------------------
%
% SHORTCUTS IN MATHEMATICS
%
%--------------------------------------------------------------------------------------------------
% environments
%
% equation + label - (use as \beq{wantedlabelname} ... \eeq)
% --> use equation + split for one alignment
\newcommand{\beq}[1]{\begin{equation}\label{#1}}
\newcommand{\eeq}{\end{equation}}
%
% split
\newcommand{\splt}[1]{\begin{split} #1 \end{split}}
%
% align - (use as \alg{...})
% --> use align for multiple alignments & each line numbered
\newcommand{\alg}[1]{\begin{align} #1 \end{align}} 
%
% array
\newcommand{\barray}[1]{\begin{array}{#1}}
\newcommand{\earray}{\end{array}}
%
%------------------------------------------------
% short notations
%
% number symbols
%\newcommand{\N}{\mathbb{N}} % integer numbers
%\newcommand{\R}{\mathbb{R}} % real numbers
%
% bold numbers
\newcommand{\Bzero}{\boldsymbol{0}}
\newcommand{\Bone}{\boldsymbol{1}}
%
% bold letters - small
\newcommand{\Ba}{\boldsymbol{a}}
\newcommand{\Bb}{\boldsymbol{b}}
\newcommand{\Bc}{\boldsymbol{c}}
\newcommand{\Bd}{\boldsymbol{d}}
\newcommand{\Be}{\boldsymbol{e}}
%\newcommand{\Bf}{\boldsymbol{f}}
\newcommand{\Bg}{\boldsymbol{g}}
\newcommand{\Bh}{\boldsymbol{h}}
\newcommand{\Bi}{\boldsymbol{i}}
\newcommand{\Bj}{\boldsymbol{j}}
\newcommand{\Bk}{\boldsymbol{k}}
\newcommand{\Bl}{\boldsymbol{l}}
\newcommand{\Bm}{\boldsymbol{m}}
\newcommand{\Bn}{\boldsymbol{n}}
\newcommand{\Bo}{\boldsymbol{o}}
\newcommand{\Bp}{\boldsymbol{p}}
\newcommand{\Bq}{\boldsymbol{q}}
\newcommand{\Br}{\boldsymbol{r}}
\newcommand{\Bs}{\boldsymbol{s}}
\newcommand{\Bt}{\boldsymbol{t}}
\newcommand{\Bu}{\boldsymbol{u}}
\newcommand{\Bv}{\boldsymbol{v}}
\newcommand{\Bw}{\boldsymbol{w}}
\newcommand{\Bx}{\boldsymbol{x}}
\newcommand{\By}{\boldsymbol{y}}
\newcommand{\Bz}{\boldsymbol{z}}
%
% bold letters - capital
\newcommand{\BA}{\boldsymbol{A}}
\newcommand{\BB}{\boldsymbol{B}}
\newcommand{\BC}{\boldsymbol{C}}
\newcommand{\BD}{\boldsymbol{D}}
\newcommand{\BE}{\boldsymbol{E}}
\newcommand{\BF}{\boldsymbol{F}}
\newcommand{\BG}{\boldsymbol{G}}
\newcommand{\BH}{\boldsymbol{H}}
\newcommand{\BI}{\boldsymbol{I}}
\newcommand{\BJ}{\boldsymbol{J}}
\newcommand{\BK}{\boldsymbol{K}}
\newcommand{\BL}{\boldsymbol{L}}
\newcommand{\BM}{\boldsymbol{M}}
\newcommand{\BN}{\boldsymbol{N}}
\newcommand{\BO}{\boldsymbol{O}}
\newcommand{\BP}{\boldsymbol{P}}
\newcommand{\BQ}{\boldsymbol{Q}}
\newcommand{\BR}{\boldsymbol{R}}
\newcommand{\BS}{\boldsymbol{S}}
\newcommand{\BT}{\boldsymbol{T}}
\newcommand{\BU}{\boldsymbol{U}}
\newcommand{\BV}{\boldsymbol{V}}
\newcommand{\BW}{\boldsymbol{W}}
\newcommand{\BX}{\boldsymbol{X}}
\newcommand{\BY}{\boldsymbol{Y}}
\newcommand{\BZ}{\boldsymbol{Z}}
%
% bold greek symbols - small
\newcommand{\Balpha}{\boldsymbol{\alpha}}
\newcommand{\Bbeta}{\boldsymbol{\beta}}
\newcommand{\Bgamma}{\boldsymbol{\gamma}}
\newcommand{\Bdelta}{\boldsymbol{\delta}}
\newcommand{\Bepsilon}{\boldsymbol{\epsilon}}
\newcommand{\Bvarepsilon}{\boldsymbol{\varepsilon}}
\newcommand{\Bzeta}{\boldsymbol{\zeta}}
\newcommand{\Beta}{\boldsymbol{\eta}}
\newcommand{\Btheta}{\boldsymbol{\theta}}
\newcommand{\Biota}{\boldsymbol{\iota}}
\newcommand{\Bkappa}{\boldsymbol{\kappa}}
\newcommand{\Bvarkappa}{\boldsymbol{\varkappa}}
\newcommand{\Blambda}{\boldsymbol{\lambda}}
\newcommand{\Bmu}{\boldsymbol{\mu}}
\newcommand{\Bnu}{\boldsymbol{\nu}}
\newcommand{\Bxi}{\boldsymbol{\xi}}
\newcommand{\Bpi}{\boldsymbol{\pi}}
\newcommand{\Brho}{\boldsymbol{\rho}}
\newcommand{\Bvarrho}{\boldsymbol{\varrho}}
\newcommand{\Bsigma}{\boldsymbol{\sigma}}
\newcommand{\Btau}{\boldsymbol{\tau}}
\newcommand{\Bupsilon}{\boldsymbol{\upsilon}}
\newcommand{\Bphi}{\boldsymbol{\phi}}
\newcommand{\Bvarphi}{\boldsymbol{\varphi}}
\newcommand{\Bchi}{\boldsymbol{\chi}}
\newcommand{\Bpsi}{\boldsymbol{\psi}}
\newcommand{\Bomega}{\boldsymbol{\omega}}
%
% bold greek symbols - capital
\newcommand{\Alpha}{A}
\newcommand{\BAlpha}{\boldsymbol{\Alpha}}
\newcommand{\BBeta}{\boldsymbol{B}}
\newcommand{\BGamma}{\boldsymbol{\Gamma}}
\newcommand{\BDelta}{\boldsymbol{\Delta}}
\newcommand{\Epsilon}{E}
\newcommand{\BEpsilon}{\boldsymbol{\Epsilon}}
\newcommand{\Zeta}{Z}
\newcommand{\BZeta}{\boldsymbol{\Zeta}}
\newcommand{\Eta}{H}
\newcommand{\BEta}{\boldsymbol{\Eta}}
\newcommand{\BTheta}{\boldsymbol{\Theta}}
\newcommand{\Iota}{I}
\newcommand{\BIota}{\boldsymbol{\Iota}}
\newcommand{\Kappa}{K}
\newcommand{\BKappa}{\boldsymbol{\Kappa}}
\newcommand{\BLambda}{\boldsymbol{\Lambda}}
\newcommand{\Mu}{M}
\newcommand{\BMu}{\boldsymbol{\Mu}}
\newcommand{\Nu}{N}
\newcommand{\BNu}{\boldsymbol{\Nu}}
\newcommand{\BXi}{\boldsymbol{\Xi}}
\newcommand{\BPi}{\boldsymbol{\Pi}}
\newcommand{\Rho}{R}
\newcommand{\BRho}{\boldsymbol{\Rho}}
\newcommand{\BSigma}{\boldsymbol{\Sigma}}
\newcommand{\Tau}{T}
\newcommand{\BTau}{\boldsymbol{\Tau}}
\newcommand{\BUpsilon}{\boldsymbol{\Upsilon}}
\newcommand{\BPhi}{\boldsymbol{\Phi}}
\newcommand{\Chi}{X}
\newcommand{\BChi}{\boldsymbol{\Chi}}
\newcommand{\BPsi}{\boldsymbol{\Psi}}
\newcommand{\BOmega}{\boldsymbol{\Omega}}
%
% special stuff
\newcommand{\delOmega}{\partial\Omega}
%
%------------------------------------------------
%
% partial derivative - (use by \delby{}{})
\newcommand{\delby}[2]{\frac{\partial #1}{\partial #2}}
%
% integral - (use by \intlim{start}{stop}{integrand}{variable})
\newcommand{\intlim}[4]{
  \int\limits_{#1}^{#2}{#3}\hspace{2pt}d#4  
}
%
% sum - (use by \sumlim{start}{stop}{summands})
\newcommand{\sumlim}[3]{
  \sum\limits_{#1}^{#2}{#3} 
}
%
% norm with varying size
%\newcommand{\norm}[1]{\left\lVert#1\right\rVert}
%
%--------------------------------------------------------------------------------------------------
%
%--------------------------------------------------------------------------------------------------

%
\newcommand{\Nk}{\mathcal{N}_k}
\newcommand{\Or}{\Omega_0}
\newcommand{\Om}{\Omega_R}
\newcommand{\intor}{\intl{\Or}{}}
\newcommand{\intom}{\intl{\Om}{}}
\newcommand{\intorb}{\intl{\partial\Or}{}}
\newcommand{\vT}{\makebf{T}}
\newcommand{\vt}{\makebf{t}}
\newcommand{\vsig}{\makebf{\sigma}}
\newcommand{\vSi}{\vS^{iso}}
\newcommand{\vSa}{\vS^{aniso}}
\newcommand{\vSf}{\vS^{act}}
\newcommand{\vPi}{\vP^{iso}}
\newcommand{\vPa}{\vP^{aniso}}
\newcommand{\vPf}{\vP^{act}}
\newcommand{\Sfun}{\makebf{\Upsilon}}
\newcommand{\force}{\vG}
\DeclareMathOperator{\divergence}{\nabla}
\DeclareMathOperator{\tr}{tr}
\DeclareMathOperator{\supp}{supp}
\DeclareMathOperator{\grad}{grad}
\renewcommand{\div}[1]{\divergence\cdot\left(#1\right)}
\newcommand{\m}[1]{\ensuremath{\left(\begin{matrix}#1\end{matrix}\right)}}
\newcommand{\sumnk}{\suml{i\in{\Nk}}{}}
\newcommand{\sumgp}{\suml{p=1}{G}}
\newcommand{\jmp}{J^m_p}
\newcommand{\re}[1]{\stackrel{\eqref{#1}}{=}}
\newcommand{\sumvk}{\suml{m\in V_k}{}}
\newcommand{\pmp}{X^m_p}%\Phi_m(X_p)
\usepackage{mathtools}
\DeclarePairedDelimiter{\ceil}{\lceil}{\rceil}
\newcommand{\Dp}[1]{\nabla\varphi_{#1}(X)}
\newcommand{\dpk}{\Dp{k}}
\newcommand{\dpi}{\Dp{i}}
\newcommand{\dpj}{\Dp{j}}
% Displacement basis ansatz functions
\newcommand{\bD}{N}
% Pressure basis ansatz functions
\newcommand{\bP}{\tilde{N}}
\newcommand{\dNkmp}{\bD^\nabla_{l(k,m)}(X_p)}
\newcommand{\bPkmp}{\bP_{l(k,m)}(X_p)}
\newcommand{\lfo}{\lambda_f^{opt}}

\allowdisplaybreaks
\begin{document}

Offene Fragen
\begin{itemize}
  \item Formulierung von diskretisierter ODE mit NB als explizite ODE (DAE-Index, Umformung analytisch etc)?
  \item Umformung in System erster Ordnung VOR oder NACH FEM-Ortsdiskretisierung?
  Im moment NACH ortsdiskretisierung (geplant?!?)
\end{itemize}

As perspective, we will work on the \e{reference configuration} regarding all quantities and relate to the \e{current configuration} only where necessary.
The quantities referred to in the reference configuration are upper-case letters and we denote quantities in the current configuration by lower-case letters.
Bold face is used to indicate vectorial quantities.

We model the movement/deformation of a continuous muscle over time, whose shape corresponds to the reference domain $\Or\subset\R^3$.
The movement of each particle $X\in\R^3$ in the body is described by a \e{motion} $\chi(X,t)\in\R^3$, which gives the position of each $X\in\R^3$ at time
$t\in[0,T]$ with $\Or = \chi(\Or,0)$.

We define the quantities
\begin{align}
	\text{velocity field} && \vV(X,t) &:= \d{\chi}{t}(X,T),\\
	\text{acceleration field} && \vA(X,t) &:= \d{\vV}{t}(X,t) = \dd{\chi}{t}(X,t),\\
	\text{deformation gradient} && \vF(X,t) &:= \d{\chi}{X}(X,t) = \m{\nabla \chi_1 & \nabla \chi_2 & \nabla \chi_3}^T \in \R^{3\times 3},\\
	\text{right cauchy strain tensor} && \vC(X,t) &:= \vF(X,t)^T\vF(X,t)\\
	\text{volume ratio} && J(X,t) &:= \det(\vF(X,t)).
\end{align}

\section{Derivation of governing equations}
Assuming a mass $\rho_0(X)$ at each point we define the total momentum
\[
 	\vL(t) := \intor \rho_0(X)\vV(X,t) dX,
\] 
for which we postulate the balance equation
\begin{align}
	\force(t) \stackrel{!}{=} \d{\vL}{t}(t) = \vL'(t) = \intor \rho_0(X)\d{\vV}{t}(X,t) dX = \intor \rho_0(X)\vA(X,t) dX,\label{def:bal_eq}
\end{align}
given \e{resultant forces} $\force(t)$.

\subsection{Structure of force}
We assume the forces $\force(t)$ to be composed of two different sources: Forces on the boundary and body forces.
The body forces $\vB(X,t)$ measures the force per unit reference volume on $X$ at time $t$.
These forces are self-weight or gravity, for example.
Further, we assume to have \e{traction vectors} $\vT(X,t,N)$ (first Piola-Kirchhoff traction vector) that indicates the force working per unit surface area with normal $N$ 
at the point $X\in\partial\Or$ at time $t$.
Then the resultant forces are given as
\begin{align}
	\force(t) := \intorb \vT(X,t,N)dN + \intor \vB(X,t)dX \label{def:force}
\end{align}

According to \e{Cauchy's stress theorem}, we can express tractions as tensor product
\begin{align}
	\vT(X,t,N) &= \vP(X,t)N,\label{def:T}
\end{align}
where $\vP$ denotes the \e{first Piola-Kirchhoff stress tensor}.
With this we have, following Gauss integral theorem,
\begin{align}
	\intorb \vT(X,t,N)dN &= \intorb \vP(X,t)NdN = \intor \divergence\vP(X,t)dX.\label{eq:div_form_traction}
\end{align}
Using representation \eqref{eq:div_form_traction} in the force composition 
\eqref{def:force} yields the following form of the balance equation \eqref{def:bal_eq}:
\begin{align}
	\intor \rho_0(X)\d{\vV}{t}(X,t) dX &= \intor \divergence\vP(X,t) + \vB(X,t) dX
\end{align}
As this balance must also be satisfied for each subvolume $\Omega_t\subset\Or$, we actually obtain a pointwise or local form as
\begin{align}
	\rho_0(X)\d{\vV}{t}(X,t) &= \divergence\vP(X,t) + \vB(X,t) \qquad \fo X\in\Or\label{def:maineq}
\end{align}

\subsection{Stress tensor definition}
We will actually assume
\begin{align}
	\vP(X,t) &:= \vF(X,t)\vS(X,t),
\end{align}
where $\vS(X,t)$ stands for the \e{second Piola-Kirchhoff stress tensor}.
Now, most generally one makes the assumption that any stress tensor depends on the location and the current deformation, i.e.
\[
	\vS(X,t) = \tilde{\Sfun}(\vF(X,t),X),\quad \hat{\Sfun}:\R^{3\times 3} \to \R^{3\times 3}.
\]
We assume \e{homogeneous} material, which means that a direct spatial dependence is not given and we thus omit the argument.
Further, due to the principle of material frame-indifference \cite[p.198]{Holzapfel2000}, $\vS$
may actually only depend on the rotation-invariant part of $\vF$.
For any fixed $X,t$ we can decompose $\vF = \vR\vU$ (\cite[p.85]{Holzapfel2000}) with a
\e{rotation} part $\vR$ with $\vR^T\vR = \vI$ and \e{stretch} part $\vU = \vU^T$.
Thus $\vS$ may only depend on $\vU$; since
\[
	\vC = \vF^T\vF = (\vR\vU)^T\vR\vU = \vU^T\vR^T\vR\vU = \vU\vU = \vU^2, 
\]
we will assume that
\[
	\vS(X,t) = \Sfun(\vC(X,t))
\]
for suitable tensor function $\Sfun$.
We also introduce the concept of a \e{fibre direction} $\va_0(X) \in\R^3$, where we will omit the argument $X$ in the following.
Then, for \e{hyperelastic and transversely isotropic} material we have the definition
\[
	\vS(X,t) = \Sfun(\vC(X,t)) := 2\d{\Psi}{\vC}(\vC(X,t),\va_0(X)),
\]
where $\Psi$ is called \e{strain-energy} function, see \cite[p.207]{Holzapfel2000}.

\subsubsection{Invariants}
We further have for any tensor $\vA$ with eigenvalues $\lambda_1,\lambda_2,\lambda_3$ the \e{invariants}
\begin{align}
	I_1(\vA) &:= \tr\vA = \lambda_1 + \lambda_2 + \lambda_3\label{def:I1}\\
	I_2(\vA) &:= \frac{1}{2}\left((\tr \vA)^2 - \tr\vA^2\right) = \lambda_1\lambda_2 + \lambda_1\lambda_3 + \lambda_2\lambda_3\\
	I_3(\vA) &:= \det\vA = \lambda_1\lambda_2\lambda_3\label{def:I3}\\
	I_4(\vA,\vv) &:= \vv\cdot\vA\vv = \vA : (\vv \otimes \vv)\label{def:I4} =: \lambda^2_{\vv}\\
	I_5(\vA,\vv) &:= \vv\cdot\vA^2\vv\label{def:I5}
\end{align}
with derivatives for \e{symmetric} $\vA$
\begin{align}
	\d{I_1}{\vA}(\vA) &:= \d{\tr\vA}{\vA}(\vA) = \d{\vI:\vA}{\vA}(\vA) = \vI\\
	\d{I_2}{\vA}(\vA) &:= I_1(\vA)\vI - \vA\\
	\d{I_3}{\vA}(\vA) &:= I_3\vA^{-1}\\
	\d{I_4}{\vA}(\vA) &:= \vv \otimes \vv\label{def:dI4}\\
	\d{I_5}{\vA}(\vA) &:= \vv \otimes \vA\vv + \vv\vA \otimes \vv,
\end{align}
see \cite[p.216/p.268]{Holzapfel2000}.

Now, $\Psi$ can actually be formulated using the invariants \eqref{def:I1}-\eqref{def:I5} as
\[
	\Psi(\vC) = \Psi(I_1(\vC),I_2(\vC),I_3(\vC),I_4(\vC,\va_0),I_5(\vC,\va_0)).
\]
This gives the general form
\begin{align}
	\vS(X,t) &= 2\d{\Psi}{\vC}(I_1,I_2,I_3,I_4,I_5)\\
		&= \d{\Psi}{I_1}\d{I_1}{\vC} + \d{\Psi}{I_2}\d{I_2}{\vC} + \d{\Psi}{I_3}\d{I_3}{\vC} + \d{\Psi}{I_4}\d{I_4}{\vC} + \d{\Psi}{I_5}\d{I_5}{\vC}\\
		&= 2\Biggl[\left(\d{\Psi}{I_1} + I_1\d{\Psi}{I_2}\right)\vI - \d{\Psi}{I_2}\vC + I_3\d{\Psi}{I_3}\vC^{-1}\\
		&\quad+ \d{\Psi}{I_4}\va_0\otimes\va_0 + \d{\Psi}{I_5}(\va_0\otimes\vA\va_0 + \va_0\vA\otimes\va_0)\Biggr].
\end{align}

Further, we consider \e{incompressible} material, which is expressed by the condition $J(X,t) = 1 \fo X,t$.
This leads to the constant third invariant $I_3 \equiv 1$, where an additional Lagrange-multiplier $p$ is introduced
to satisfy the condition via
\begin{align}
	\vS(X,t) = p\vC^{-1}(X,t) + 2\d{\Psi}{\vC}(\vC(X,t))
\end{align}

In the following, we will specify different $\Psi$ to create an additive split as 
\begin{align}
	\vS(X,t) = p\vC^{-1}(X,t) + \vSi(X,t) + \vSa(X,t) + \vSf(X,t)\label{def:S_split}
\end{align}

\subsubsection{Isotropic stress tensor $\vSi$}
Now, for the isotropic part we use
\[
	\Psi(I_1,I_2,I_3) = c_{10}(I_1-3) + c_{01}(I_2-3),
\]
which gives
\begin{align}
	\vSi(X,t) &= 2(c_{10} + I_1c_{01})\vI - 2c_{01}\vC(X,t)\\
	\vPi(X,t) &= 2(c_{10} + I_1c_{01})\vF(X,t) - 2c_{01}\vF(X,t)\vC(X,t)
\end{align}

\subsubsection{Anisotropic stress tensor $\vSa$}
Further, we introduce the stretch $\lambda_f$ in fibre direction $\va_0(X)$ as
\begin{align*}
	\lambda_f(X,t) &:= \sqrt{I_4(\vC(X,t),\va_0(X))}
\end{align*}
Now, according to \cite{Markert2005}, we employ
\begin{align*}
	\Psi(\lambda_f) := \sumi \frac{b_i}{d_i}(\lambda_f^{d_i} - 1) - b_i\ln\lambda_f
\end{align*}
with $n=1$ summands, so that
\begin{align}
	\d{\Psi}{\vC}(\lambda_f) &= \d{\Psi}{\lambda_f}(\lambda_f)\d{\lambda_f}{I_4}(I_4)\d{I_4}{\vC}(\vC,\va_0),\label{eq:dpsidC}
\end{align}
with
\begin{align*}
		 \d{\Psi}{\lambda_f}(\lambda_f) &= \frac{b_1(d_1-1)}{d_1}\lambda_f^{d-1} - b_1\lambda_f^{-1},\\
		 \d{\lambda_f}{I_4}(I_4) &= \frac{1}{2\sqrt{I_4}} = \frac{1}{2\lambda_f}.
\end{align*}
With \eqref{def:dI4} and the above we get
\begin{align*}
	2\d{\Psi}{\vC}(\lambda_f) &= 2\left(\frac{b_1(d_1-1)}{d_1}\lambda_f^{d-1} - b_1\lambda_f^{-1}\right) \frac{1}{2\lambda_f}\va_0\otimes\va_0\\
	&= \left(\frac{b_1(d_1-1)}{d_1}\lambda_f^{d-2} - b_1\lambda_f^{-2}\right) \va_0\otimes\va_0,
\end{align*}
leading to
\begin{align}
	\vSa(X,t) &= \left(\frac{b_1(d_1-1)}{d_1}\lambda_f^{d-2}(X,t) - b_1\lambda_f^{-2}(X,t)\right) \va_0(X)\otimes\va_0(X)\\
	\vPa(X,t) &= \left(\frac{b_1(d_1-1)}{d_1}\lambda_f^{d-2}(X,t) - b_1\lambda_f^{-2}(X,t)\right) \vF(X,t)\va_0(X)\otimes\va_0(X)
\end{align}

\subsubsection{Active force tensor $\vSf$}
Here we define
\begin{align*}
	\Psi(\lambda_f) &:= \intl{0}{\lambda_f} p^{act}(s)ds,\\
	p^{act}(\lambda_f) &:= p^{max}\gamma(\alpha,\lambda_f,\lambda_f'),
\end{align*}
with suitable constant $p^{max}$ and function $\gamma$.
Similar to \eqref{eq:dpsidC} we obtain
\begin{align}
	\d{\Psi}{\vC}(\lambda_f) &= \frac{p^{max}}{2\lambda_f}\gamma(\alpha,\lambda_f,\lambda_f')\va_0\otimes\va_0
\end{align}
This gives
\begin{align}
	\vSf(X,t) &= \frac{p^{max}}{\lambda_f(X,t)}\gamma\left(\alpha,\lambda_f(X,t),\d{\lambda_f}{t}(X,t)\right)\va_0(X)\otimes\va_0(X)\\
	\vPf(X,t) &= \frac{p^{max}}{\lambda_f(X,t)}\gamma\left(\alpha,\lambda_f(X,t),\d{\lambda_f}{t}(X,t)\right)\vF(X,t)\va_0(X)\otimes\va_0(X)
\end{align}

\subsubsection{Overall stress tensor}
Adding the different parts together as given in \eqref{def:S_split} now gives
\begin{align}
	\vS(X,t) &= 2(c_{10} + I_1c_{01})\vI - 2c_{01}\vC(X,t)\\
			 &\quad+\Biggl[\left(\frac{b_1(d_1-1)}{d_1}\lambda_f^{d-2}(X,t) - b_1\lambda_f^{-2}(X,t)\right)\nonumber\\
			 &\quad+\frac{p^{max}}{\lambda_f(X,t)}\gamma\left(\alpha,\lambda_f(X,t),\d{\lambda_f}{t}(X,t)\right)\Biggr]\va_0(X)\otimes\va_0(X)\nonumber\\
	g(\lambda_f)&:= \left(\frac{b_1(d_1-1)}{d_1}\lambda_f^{d-2} - b_1\lambda_f^{-2}\right)+\frac{p^{max}}{\lambda_f}\gamma\left(\alpha,\lambda_f,\d{\lambda_f}{t}\right)\\			 
	\vP(X,t) &= \vF(X,t)\vS(X,t)\label{def:completeP}\\
			 &= 2(c_{10} + I_1(\vC(X,t))c_{01})\vF(X,t) - 2c_{01}\vF(X,t)\vC(X,t)\nonumber\\
			 &\quad+g(\lambda_f(X,t))\vF(X,t)\va_0(X)\otimes\va_0(X)\nonumber
\end{align}

% \subsubsection{Detour}
% Prescribing $\vP$ will yield the behaviour of the overall system.
% The equation $\eqref{def:T}$ also holds for values in the current configuration, i.e.
% \[
% 	\vt(x,t,n) = \vsig(x,t)n,
% \]
% where $\vsig$ is the symmetric \e{Cauchy stress tensor}.

\section{FEM discretization}

\subsection{Weak form and linear ansatz}
In the following we will use $\varphi/\psi$ to denote node ansatz functions with linear/quadratic master basis functions, respectively.
We define the spaces
\begin{align*}
	\H^1 &:= \langle \psi_1,\ldots,\psi_{N_P} \rangle \subset C^1(\Or)\\
	\H^2 &:= \langle \varphi_1,\ldots,\varphi_N \rangle \subset C^2(\Or)
\end{align*} 
For discretization we will use Taylor-Hood elements, which means $\chi\in\H^2, p\in\H^1$ via
\begin{align}
	\chi(X,t) &= \sumi \vc_i(t)\varphi_i(X), && \vc_i(t):[0,T] \to\R^3.\label{def:chi}\\
	p(X,t) &= \sumiP d_i(t)\psi_i(X), && d_i(t):[0,T] \to\R.
\end{align}
This means we have quadratic ansatz functions for positions, velocity and linear functions for pressure.
With \eqref{def:chi} we deduce the discrete quantities
\begin{align*}
	\vV(X,t) &= \d{\chi}{t}(X,t) = \sumi \vc_i'(t)\varphi_i(X)\\
	\vA(X,t) &= \d{\vV}{t}(X,t) = \sumi \vc_i''(t)\varphi_i(X)\\
	\vF(X,t) &= \d{\chi}{X}(X,t) = \m{\grad \chi_1(X,t)\\ \grad \chi_2(X,t)\\ \grad \chi_3(X,t)}
		   = \sumi\m{c_{i,1}\d{\varphi_i}{X_1} & \dots & c_{i,1}\d{\varphi_i}{X_3}\\
		   				\vdots & \ddots & \vdots\\
		   				c_{i,3}\d{\varphi_i}{X_1} & \dots & c_{i,3}\d{\varphi_i}{X_3}}\\
		   &=\sumi \vc_i(t)\cdot \grad\varphi_i(X) = \sumi \vc_i(t)\otimes \dpi.
\end{align*}

Now, equation \eqref{def:maineq} (without body forces) and the incompressibility constraint give the system 
\begin{align}
	\rho_0(X)\d{\vV}{t}(X,t) &= \divergence\vP(X,t) && \fo X\in\Or,\label{def:maineq_nobodyforce}\\
	J(X,t) &= 1 && \fo X\in\Or,\label{def:maineq_cond}
\end{align}
for which we have the spatial weak form
\begin{align}
	\intor\rho_0(X)\d{\vV}{t}(X,t)\varphi(X)dX &= \intor\divergence\vP(X,t)\varphi(X)dX &&\fo \varphi\in\H^2\\
	\intor (J(X,t)-1)\psi(X)dX &= 0 && \fo \psi\in\H^1
\end{align}
The right hand side of \eqref{def:maineq_nobodyforce} can be transformed as
\begin{align*}
	\intor \div{\vP(X,t)}\varphi(X) dX &= \intor \div{\vP(X,t)\varphi(X)} - \vP(X,t)\divergence\varphi(X) dX\\
		 &= \underbrace{\intorb \vP(X,t)\varphi(X) dN}_{=0} - \intor \vP(X,t)\divergence\varphi(X)dX\\
		 &= \intor -\vP(X,t)\divergence\varphi(X) dX.
\end{align*}
Now, to satisfy equations \eqref{def:maineq_nobodyforce} and \eqref{def:maineq_cond}, they must hold true for every node basis function. 
Consequently, we obtain a system of $3N+N_P$ equations
\begin{align}
	\intor \rho_0(X)\d{\vV}{t}(X,t)\varphi_k(X)dX + \intor\vP(X,t)\dpk dX &= \vnull, && \fo k=1\ldots N,\label{def:weakform}\\
	\intor (J(X,t)-1)\psi_k(X)dX &= 0, && \fo k=1\ldots N_P.\label{def:weakform_nb}
\end{align}

\subsubsection{Integrals of main equation parts}
In the following we fix a $k\in\{1,\ldots, N\}$.
Then, by \eqref{eq:fphidx}, the first summand of \eqref{def:weakform} reads as
\begin{align}
	\intor \rho_0(X)\d{\vV}{t}(X,t)\varphi_k(X)dX\nonumber
		&\re{eq:fphidx} \sumvk\sumgp w_p \rho_0(\pmp)\d{\vV}{t}(\pmp,t)\Nkmp\jmp\nonumber\\
		&= \sumvk\sumgp w_p \rho_0(\pmp)\sumi \vc_i''(t)\varphi_i(\pmp) \Nkmp\jmp\nonumber\\
		&= \sumi \vc_i''(t)\sumvk\sumgp w_p \rho_0(\pmp)N_{l(i,m)}(X_p) \Nkmp\jmp\label{def:discreteVphidX}
\end{align}
Similarly, the second summand of \eqref{def:weakform} writes as
\begin{align}
		\intor\vP(X,t)\dpk dX &\re{eq:fgradphidx} \sumvk\sumgp w_p\vP(\pmp,t)\dNkmp\jmp.\label{def:discretePgradphidX}
\end{align}
Finally, for $k\in\{1\ldots N_P\}$, condition \eqref{def:maineq_cond} formulates to
\begin{align}
	\intor (J(X,t)-1)\psi_k(X)dX &= \sumvk\sumgp w_p (\det\vF(\pmp,t)-1) \Nkmp\jmp\label{eq:weak_incomp_condition}\\
	&= \sumvk\sumgp w_p \left(\det\left(\sumi \vc_i(t)\otimes \divergence\varphi_i(\pmp)\right)-1\right) \Nkmp\jmp.\nonumber
\end{align}

\subsubsection{Local tensor evaluations}
The most elaborative part is to evaluate the stress tensor $\vP$ at all volumes and Gauss points.
We have from \eqref{def:completeP} that
\begin{align*}
		\vP(\pmp,t) &= p(\pmp,t)\vF^{-T}(\pmp,t) + 2(c_{10} + I_1(\vC(\pmp,t))c_{01})\vF(\pmp,t)\\
			 &- 2c_{01}\vF(\pmp,t)\vC(\pmp,t)+g(\lambda_f(\pmp,t))\vF(\pmp,t)\va_0(\pmp)\otimes\va_0(\pmp).
\end{align*}
In the following we derive representations of the various components of $\vP$, where we assume $X\in\Or$.
\begin{align*}
	\vF(\pmp,t) &= \sumi \vc_i(t)\otimes \dNimp,\\
	I_1(\vC(X,t)) &= \tr\vC(X,t) = \tr(\vF(X,t)^T\vF(X,t)) = \vF(X,t) : \vF(X,t)\\
	&= \sumi \vc_i(t)\otimes \dpi : \sumi \vc_i(t)\otimes \dpi\\
	&= \sumi\sumj \left(\vc_i(t)\otimes \dpi\right) : (\vc_j(t)\otimes \dpj)\\
	&= \sumi\sumj (\vc_i(t) \cdot \vc_j(t))(\dpi \cdot \dpj),\\
	\Rightarrow I_1(\vC(\pmp,t)) &= \sumi\sumj (\vc_i(t) \cdot \vc_j(t))(\dNimp \cdot \dNjmp),\\
\end{align*}
\begin{align*}
	\vF(X,t)^T\vF(X,t) &= \sumi\sumj \bigl(\vc_i(t)\otimes\dpi\bigr)^T\vc_j(t)\otimes\dpj\\
	 &= \sumi\sumj \bigl(\dpi\otimes\vc_i(t)\bigr)\vc_j(t)\otimes\dpj\\
	 &=\sumi\sumj (\vc_i(t)\cdot \vc_j(t))\dpi\otimes\dpj,\\
	\lambda_f(X,t)^2 &=I_4(\vC(X,t),\va_0(X)) = \vC(X,t):(\va_0(X)\otimes \va_0(X))\\
			&=  \vF(X,t)^T\vF(X,t) : (\va_0(X)\otimes \va_0(X))\\
			&=  \sumi\sumj (\vc_i(t)\cdot \vc_j(t))\bigl(\dpi\otimes\dpj\bigr) : (\va_0(X)\otimes \va_0(X))\\
			&=  \sumi\sumj (\vc_i(t)\cdot \vc_j(t))\left(\dpi\cdot \va_0(X)\right)(\dpj\cdot \va_0(X)),\\
	\Rightarrow \lambda_f(\pmp,t) &=
    		\left(\sumi\sumj (\vc_i(t)\cdot \vc_j(t))\left(\dNimp\cdot \va_0(\pmp)\right)(\dNjmp\cdot \va_0(\pmp))\right)^\frac{1}{2},\\
\end{align*}
\begin{align*}
    \vF(X,t)(\va_0(X)\otimes\va_0(X)) &= \sumi (\vc_i(t)\otimes \dpi)(\va_0(X)\otimes\va_0(X))\\
    &= \sumi (\vc_i(t)\otimes \va_0(X))(\dpi\cdot\va_0(X)),\\
    \Rightarrow \vF(\pmp,t)(\va_0(\pmp)\otimes\va_0(\pmp)) 
    	&= \sumi (\vc_i(t)\otimes \va_0(\pmp))(\dNimp\cdot\va_0(\pmp)).\\
\end{align*}

\begin{align*}
	\d{\lambda_f}{t}(X,t) &= \frac{1}{2\lambda_f(X,t)} \suml{i,j}{N} (\vc'_i(t)\cdot \vc_j(t) + \vc_i(t)\cdot \vc'_j(t))\left(\dpi\cdot a_0(X)\right)(\dpj\cdot a_0(X))\\
		&= \frac{1}{\lambda_f(X,t)}\suml{i,j}{N} (\vc_i(t)\cdot \vc'_j(t))\left(\dpi\cdot a_0(X)\right)(\dpj\cdot a_0(X))
\end{align*}
%     \vF(X,t)\vF(X,t)^T\vF(X,t) &= \sumk\vc_k(t)\otimes\dpk\sumi\sumj (\vc_i(t)\cdot \vc_j(t))\dpi\otimes\dpj\nonumber\\
%     	&= \suml{i,j,k}{N}(\vc_i(t)\cdot \vc_j(t))(\vc_k(t)\otimes\dpk)\dpi\otimes\dpj\nonumber\\
%     	&= \suml{i,j,k}{N}(\vc_i(t)\cdot \vc_j(t))(\dpk\cdot\dpi)(\vc_k(t)\otimes\dpj)\nonumber\\

% 	\vF(\pmp,t)\vC(\pmp,t) &= \vF(\pmp,t)\vF(\pmp,t)^T\vF(\pmp,t)\\
%     	&= \suml{i,j,k}{N}(\vc_i(t)\cdot \vc_j(t))(\divergence\varphi_k(\pmp)\cdot\divergence\varphi_i(\pmp))(\vc_k(t)\otimes\divergence\varphi_j(\pmp))\\
	%&= \suml{i,j,q\in E_m}{}(\vc_i(t)\cdot \vc_j(t))(\nabla N_{l(q,m)}(X_p)\cdot\nabla N_{l(i,m)}(X_p))(\vc_q(t)\otimes\nabla N_{l(j,m)}(X_p))\\
	%&= \suml{i,j\in E_m}{} (\vc_i(t)\cdot \vc_j(t))\left(\nabla N_{l(i,m)}(X_p)\cdot a_0(\pmp)\right)(\nabla N_{l(j,m)}(X_p))\cdot a_0(\pmp))\nonumber
	%&= \suml{i\in E_m}{} (\vc_i(t)\otimes \va_0(\pmp))(\nabla N_{l(i,m)}(X_p)\cdot\va_0(\pmp))\nonumber

\subsection{Discrete system}
\input{discrete}

\newpage
\section{Appendix, additional derivations}

\subsection{Calculation of $\nabla_{\Bc_m}\left\{\det \BF(X,t)\right\}$}
Calculation of the determinant:
\alg{
  \nabla_{\Bc_m}\left\{\det \BF(X,t)\right\}
  &=\nabla_{\Bc_m}\left\{\det \left( \sumlim{i=1}{N}{\Bc_i(t) \otimes \nabla \varphi_i(X)} \right) \right\} \nonumber\\
  &=\nabla_{\Bc_m}\left\{\det \left( \sumlim{i=1}{N}{
      \begin{bmatrix}
        c_{i1} \varphi_{i,1}  & c_{i1} \varphi_{i,2}  & c_{i1} \varphi_{i,3} \\
        c_{i2} \varphi_{i,1}  & c_{i2} \varphi_{i,2}  & c_{i2} \varphi_{i,3} \\
        c_{i3} \varphi_{i,1}  & c_{i3} \varphi_{i,2}  & c_{i3} \varphi_{i,3}
      \end{bmatrix}
    } \right) \right\} \nonumber\\
  &=\nabla_{\Bc_m}\Bigg\{
      \left( \sumlim{i=1}{N}{c_{i1}\varphi_{i,1}}\right) \left[
        \left( \sumlim{i=1}{N}{c_{i2}\varphi_{i,2}}\right) \left( \sumlim{i=1}{N}{c_{i3}\varphi_{i,3}}\right)
        - \left( \sumlim{i=1}{N}{c_{i2}\varphi_{i,3}}\right) \left( \sumlim{i=1}{N}{c_{i3}\varphi_{i,2}}\right)
      \right]
      \nonumber\\
      &\hspace{11.5mm} -\left( \sumlim{i=1}{N}{c_{i1}\varphi_{i,2}}\right) \left[
        \left( \sumlim{i=1}{N}{c_{i2}\varphi_{i,1}}\right) \left( \sumlim{i=1}{N}{c_{i3}\varphi_{i,3}}\right)
        - \left( \sumlim{i=1}{N}{c_{i2}\varphi_{i,3}}\right) \left( \sumlim{i=1}{N}{c_{i3}\varphi_{i,1}}\right)
      \right]
      \nonumber\\
      &\hspace{11.5mm} +\left( \sumlim{i=1}{N}{c_{i1}\varphi_{i,3}}\right) \left[
        \left( \sumlim{i=1}{N}{c_{i2}\varphi_{i,1}}\right) \left( \sumlim{i=1}{N}{c_{i3}\varphi_{i,2}}\right)
        - \left( \sumlim{i=1}{N}{c_{i2}\varphi_{i,2}}\right) \left( \sumlim{i=1}{N}{c_{i3}\varphi_{i,1}}\right)
      \right]
    \Bigg\}\nonumber\\
  &=\nabla_{\Bc_m}\Bigg\{
  \sumlim{i,j,k=1}{N}{c_{i1} \varphi_{i,1} \, c_{j2} \varphi_{j,2} \, c_{k3}\varphi_{k,3}}
    -\sumlim{i,j,k=1}{N}{c_{i1} \varphi_{i,1} \, c_{j2} \varphi_{j,3} \, c_{k3}\varphi_{k,2}}
    \nonumber\\
    &\hspace{11.5mm}-\sumlim{i,j,k=1}{N}{c_{i1} \varphi_{i,2} \, c_{j2} \varphi_{j,1} \, c_{k3}\varphi_{k,3}}
    +\sumlim{i,j,k=1}{N}{c_{i1} \varphi_{i,2} \, c_{j2} \varphi_{j,3} \, c_{k3}\varphi_{k,1}}
    \nonumber\\
    &\hspace{11.5mm}+\sumlim{i,j,k=1}{N}{c_{i1} \varphi_{i,3} \, c_{j2} \varphi_{j,1} \, c_{k3}\varphi_{k,2}}
    -\sumlim{i,j,k=1}{N}{c_{i1} \varphi_{i,3} \, c_{j2} \varphi_{j,2} \, c_{k3}\varphi_{k,1}} 
    \Bigg\}\nonumber\\
  &=\nabla_{\Bc_m}\Bigg\{
    \sumlim{i,j,k=1}{N}{
    c_{i1} c_{j2} c_{k3} (
      \varphi_{i,1} \varphi_{j,2} \varphi_{k,3}-\varphi_{i,1} \varphi_{j,3} \varphi_{k,2}-\varphi_{i,2} \varphi_{j,1} \varphi_{k,3}
      }\nonumber\\
      &\hspace{20mm}+\varphi_{i,2} \varphi_{j,3} \varphi_{k,1}+\varphi_{i,3} \varphi_{j,1} \varphi_{k,2}
      -\varphi_{i,3} \varphi_{j,2} \varphi_{k,1} )
    \Bigg\}\nonumber\\
  &=\nabla_{\Bc_m}\Bigg\{
      \sumlim{i,j,k=1}{N}{
      c_{i1} c_{j2} c_{k3} 
      \det  \begin{bmatrix}
              \varphi_{i,1} & \varphi_{i,2} & \varphi_{i,3} \\
              \varphi_{j,1} & \varphi_{j,2} & \varphi_{j,3} \\
              \varphi_{k,1} & \varphi_{k,2} & \varphi_{k,3}
            \end{bmatrix}
      }
    \Bigg\} \nonumber\\
  &=\nabla_{\Bc_m}\Bigg\{
      \sumlim{i,j,k=1}{N}{
      c_{i1}(t) c_{j2}(t) c_{k3}(t)
      \det \begin{bmatrix}
              \nabla \varphi_{i}(X) \\
              \nabla \varphi_{j}(X) \\
              \nabla \varphi_{k}(X)
            \end{bmatrix}
      }
    \Bigg\}
}
%
%-------
\newpage
%-------
%
Calculation of the gradient:
\alg{
  \nabla_{\Bc_m}\left\{\det \BF(X,t)\right\}
  &=\nabla_{\Bc_m}\Bigg\{
      \sumlim{i,j,k=1}{N}{
      c_{i1}(t) c_{j2}(t) c_{k3}(t)
      \det \begin{bmatrix}
              \nabla \varphi_{i}(X) \\
              \nabla \varphi_{j}(X) \\
              \nabla \varphi_{k}(X)
            \end{bmatrix}
      }
    \Bigg\} \nonumber\\
  &=\sumlim{i,j,k=1}{N}{\left(
      \begin{bmatrix}
        \delby{c_{i1}}{c_{m1}}c_{j2}c_{k3} + c_{i1} \cancel{\delby{c_{j2}}{c_{m1}}} c_{k3} 
          + c_{i1}c_{j2}\cancel{\delby{c_{k3}}{c_{m1}}} \\
        \cancel{\delby{c_{i1}}{c_{m2}}}c_{j2}c_{k3} + c_{i1} \delby{c_{j2}}{c_{m2}} c_{k3} 
          + c_{i1}c_{j2}\cancel{\delby{c_{k3}}{c_{m2}}} \\
        \cancel{\delby{c_{i1}}{c_{m3}}}c_{j2}c_{k3} + c_{i1} \cancel{\delby{c_{j2}}{c_{m3}}} c_{k3} 
        + c_{i1}c_{j2}\delby{c_{k3}}{c_{m3}}
      \end{bmatrix}
      \det \begin{bmatrix}
              \nabla \varphi_{i}(X) \\
              \nabla \varphi_{j}(X) \\
              \nabla \varphi_{k}(X)
            \end{bmatrix}
      \right)}
      \nonumber\\
  &=\sumlim{i,j,k=1}{N}{\left(
      \begin{bmatrix}
        \delta_{im}c_{j2}c_{k3}\\
        c_{i1}\delta_{jm}c_{k3}\\
        c_{i1}c_{j2}\delta_{km}
      \end{bmatrix}
      \det \begin{bmatrix}
              \nabla \varphi_{i}(X) \\
              \nabla \varphi_{j}(X) \\
              \nabla \varphi_{k}(X)
            \end{bmatrix}
      \right)}
      \nonumber\\
  &=\begin{pmatrix}
      \sumlim{j,k=1}{N}{
        c_{j2}c_{k3} \det \begin{bmatrix}
                            \nabla \varphi_{m}(X) \\
                            \nabla \varphi_{j}(X) \\
                            \nabla \varphi_{k}(X)
                          \end{bmatrix}
      } \\
      \sumlim{i,k=1}{N}{
        c_{i1}c_{k3} \det \begin{bmatrix}
                            \nabla \varphi_{i}(X) \\
                            \nabla \varphi_{m}(X) \\
                            \nabla \varphi_{k}(X)
                          \end{bmatrix}
      } \\
      \sumlim{i,j=1}{N}{
        c_{i1}c_{j2} \det \begin{bmatrix}
                            \nabla \varphi_{i}(X) \\
                            \nabla \varphi_{j}(X) \\
                            \nabla \varphi_{m}(X)
                          \end{bmatrix}
      }            
    \end{pmatrix}
    \nonumber\\
  &=\sumlim{i,j=1}{N}{
      \begin{pmatrix}
        c_{i2}c_{j3}\det \begin{bmatrix}
                            \nabla \varphi_{m}(X) \\
                            \nabla \varphi_{i}(X) \\
                            \nabla \varphi_{j}(X)
                          \end{bmatrix}
        \\
        c_{i1}c_{j3}\det \begin{bmatrix}
                            \nabla \varphi_{i}(X) \\
                            \nabla \varphi_{m}(X) \\
                            \nabla \varphi_{j}(X)
                          \end{bmatrix}
        \\
        c_{i1}c_{j2}\det \begin{bmatrix}
                            \nabla \varphi_{i}(X) \\
                            \nabla \varphi_{j}(X) \\
                            \nabla \varphi_{m}(X)
                          \end{bmatrix}
      \end{pmatrix}
    }
    =:\ \begin{pmatrix}
          \nabla_{c_{m1}}\left\{\det \BF(X,t)\right\} \\
          \nabla_{c_{m2}}\left\{\det \BF(X,t)\right\} \\
          \nabla_{c_{m3}}\left\{\det \BF(X,t)\right\}                  
        \end{pmatrix}
}
%
%-------
\newpage
%-------
%
It holds that:
\beq{}
  \boxed{
  \det(\BS+\BU) = \det \BS + \det \BS (\BS^{T-1}:\BU) \,.
  }
\eeq
Furthermore, for the inverse of a matrix/tensor, it holds
\beq{}
  \BA^{-1} = (\det \BA)^{-1} \textnormal{adj}\BA
  \Longrightarrow
  \BA^{T-1} = (\det \BA)^{-1} \textnormal{cof}\BA \,,
\eeq
where
\beq{}
  \textnormal{cof}\BA = \textnormal{cof}
    \begin{pmatrix}
      A_{11}  & A_{12}  & A_{13} \\
      A_{21}  & A_{22}  & A_{23} \\
      A_{31}  & A_{32}  & A_{33}
    \end{pmatrix}
    = \begin{pmatrix}
        A_{22}A_{33}-A_{23}A_{32} & A_{23}A_{31}-A_{21}A_{33} & A_{21}A_{32}-A_{22}A_{31} \\
        A_{13}A_{32}-A_{12}A_{33} & A_{11}A_{33}-A_{13}A_{31} & A_{12}A_{31}-A_{11}A_{32} \\
        A_{12}A_{23}-A_{13}A_{22} & A_{13}A_{21}-A_{11}A_{23} & A_{11}A_{22}-A_{12}A_{21}
      \end{pmatrix} \,.
\eeq
"{\bf \underline{Lemma}}": \par
For $\BS:=\BF$ and $\BU:=\BU^m_l$, $l=1,2,3$, with
\beq{}
  \BU^m_1:= \begin{pmatrix}
            ~   & \nabla \varphi_m  & ~ \\
            0   & 0                 & 0 \\
            0   & 0                 & 0
          \end{pmatrix},\,
  \BU^m_2:= \begin{pmatrix}
            0   & 0                 & 0 \\
            ~   & \nabla \varphi_m  & ~ \\            
            0   & 0                 & 0
          \end{pmatrix},\,
  \BU^m_3:= \begin{pmatrix}
            0   & 0                 & 0 \\
            0   & 0                 & 0 \\
            ~   & \nabla \varphi_m  & ~            
          \end{pmatrix},          
\eeq
it holds that
\beq{}
  \boxed{
  \nabla_{c_{ml}}\left\{\det \BF(X,t)\right\} = \det \BF \left(\BF^{T-1}:\BU^m_l\right) \,.
  }
\eeq
"{\bf \underline{Proof}}": \par
Setting $l:=1$ yields:
\alg{
  \det \BF \left(\BF^{T-1}:\BU^m_1\right)
  &= \cancel{\det \BF (\det \BF)^{-1}} \textnormal{cof}\BF : \BU^m_1 \nonumber\\
  &= \textnormal{cof}\BF 
        \begin{pmatrix}
           ~   & \nabla \varphi_m  & ~ \\
           0   & 0                 & 0 \\
           0   & 0                 & 0
        \end{pmatrix}
    \nonumber\\
  &= \varphi_{m,1} (F_{22}F_{33}-F_{23}F_{32}) +
      \varphi_{m,2} (F_{23}F_{31}-F_{21}F_{33}) +
      \varphi_{m,3} (F_{21}F_{32}-F_{22}F_{31})
    \nonumber\\
  &= \varphi_{m,1} \left(
        \sumlim{i,j=1}{N}{c_{i2}\varphi_{i,2}c_{j3}\varphi_{j,3}}
        - \sumlim{i,j=1}{N}{c_{i2}\varphi_{i,3}c_{j3}\varphi_{j,2}}
      \right) \nonumber\\
    &\hspace{6mm}+\varphi_{m,2} \left(
        \sumlim{i,j=1}{N}{c_{i2}\varphi_{i,3}c_{j3}\varphi_{j,1}}
        - \sumlim{i,j=1}{N}{c_{i2}\varphi_{i,1}c_{j3}\varphi_{j,3}}
      \right)\nonumber\\
    &\hspace{6mm}+\varphi_{m,3} \left(
        \sumlim{i,j=1}{N}{c_{i2}\varphi_{i,1}c_{j3}\varphi_{j,2}}
        - \sumlim{i,j=1}{N}{c_{i2}\varphi_{i,2}c_{j3}\varphi_{j,1}}
      \right) \nonumber\\
  &= \sumlim{i,j=1}{N}{c_{i2}c_{j3}
      (
      \varphi_{m,1}\varphi_{i,2} \varphi_{j,3}-\varphi_{m,1}\varphi_{i,3} \varphi_{j,2}+\varphi_{m,2}\varphi_{i,3} \varphi_{j,1} 
      }\nonumber\\
      &\hspace{11mm}-\varphi_{m,2}\varphi_{i,1} \varphi_{j,3}+\varphi_{m,3}\varphi_{i,1} \varphi_{j,2} - \varphi_{m,3}\varphi_{i,2} \varphi_{j,1}
      ) \nonumber\\
  &= \sumlim{i,j=1}{N}{c_{i2}c_{j3} 
      \det \begin{bmatrix}
              \nabla \varphi_{m} \\
              \nabla \varphi_{i} \\
              \nabla \varphi_{j}           
            \end{bmatrix}
     }
  = \nabla_{c_{m1}}\left\{\det \BF(X,t)\right\} \,.
}



% \begin{align*}
% 	\intor \rho_0(X)\d{\vV}{t}(X,t)\varphi_k(X)dX
% 		&= \intor \rho_0(X)\sumi \vc_i''(t)\varphi_i(X)\varphi_k(X)dX\\
% 		&= \sumi \vc_i''(t) \intor \rho_0(X)\varphi_i(X)\varphi_k(X)dX\\
% 		&= \sumnk \vc_i''(t) \intor \rho_0(X)\varphi_i(X)\varphi_k(X)dX
% \end{align*}
% The second term of \eqref{def:weakform}, with \eqref{def:completeP}, reads as
% \begin{align*}
% 	&\intor\vP(X,t)\dpk dX\\ 
% 		=& \intor2(c_{10} + I_1(\vC(X,t))c_{01})\vF(X,t) - 2c_{01}\vF(X,t)\vC(X,t)\\
% 		&\quad+g(\lambda_f(X,t))\vF(X,t)\va_0(X)\otimes\va_0(X)\dpkdX\\
% 		=& \intor2(c_{10} + I_1(\vC(X,t))c_{01})\sumi (\vc_i(t)\otimes \dpi)\dpkdX\\
% 		&-2c_{01}\intor\suml{i,j,l}{N}(\vc_i(t)\cdot \vc_j(t))(\divergence\varphi_l(X)\cdot\dpi)(\vc_l(t)\otimes\dpj)\dpkdX\\
% 		&+\intor g(\lambda_f(X,t))\sumi (\vc_i(t)\cdot \va_0(X))(\dpi\cdot\va_0(X))\dpkdX\\
% 		=& \sumnk\vc_i(t)\intor2(c_{10} + I_1(\vC(X,t))c_{01})(\dpi\cdot\dpk)dX\\
% 		&-2c_{01}\suml{i,j,l\in\Nk}{}(\vc_i(t)\cdot \vc_j(t))\vc_l(t)
% 		\intor(\divergence\varphi_l(X)\cdot\dpi)(\dpj\cdot\dpk)dX\\
% 		&+\sumnk \intor g(\lambda_f(X,t))(\vc_i(t)\cdot \va_0(X))(\dpi\cdot\va_0(X))\dpkdX
% \end{align*}

\bibliographystyle{plain}
\bibliography{cbm_library}

\end{document}
