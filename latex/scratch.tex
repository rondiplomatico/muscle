%%This is a very basic article template.
%%There is just one section and two subsections.
\documentclass[a4paper,11pt]{article}
%\documentclass[a4paper,12pt]{article}
\usepackage[english]{babel}
\usepackage{amsmath}
\usepackage{amssymb}
\usepackage{amsthm}
\usepackage[makeroom]{cancel}

% Graphics
\usepackage{graphicx}
\graphicspath{{img/}}
\DeclareGraphicsExtensions{.eps,.png,.pdf,.jpg}

% Windows-only
\usepackage[utf8]{inputenc}
\usepackage{epstopdf}
%pdf latex -synctex=1 --shell-escape -interaction=nonstopmode --src-specials

%\usepackage{libertine}
%\usepackage[ttscale=.875]{libertine}

% Display
\usepackage[font={small,it}]{caption}
%\usepackage{multirow}
\usepackage{fancyvrb}
\usepackage{listings}

% Illustrations
%\usepackage{tikz}
%usetikzlibrary{arrows,automata}

% Algorithms
%\usepackage{algorithm}
%\usepackage{algpseudocode}

% For bibliography
\usepackage{url}

% \usepackage{setspace}
\usepackage{geometry}
\geometry{%
	top=25mm, bottom=30mm,
	left=20mm, right=20mm, twoside
}

\setlength{\parskip}{8pt}
\setlength{\parindent}{0cm}

\newlength{\myleftmargin}
\setlength{\myleftmargin}{-6ex}
\newlength{\fixboxwidth}
\setlength{\fixboxwidth}{\marginparwidth}
\addtolength{\fixboxwidth}{\myleftmargin}
\newcommand{\fix}[1]{\marginpar{%
    \fbox{\parbox{\fixboxwidth}{\footnotesize \red{#1}}}}}

%%%%%%%%%%%%%%%%%%%%%%%%%% Text commands
\usepackage{color}
\definecolor{emph}{rgb}{.1,.4,.1}
\definecolor{dark-red}{rgb}{0.4,0.15,0.15}
\definecolor{dark-blue}{rgb}{0.15,0.15,0.4}
\definecolor{medium-blue}{rgb}{0,0,0.5}
\newcommand{\ow}[1]{``\emph{#1}''}%\textcolor{emph}{\emph{#1}}
\newcommand{\e}[1]{\emph{#1}}
\newcommand{\red}[1]{\textcolor{red}{#1}}
%\newenvironment{redp}{\par\color{red}}{\par}

%%%%%%%%%%%%%%%%%%%% Figure stuff
\newenvironment{myfig}[2]{%
\def\mystupidworkaround{#1}
\def\mystupidworkaroundtwo{#2}
\begin{center}
}{%
	\captionof{figure}{\mystupidworkaroundtwo}
  	\label{\mystupidworkaround}
\end{center}
}
% \usepackage{float}
% \newenvironment{myfig}[2]{%
% \def\mystupidworkaround{#1}
% \def\mystupidworkaroundtwo{#2}
% \begin{figure}[H]
% 	\centering
% }{%
%  	\caption{\mystupidworkaroundtwo}
%  	\label{\mystupidworkaround}
% \end{figure}
% }

% % MathBBM / MathCal symbols
\newcommand{\R}{\mathbb{R}}
\newcommand{\N}{\mathbb{N}}
\newcommand{\D}{\mathcal{D}}
\renewcommand{\P}{\mathcal{P}}
\renewcommand{\L}{\mathcal{L}}
\newcommand{\Udeim}{\mathcal{U}}
\newcommand{\E}{\mathcal{E}}
\newcommand{\V}{\mathcal{V}}
\newcommand{\W}{\mathcal{W}}
\newcommand{\Rp}{\mathbb{R}_+}

%%%%%%%%%%%%%%%%%%%%%%%% Dimensions
% State space: d
\def\dis{l} % input space
\def\dos{k} % output space
\def\dpar{p} % parameter space

%%%%%%%%%%%%%%%%%%%%%%%% Kernel related 
\newcommand{\K}{K}
\renewcommand{\H}{\mathcal{H}}
\newcommand{\Pxdot}{\K(\vx,\cdot)}
\newcommand{\Pydot}{\K(\vy,\cdot)}
\newcommand{\Pxidot}{\K(\vx_i,\cdot)}
\newcommand{\Pxjdot}{\K(\vx_j,\cdot)}
\newcommand{\Pyidot}{\K(\vy_i,\cdot)}
\newcommand{\Kxix}{\K(\vx_i,\vx)}
\newcommand{\spH}[2]{\sp{#1}{#2}_{\H}}
\newcommand{\spHq}[2]{\sp{#1}{#2}_{\H^q}}
\newcommand{\noH}[1]{\norm{#1}{\H}}
\newcommand{\noHq}[1]{\norm{#1}{\H^q}}
\newcommand{\Kh}{\tilde{\K}}
\newcommand{\vci}{\vc_i}

%%%%%%%%%%%%%%%%%%%%%%%% Vectors and Matrices
\newcommand{\mx}[1]{\ensuremath{\left(\begin{matrix}#1\end{matrix}\right)}}
\newcommand{\sm}[1]{\ensuremath{\left(\begin{smallmatrix}#1\end{smallmatrix}\right)}}
\newcommand{\makebf}[1]{\boldsymbol{#1}}
\newcommand{\vnull}{\makebf{0}}
\newcommand{\vone}{\makebf{1}}
\newcommand{\vA}{\makebf{A}}
\newcommand{\vB}{\makebf{B}}
\newcommand{\vC}{\makebf{C}}
\newcommand{\vD}{\makebf{D}}
\newcommand{\vE}{\makebf{E}}
\newcommand{\vF}{\makebf{F}}
\newcommand{\vG}{\makebf{G}}
\newcommand{\vH}{\makebf{H}}
\newcommand{\vI}{\makebf{I}}
\newcommand{\vJ}{\makebf{J}}
\newcommand{\vK}{\makebf{K}}
\newcommand{\vL}{\makebf{L}}
\newcommand{\vM}{\makebf{M}}
\newcommand{\vN}{\makebf{N}}
\newcommand{\vP}{\makebf{P}}
\newcommand{\vQ}{\makebf{Q}}
\newcommand{\vR}{\makebf{R}}
\newcommand{\vS}{\makebf{S}}
\newcommand{\vU}{\makebf{U}}
\newcommand{\vV}{\makebf{V}}
\newcommand{\vW}{\makebf{W}}
\newcommand{\vX}{\makebf{X}}
\newcommand{\vY}{\makebf{Y}}
\newcommand{\va}{\makebf{a}}
\newcommand{\vb}{\makebf{b}}
\newcommand{\vc}{\makebf{c}}
\newcommand{\vd}{\makebf{d}}
\newcommand{\ve}{\makebf{e}}
\newcommand{\vf}{\makebf{f}}
\newcommand{\vg}{\makebf{g}}
\newcommand{\vj}{\makebf{j}}
\newcommand{\vk}{\makebf{k}}
\newcommand{\vn}{\makebf{n}}
\newcommand{\vp}{\makebf{p}}
\newcommand{\vq}{\makebf{q}}
\newcommand{\vr}{\makebf{r}}
\newcommand{\vs}{\makebf{s}}
\newcommand{\vu}{\makebf{u}}
\newcommand{\vv}{\makebf{v}}
\newcommand{\vw}{\makebf{w}}
\newcommand{\vx}{\makebf{x}}
\newcommand{\vy}{\makebf{y}}
\newcommand{\vz}{\makebf{z}}
\newcommand{\val}{\makebf{\alpha}}
\newcommand{\vmu}{{\makebf{\mu}}}
\newcommand{\vxi}{\makebf{\xi}}
\newcommand{\veta}{\makebf{\eta}}
\newcommand{\vSig}{\makebf{\Sigma}}
\newcommand{\vLam}{\makebf{\Lambda}}

% Tilde: Projected quantities 
\newcommand{\vtA}{\tilde{\vA}}
\newcommand{\vtB}{\tilde{\vB}}
\newcommand{\vtC}{\tilde{\vC}}
\newcommand{\vtJ}{\tilde{\vJ}}
\newcommand{\vtM}{\tilde{\vM}}
\newcommand{\vtQ}{\tilde{\vQ}}
\newcommand{\vtU}{\tilde{\vU}}
\newcommand{\vtV}{\tilde{\vV}}
\newcommand{\vtc}{\tilde{\vc}}
\newcommand{\vtq}{\tilde{\vq}}
\newcommand{\vtx}{\tilde{\vx}}
\newcommand{\vtv}{\tilde{\vv}}
\newcommand{\vtz}{\tilde{\vz}}
\newcommand{\tf}{\tilde{f}}
\newcommand{\tlambda}{\tilde{\lambda}}

% Hat: Approximated quantities
\newcommand{\hf}{\hat{f}}
\newcommand{\vhf}{{\hat{\vf}}}
\newcommand{\vhV}{{\hat{\vV}}}
\newcommand{\vhW}{{\hat{\vW}}}
\newcommand{\vhP}{\hat{\vP}}
\newcommand{\vhR}{{\hat{\vR}}}
\newcommand{\vhU}{\hat{\vU}}

% Tilde hat: Projected approximated quantities
\newcommand{\vthf}{\tilde{\vhf}}

%%%%%%%%%%%%%%%%%%%%%%%% Misc
\renewcommand{\O}[1]{\ensuremath{\mathcal{O}\left(#1\right)}}
\newcommand{\bs}{\backslash}
\newcommand{\fo}{~\forall~}
\newcommand{\ex}{~\exists~}
\newcommand{\exu}{~\exists!~}
\newcommand{\ep}{\epsilon}
\newcommand{\es}{\emptyset}
\newcommand{\norm}[2]{\left|\left|#1\right|\right|_{#2}}
\newcommand{\no}[1]{\norm{#1}{}}
\newcommand{\noG}[1]{\norm{#1}{G}}
\newcommand{\Mmo}{\mathcal{C}}
\newcommand{\mmo}{\mathcal{D}}
\newcommand{\PM}{\P_{\Mmo}}
\newcommand{\Pm}{\P_{\mmo}}
\newcommand{\flux}{f}%
\newcommand{\sat}{s}%s^{\ep,\tau}
\newcommand{\col}{p{2cm}}

\renewcommand{\sp}[2]{\left\langle #1,#2 \right\rangle}
\newcommand{\spG}[2]{\sp{#1}{#2}_{G}}
% \newcommand{\spHq}[2]{\sp{#1}{#2}_{\H^q}}
\newcommand{\Ball}[2]{B_{#1}\left(#2\right)}

%%%%%%%%%%%%%%%%%%%%%%%%% Sizes
\newcommand{\single}{.8\textwidth} % 1 image in row
\newcommand{\third}{.32\textwidth}
\newcommand{\half}{.49\textwidth}
\newcommand{\quarter}{.24\textwidth}

%%%%%%%%%%%%%%%%%%%%%%%%% Sums, integrals, limits 
\newcommand{\suml}[2]{\sum\limits_{#1}^{#2}}
\newcommand{\sumi}{\suml{i=1}{N}}
\newcommand{\sumim}{\suml{i=1}{m}}
\newcommand{\sumj}{\suml{j=1}{N}}
\newcommand{\sumk}{\suml{k=1}{N}}
\newcommand{\sumjq}{\suml{j=1}{q}}
\newcommand{\sumr}{\suml{i=1}{r}}
\newcommand{\sumjk}{\suml{j=1}{k}}
\newcommand{\sumik}{\suml{j=1}{k}}
\newcommand{\sumkM}{\suml{k=1}{M}}
\newcommand{\intl}[2]{\int\limits_{#1}^{#2}}

%%%%%%%%%%%%%%%%%%%%%%%%%% SVR-Commands
\newcommand{\ap}{\val^+}
\newcommand{\am}{\val^-}
\newcommand{\aip}{\alpha_i^+}
\newcommand{\aim}{\alpha_i^-}
\newcommand{\ajp}{\alpha_j^+}
\newcommand{\ajm}{\alpha_j^-}
\newcommand{\apm}{\ap-\am}
\newcommand{\apmb}{\left(\ap-\am\right)}
\newcommand{\dw}{\nabla W}
\newcommand{\dwp}{\dw^+(\ap,\am)}
\newcommand{\dwpi}{\dw_i^+(\ap,\am)}
\newcommand{\dwpj}{\dw_j^+(\ap,\am)}
\newcommand{\dwm}{\dw^-(\ap,\am)}
\newcommand{\dwmi}{\dw_i^-(\ap,\am)}
\newcommand{\dwmj}{\dw_j^-(\ap,\am)}
\newcommand{\vdw}{\nabla \vW}
\newcommand{\clip}[3]{\left[#1\right]^{#2}_{#3}}
\newcommand{\nonp}[1]{\norm{#1}{\Np}}
\newcommand{\ripp}{r_i^{++}}
\newcommand{\sjpp}{s_j^{++}}
\newcommand{\xip}{\xi_i^+}
\newcommand{\xim}{\xi_i^-}
\newcommand{\noep}[1]{\left|#1\right|_\ep}
\newcommand{\wplus}{\nabla\vw^+} 
\newcommand{\wm}{\nabla\vw^-}

%%%%%%%%%%%%%%%%%%%%%%%%%% VKOGA-Commands
\newcommand{\tp}{\tilde{\phi}}
\newcommand{\sfx}{{s_{f,X}}}
\newcommand{\sfxm}{{s_{f,X_m}}}
\newcommand{\Pq}{\P^q}
\newcommand{\PHX}{\P_{\H^X}}
%\newcommand{\PHXq}{\Pq_{\H^X}}
\newcommand{\Kinj}{(\vK^{-1})_j^T}
\newcommand{\OX}{\Omega_X}
\newcommand{\OXm}{\vx\in\Omega\bs\Omega_{X_{m-1}}}
\newcommand{\nKx}{\nabla\vK(\vx)}
\newcommand{\gxf}{G_{X,\vf}} % vectorial gain function
\newcommand{\px}{\phi_{\vx}}
\newcommand{\pxv}{\px^{\nabla v}}
\newcommand{\tpx}{\tp_{\vx}}
\newcommand{\tpv}{\tp^{\nabla v}}
\newcommand{\vtpx}{\makebf{\tp}_{\vx}}
\newcommand{\tpxv}{\tpx^{\nabla v}}
\newcommand{\tN}{\tilde{N}}
\newcommand{\im}{{i_{max}}}

%%%%%%%%%%%%%%%%%%%%%% DEIM commands
\newcommand{\ei}{{\wp_i}}
\newcommand{\DEI}{\DE_{EI}}
\newcommand{\DEIJ}{\DE}
\newcommand{\Prm}{\Pi_m}
\newcommand{\Pmd}{\Pi_{m'}}
\newcommand{\proj}{\bigl(\vI - \vV\vW^T\bigr)}

%%%%%%%%%%%%%%%%%%%%%% Operators
\DeclareMathOperator{\rangeop}{range}
\DeclareMathOperator{\meanop}{mean}
\DeclareMathOperator{\sgn}{sign}
\DeclareMathOperator{\diagmat}{diag}
\DeclareMathOperator{\divop}{div}
\newcommand{\sign}[1]{\sgn\left(#1\right)}
\newcommand{\range}[1]{\rangeop\left(#1\right)}
\newcommand{\diag}[1]{\diagmat\left(#1\right)}
\renewcommand{\d}[2]{\frac{\partial #1}{\partial #2}}
\newcommand{\dx}{\partial_x}%
\newcommand{\dd}[2]{\frac{\partial^2 #1}{\partial {#2}^2}}

%%%%%%%%%%%%%%%%%%%%%% Error estimation
\newcommand{\intt}{\intl{0}{t}}
\newcommand{\exo}{E_0} % initial error
\newcommand{\exomu}{\exo(\vmu)}
\newcommand{\ea}{E_A} % approximation error
\newcommand{\eproj}{\left(\vI-\vV\vW^T\right)} %projection error
\newcommand{\fd}{{h_{X,\Omega}}} % fill distance
\renewcommand{\Ball}[2]{\overline{B_{#1}\left(#2\right)}}
\newcommand{\Bts}{\Ball{\Theta}{s}}
\newcommand{\Bty}{\Ball{\Theta}{y}}
\newcommand{\Bf}{\mathcal{B}} % bell functions
\newcommand{\DE}{\Delta}
\newcommand{\DGLE}{\DE_{GLE}}
\newcommand{\DLSLE}{\DE_{LSLE}} 
\newcommand{\Gno}{\Gamma_1^c}

%%%%%%%%%%%%%%%%%%%% Code
\newcommand{\ML}{\textsc{MatLab}}%{\tiny\texttrademark~}
\newcommand{\KM}{\textit{KerMor}}
\newcommand{\code}[1]{\lstinline$#1$}
\definecolor{c-keywords}{rgb}{.1,.1,.5}
\definecolor{c-identifier}{rgb}{0.3,0.3,0.3}

\usepackage{verbatim}
\lstset{language=Matlab,
	basicstyle=\scriptsize\ttfamily,
	keywordstyle=\bfseries\ttfamily\color{c-keywords},
	identifierstyle=\color{c-identifier},
	commentstyle=\color{green},
	stringstyle=\ttfamily,
	showstringspaces=false,
	fancyvrb=true,
	xleftmargin=5pt,
	morekeywords={doc,help,methods,events,properties,abstract,
                      classdef,double,true,false,this,access,setaccess,getaccess,
		      varargin, varargout}
}
 
% %%%%%%%%%%%%%%%%%%%%%%%%%%%%%%%%%%%%%%%%%%%%%%%%%%%%%%%%%%%%%%%%%%%%%
% %% THEOREM & LEMMA COMMANDS
% %%%%%%%%%%%%%%%%%%%%%%%%%%%%%%%%%%%%%%%%%%%%%%%%%%%%%%%%%%%%%%%%%%%%%
\newtheorem{theorem}{Theorem}[section]
\newtheorem{lemma}[theorem]{Lemma}
\newtheorem{corollary}[theorem]{Corollary}
\newtheorem{proposition}[theorem]{Proposition}
 	
% Option 1: Continuous numbering also for defs & remarks
\theoremstyle{definition}
\newtheorem{definition}[theorem]{Definition}
\theoremstyle{remark}
\newtheorem{remark}[theorem]{Remark}
% 	% Option 2: Own counters for defs & remarks
% 	\theoremstyle{definition}
% 	\newtheorem{definition}{Definition}[section]
% 	\theoremstyle{remark}
% 	\newtheorem{remark}{Remark}[section]
% 	
% 	%\newtheorem{example}[theorem]{Example}
% 	%\newtheorem{xca}[theorem]{Exercise}

%--------------------------------------------------------------------------------------------------
%
% place to define macros, shortcuts, etc. ...
%
%--------------------------------------------------------------------------------------------------
%
 % e.g., i.e., c.f.
%
%--------------------------------------------------------------------------------------------------
%
\newcommand{\eg}{e.g.~}
\newcommand{\ie}{i.e.~}
\newcommand{\iec}{i.e.,}
\newcommand{\cf}{c.f.~}
%
%--------------------------------------------------------------------------------------------------
%
% ENVIRONMENTS
%
%--------------------------------------------------------------------------------------------------
% figure
\newcommand\bfg[1]{
        \begin{figure}[#1]
        \begin{center}
}
\newcommand\efg{
        \end{center}
        \end{figure}
} 
%
% minipage  <--- NOT WORKING ...?
\newcommand\bmp[1]{
        \begin{minipage}{#1}
        \begin{center}
}
\newcommand\emp[1]{
        \end{center}
        \end{minipage}
}                
%
%--------------------------------------------------------------------------------------------------
%
 % REFERENCES
%
%--------------------------------------------------------------------------------------------------
% equation
\newcommand{\eqnrefE}[1]{\textup{Equation~}(\ref{#1})}
\newcommand{\eqnref}[1]{(\ref{#1})}
%
% figure
\newcommand{\figref}[1]{\textup{Figure~}\ref{#1}}
%
% table
\newcommand{\tabref}[1]{\textup{Table~}\ref{#1}}
%
% chapter
\newcommand{\chref}[1]{\textup{Chapter~}\ref{#1}}
%
% section
\newcommand{\secref}[1]{\textup{Section~}\ref{#1}}
%
% subsection
\newcommand{\subsecref}[1]{\textup{Subsection~}\ref{#1}}
%
%--------------------------------------------------------------------------------------------------
%
% SHORTCUTS IN MATHEMATICS
%
%--------------------------------------------------------------------------------------------------
% environments
%
% equation + label - (use as \beq{wantedlabelname} ... \eeq)
% --> use equation + split for one alignment
\newcommand{\beq}[1]{\begin{equation}\label{#1}}
\newcommand{\eeq}{\end{equation}}
%
% split
\newcommand{\splt}[1]{\begin{split} #1 \end{split}}
%
% align - (use as \alg{...})
% --> use align for multiple alignments & each line numbered
\newcommand{\alg}[1]{\begin{align} #1 \end{align}} 
%
% array
\newcommand{\barray}[1]{\begin{array}{#1}}
\newcommand{\earray}{\end{array}}
%
%------------------------------------------------
% short notations
%
% number symbols
%\newcommand{\N}{\mathbb{N}} % integer numbers
%\newcommand{\R}{\mathbb{R}} % real numbers
%
% bold numbers
\newcommand{\Bzero}{\boldsymbol{0}}
\newcommand{\Bone}{\boldsymbol{1}}
%
% bold letters - small
\newcommand{\Ba}{\boldsymbol{a}}
\newcommand{\Bb}{\boldsymbol{b}}
\newcommand{\Bc}{\boldsymbol{c}}
\newcommand{\Bd}{\boldsymbol{d}}
\newcommand{\Be}{\boldsymbol{e}}
%\newcommand{\Bf}{\boldsymbol{f}}
\newcommand{\Bg}{\boldsymbol{g}}
\newcommand{\Bh}{\boldsymbol{h}}
\newcommand{\Bi}{\boldsymbol{i}}
\newcommand{\Bj}{\boldsymbol{j}}
\newcommand{\Bk}{\boldsymbol{k}}
\newcommand{\Bl}{\boldsymbol{l}}
\newcommand{\Bm}{\boldsymbol{m}}
\newcommand{\Bn}{\boldsymbol{n}}
\newcommand{\Bo}{\boldsymbol{o}}
\newcommand{\Bp}{\boldsymbol{p}}
\newcommand{\Bq}{\boldsymbol{q}}
\newcommand{\Br}{\boldsymbol{r}}
\newcommand{\Bs}{\boldsymbol{s}}
\newcommand{\Bt}{\boldsymbol{t}}
\newcommand{\Bu}{\boldsymbol{u}}
\newcommand{\Bv}{\boldsymbol{v}}
\newcommand{\Bw}{\boldsymbol{w}}
\newcommand{\Bx}{\boldsymbol{x}}
\newcommand{\By}{\boldsymbol{y}}
\newcommand{\Bz}{\boldsymbol{z}}
%
% bold letters - capital
\newcommand{\BA}{\boldsymbol{A}}
\newcommand{\BB}{\boldsymbol{B}}
\newcommand{\BC}{\boldsymbol{C}}
\newcommand{\BD}{\boldsymbol{D}}
\newcommand{\BE}{\boldsymbol{E}}
\newcommand{\BF}{\boldsymbol{F}}
\newcommand{\BG}{\boldsymbol{G}}
\newcommand{\BH}{\boldsymbol{H}}
\newcommand{\BI}{\boldsymbol{I}}
\newcommand{\BJ}{\boldsymbol{J}}
\newcommand{\BK}{\boldsymbol{K}}
\newcommand{\BL}{\boldsymbol{L}}
\newcommand{\BM}{\boldsymbol{M}}
\newcommand{\BN}{\boldsymbol{N}}
\newcommand{\BO}{\boldsymbol{O}}
\newcommand{\BP}{\boldsymbol{P}}
\newcommand{\BQ}{\boldsymbol{Q}}
\newcommand{\BR}{\boldsymbol{R}}
\newcommand{\BS}{\boldsymbol{S}}
\newcommand{\BT}{\boldsymbol{T}}
\newcommand{\BU}{\boldsymbol{U}}
\newcommand{\BV}{\boldsymbol{V}}
\newcommand{\BW}{\boldsymbol{W}}
\newcommand{\BX}{\boldsymbol{X}}
\newcommand{\BY}{\boldsymbol{Y}}
\newcommand{\BZ}{\boldsymbol{Z}}
%
% bold greek symbols - small
\newcommand{\Balpha}{\boldsymbol{\alpha}}
\newcommand{\Bbeta}{\boldsymbol{\beta}}
\newcommand{\Bgamma}{\boldsymbol{\gamma}}
\newcommand{\Bdelta}{\boldsymbol{\delta}}
\newcommand{\Bepsilon}{\boldsymbol{\epsilon}}
\newcommand{\Bvarepsilon}{\boldsymbol{\varepsilon}}
\newcommand{\Bzeta}{\boldsymbol{\zeta}}
\newcommand{\Beta}{\boldsymbol{\eta}}
\newcommand{\Btheta}{\boldsymbol{\theta}}
\newcommand{\Biota}{\boldsymbol{\iota}}
\newcommand{\Bkappa}{\boldsymbol{\kappa}}
\newcommand{\Bvarkappa}{\boldsymbol{\varkappa}}
\newcommand{\Blambda}{\boldsymbol{\lambda}}
\newcommand{\Bmu}{\boldsymbol{\mu}}
\newcommand{\Bnu}{\boldsymbol{\nu}}
\newcommand{\Bxi}{\boldsymbol{\xi}}
\newcommand{\Bpi}{\boldsymbol{\pi}}
\newcommand{\Brho}{\boldsymbol{\rho}}
\newcommand{\Bvarrho}{\boldsymbol{\varrho}}
\newcommand{\Bsigma}{\boldsymbol{\sigma}}
\newcommand{\Btau}{\boldsymbol{\tau}}
\newcommand{\Bupsilon}{\boldsymbol{\upsilon}}
\newcommand{\Bphi}{\boldsymbol{\phi}}
\newcommand{\Bvarphi}{\boldsymbol{\varphi}}
\newcommand{\Bchi}{\boldsymbol{\chi}}
\newcommand{\Bpsi}{\boldsymbol{\psi}}
\newcommand{\Bomega}{\boldsymbol{\omega}}
%
% bold greek symbols - capital
\newcommand{\Alpha}{A}
\newcommand{\BAlpha}{\boldsymbol{\Alpha}}
\newcommand{\BBeta}{\boldsymbol{B}}
\newcommand{\BGamma}{\boldsymbol{\Gamma}}
\newcommand{\BDelta}{\boldsymbol{\Delta}}
\newcommand{\Epsilon}{E}
\newcommand{\BEpsilon}{\boldsymbol{\Epsilon}}
\newcommand{\Zeta}{Z}
\newcommand{\BZeta}{\boldsymbol{\Zeta}}
\newcommand{\Eta}{H}
\newcommand{\BEta}{\boldsymbol{\Eta}}
\newcommand{\BTheta}{\boldsymbol{\Theta}}
\newcommand{\Iota}{I}
\newcommand{\BIota}{\boldsymbol{\Iota}}
\newcommand{\Kappa}{K}
\newcommand{\BKappa}{\boldsymbol{\Kappa}}
\newcommand{\BLambda}{\boldsymbol{\Lambda}}
\newcommand{\Mu}{M}
\newcommand{\BMu}{\boldsymbol{\Mu}}
\newcommand{\Nu}{N}
\newcommand{\BNu}{\boldsymbol{\Nu}}
\newcommand{\BXi}{\boldsymbol{\Xi}}
\newcommand{\BPi}{\boldsymbol{\Pi}}
\newcommand{\Rho}{R}
\newcommand{\BRho}{\boldsymbol{\Rho}}
\newcommand{\BSigma}{\boldsymbol{\Sigma}}
\newcommand{\Tau}{T}
\newcommand{\BTau}{\boldsymbol{\Tau}}
\newcommand{\BUpsilon}{\boldsymbol{\Upsilon}}
\newcommand{\BPhi}{\boldsymbol{\Phi}}
\newcommand{\Chi}{X}
\newcommand{\BChi}{\boldsymbol{\Chi}}
\newcommand{\BPsi}{\boldsymbol{\Psi}}
\newcommand{\BOmega}{\boldsymbol{\Omega}}
%
% special stuff
\newcommand{\delOmega}{\partial\Omega}
%
%------------------------------------------------
%
% partial derivative - (use by \delby{}{})
\newcommand{\delby}[2]{\frac{\partial #1}{\partial #2}}
%
% integral - (use by \intlim{start}{stop}{integrand}{variable})
\newcommand{\intlim}[4]{
  \int\limits_{#1}^{#2}{#3}\hspace{2pt}d#4  
}
%
% sum - (use by \sumlim{start}{stop}{summands})
\newcommand{\sumlim}[3]{
  \sum\limits_{#1}^{#2}{#3} 
}
%
% norm with varying size
%\newcommand{\norm}[1]{\left\lVert#1\right\rVert}
%
%--------------------------------------------------------------------------------------------------
%
%--------------------------------------------------------------------------------------------------

%
\newcommand{\Ns}{\mathcal{N}}
\newcommand{\Nk}{\mathcal{N}_k}
\newcommand{\Or}{\Omega_0}
\newcommand{\Om}{\Omega_R}
\newcommand{\intor}{\intl{\Or}{}}
\newcommand{\intom}{\intl{\Om}{}}
\newcommand{\intorb}{\intl{\partial\Or}{}}
\newcommand{\vT}{\makebf{T}}
\newcommand{\vt}{\makebf{t}}
\newcommand{\vsig}{\makebf{\sigma}}
\newcommand{\vSi}{\vS^{iso}}
\newcommand{\vSa}{\vS^{aniso}}
\newcommand{\vSf}{\vS^{act}}
\newcommand{\vPi}{\vP^{iso}}
\newcommand{\vPa}{\vP^{aniso}}
\newcommand{\vPf}{\vP^{act}}
\newcommand{\Sfun}{\makebf{\Upsilon}}
\newcommand{\force}{\vG}
\DeclareMathOperator{\divergence}{\nabla}
\DeclareMathOperator{\tr}{tr}
\DeclareMathOperator{\supp}{supp}
\DeclareMathOperator{\grad}{grad}
\renewcommand{\div}[1]{\divergence\cdot\left(#1\right)}
\newcommand{\m}[1]{\ensuremath{\left(\begin{matrix}#1\end{matrix}\right)}}
\newcommand{\sumnk}{\suml{i\in{\Nk}}{}}
\newcommand{\sumgp}{\suml{p=1}{G}}
\newcommand{\sumiP}{\suml{i=1}{N_p}}
\newcommand{\jmp}{J^m_p}
\newcommand{\re}[1]{\stackrel{\eqref{#1}}{=}}
\newcommand{\sumvk}{\suml{m\in V_k}{}}
\newcommand{\pmp}{X^m_p}%\Phi_m(X_p)
\usepackage{mathtools}
\DeclarePairedDelimiter{\ceil}{\lceil}{\rceil}
\newcommand{\Dp}[1]{\nabla\varphi_{#1}(X)}
\newcommand{\dpk}{\Dp{k}}
\newcommand{\dpi}{\Dp{i}}
\newcommand{\dpj}{\Dp{j}}
\newcommand{\dNkmp}{N^\nabla_{k,m}(X_p)}
\newcommand{\dNimp}{N^\nabla_{i,m}(X_p)}
\newcommand{\dNjmp}{N^\nabla_{j,m}(X_p)}
\newcommand{\Nkmp}{N_{l(k,m)}(X_p)}
\newcommand{\lfo}{\lambda_f^{opt}}

\allowdisplaybreaks
\begin{document}
As perspective, we will work on the \e{reference configuration} regarding all quantities and relate to the \e{current configuration} only where necessary.
The quantities referred to in the reference configuration are upper-case letters and we denote quantities in the current configuration by lower-case letters.
Bold face is used to indicate vectorial quantities.

We model the movement/deformation of a continuous muscle over time, whose shape corresponds to the reference domain $\Or\subset\R^3$.
We choose the units $[X] = mm, [t] = ms$.
The movement of each particle $X\in\R^3$ in the body is described by a \e{motion} $\chi(X,t)\in\R^3$ (with $[\chi(X,t)] = mm$),
which gives the position of each $X\in\R^3$ at time $t\in[0,T]$ with $\Or = \chi(\Or,0)$.

We define the quantities
\begin{align}
	&\text{velocity field} & \vV(X,t) &:= \d{\chi}{t}(X,t), && \left[\frac{mm}{ms}\right]\\
	&\text{acceleration field} & \vA(X,t) &:= \d{\vV}{t}(X,t) = \dd{\chi}{t}(X,t), && \left[\frac{mm}{ms^2}\right]\\
	&\text{deformation gradient} & \vF(X,t) &:= \d{\chi}{X}(X,t) = \m{\nabla \chi_1 & \nabla \chi_2 & \nabla \chi_3}^T \in \R^{3\times 3}, && [-]\\
	&\text{right cauchy strain tensor} & \vC(X,t) &:= \vF(X,t)^T\vF(X,t), && [-]\\
	&\text{volume ratio} & J(X,t) &:= \det(\vF(X,t)). && [-]
\end{align}

\section{Derivation of governing equations}
Assuming a mass $\rho_0(X)$ at each point we define the total momentum
\[
 	\vL(t) := \intor \rho_0(X)\vV(X,t) dX,
\] 
for which we postulate the balance equation
\begin{align}
	\force(t) \stackrel{!}{=} \d{\vL}{t}(t) = \vL'(t) = \intor \rho_0(X)\d{\vV}{t}(X,t) dX = \intor \rho_0(X)\vA(X,t) dX,\label{def:bal_eq}
\end{align}
given \e{resultant forces} $\force(t)$.

\subsection{Structure of force}
We assume the forces $\force(t)$ to be composed of two different sources: Forces on the boundary and body forces.
The body forces $\vB(X,t)$ measures the force per unit reference volume on $X$ at time $t$.
These forces are self-weight or gravity, for example.
Further, we assume to have \e{traction vectors} $\vT(X,t,N)$ (first Piola-Kirchhoff traction vector) that indicates the force working per unit surface area with normal $N$ 
at the point $X\in\partial\Or$ at time $t$.
Then the resultant forces are given as
\begin{align}
	\force(t) := \intorb \vT(X,t,N)dN + \intor \vB(X,t)dX \label{def:force}
\end{align}

According to \e{Cauchy's stress theorem}, we can express tractions as tensor product
\begin{align}
	\vT(X,t,N) &= \vP(X,t)N,\label{def:T}
\end{align}
where $\vP$ denotes the \e{first Piola-Kirchhoff stress tensor}.
With this we have, following Gauss integral theorem,
\begin{align}
	\intorb \vT(X,t,N)dN &= \intorb \vP(X,t)NdN = \intor \divergence\vP(X,t)dX.\label{eq:div_form_traction}
\end{align}
Using representation \eqref{eq:div_form_traction} in the force composition 
\eqref{def:force} yields the following form of the balance equation \eqref{def:bal_eq}:
\begin{align}
	\intor \rho_0(X)\d{\vV}{t}(X,t) dX &= \intor \divergence\vP(X,t) + \vB(X,t) dX
\end{align}
As this balance must also be satisfied for each subvolume $\Omega_t\subset\Or$, we actually obtain a pointwise or local form as
\begin{align}
	\rho_0(X)\d{\vV}{t}(X,t) &= \divergence\vP(X,t) + \vB(X,t) \qquad \fo X\in\Or\label{def:maineq}
\end{align}

\subsection{Stress tensor definition}
We will actually assume
\begin{align}
	\vP(X,t) &:= \vF(X,t)\vS(X,t),
\end{align}
where $\vS(X,t)$ stands for the \e{second Piola-Kirchhoff stress tensor}.
Now, most generally one makes the assumption that any stress tensor depends on the location and the current deformation, i.e.
\[
	\vS(X,t) = \tilde{\Sfun}(\vF(X,t),X),\quad \hat{\Sfun}:\R^{3\times 3} \to \R^{3\times 3}.
\]
We assume \e{homogeneous} material, which means that a direct spatial dependence is not given and we thus omit the argument.
Further, due to the principle of material frame-indifference \cite[p.198]{Holzapfel2000}, $\vS$
may actually only depend on the rotation-invariant part of $\vF$.
For any fixed $X,t$ we can decompose $\vF = \vR\vU$ (\cite[p.85]{Holzapfel2000}) with a
\e{rotation} part $\vR$ with $\vR^T\vR = \vI$ and \e{stretch} part $\vU = \vU^T$.
Thus $\vS$ may only depend on $\vU$; since
\[
	\vC = \vF^T\vF = (\vR\vU)^T\vR\vU = \vU^T\vR^T\vR\vU = \vU\vU = \vU^2, 
\]
we will assume that
\[
	\vS(X,t) = \Sfun(\vC(X,t))
\]
for suitable tensor function $\Sfun$.
We also introduce the concept of a \e{fibre direction} $\va_0(X) \in\R^3$, where we will omit the argument $X$ in the following.
Then, for \e{hyperelastic and transversely isotropic} material we have the definition
\[
	\vS(X,t) = \Sfun(\vC(X,t)) := 2\d{\Psi}{\vC}(\vC(X,t),\va_0(X)),
\]
where $\Psi$ is called \e{strain-energy} function, see \cite[p.207]{Holzapfel2000}.

\subsubsection{Invariants}
We further have for any tensor $\vA$ with eigenvalues $\lambda_1,\lambda_2,\lambda_3$ the \e{invariants}
\begin{align}
	I_1(\vA) &:= \tr\vA = \lambda_1 + \lambda_2 + \lambda_3\label{def:I1}\\
	I_2(\vA) &:= \frac{1}{2}\left((\tr \vA)^2 - \tr\vA^2\right) = \lambda_1\lambda_2 + \lambda_1\lambda_3 + \lambda_2\lambda_3\\
	I_3(\vA) &:= \det\vA = \lambda_1\lambda_2\lambda_3\label{def:I3}\\
	I_4(\vA,\vv) &:= \vv\cdot\vA\vv = \vA : (\vv \otimes \vv)\label{def:I4} =: \lambda^2_{\vv}\\
	I_5(\vA,\vv) &:= \vv\cdot\vA^2\vv\label{def:I5}
\end{align}
with derivatives for \e{symmetric} $\vA$
\begin{align}
	\d{I_1}{\vA}(\vA) &:= \d{\tr\vA}{\vA}(\vA) = \d{\vI:\vA}{\vA}(\vA) = \vI\\
	\d{I_2}{\vA}(\vA) &:= I_1(\vA)\vI - \vA\\
	\d{I_3}{\vA}(\vA) &:= I_3\vA^{-1}\\
	\d{I_4}{\vA}(\vA) &:= \vv \otimes \vv\label{def:dI4}\\
	\d{I_5}{\vA}(\vA) &:= \vv \otimes \vA\vv + \vv\vA \otimes \vv,
\end{align}
see \cite[p.216/p.268]{Holzapfel2000}.

Now, $\Psi$ can actually be formulated using the invariants \eqref{def:I1}-\eqref{def:I5} as
\[
	\Psi(\vC) = \Psi(I_1(\vC),I_2(\vC),I_3(\vC),I_4(\vC,\va_0),I_5(\vC,\va_0)).
\]
This gives the general form
\begin{align}
	\vS(X,t) &= 2\d{\Psi}{\vC}(I_1,I_2,I_3,I_4,I_5)\\
		&= \d{\Psi}{I_1}\d{I_1}{\vC} + \d{\Psi}{I_2}\d{I_2}{\vC} + \d{\Psi}{I_3}\d{I_3}{\vC} + \d{\Psi}{I_4}\d{I_4}{\vC} + \d{\Psi}{I_5}\d{I_5}{\vC}\\
		&= 2\Biggl[\left(\d{\Psi}{I_1} + I_1\d{\Psi}{I_2}\right)\vI - \d{\Psi}{I_2}\vC + I_3\d{\Psi}{I_3}\vC^{-1}\\
		&\quad+ \d{\Psi}{I_4}\va_0\otimes\va_0 + \d{\Psi}{I_5}(\va_0\otimes\vA\va_0 + \va_0\vA\otimes\va_0)\Biggr].
\end{align}

Further, we consider \e{incompressible} material, which is expressed by the condition $J(X,t) = 1 \fo X,t$.
This leads to the constant third invariant $I_3 \equiv 1$, where an additional Lagrange-multiplier $p$ is introduced
to satisfy the condition via
\begin{align}
	\vS(X,t) = p\vC^{-1}(X,t) + 2\d{\Psi}{\vC}(\vC(X,t))
\end{align}

In the following, we will specify different $\Psi$ to create an additive split as 
\begin{align}
	\vS(X,t) = p\vC^{-1}(X,t) + \vSi(X,t) + \vSa(X,t) + \vSf(X,t)\label{def:S_split}
\end{align}

\subsubsection{Isotropic stress tensor $\vSi$}
Now, for the isotropic part we use
\[
	\Psi(I_1,I_2,I_3) = c_{10}(I_1-3) + c_{01}(I_2-3),
\]
which gives
\begin{align}
	\vSi(X,t) &= 2(c_{10} + I_1c_{01})\vI - 2c_{01}\vC(X,t)\\
	\vPi(X,t) &= 2(c_{10} + I_1c_{01})\vF(X,t) - 2c_{01}\vF(X,t)\vC(X,t)
\end{align}

\subsubsection{Anisotropic stress tensor $\vSa$}
Further, we introduce the stretch $\lambda_f$ in fibre direction $\va_0(X)$ as
\begin{align*}
	\lambda_f(X,t) &:= \sqrt{I_4(\vC(X,t),\va_0(X))}
\end{align*}
Now, according to \cite{Markert2005}, we employ
\begin{align*}
	\Psi(\lambda_f) := \sumi \frac{b_i}{d_i}(\lambda_f^{d_i} - 1) - b_i\ln\lambda_f
\end{align*}
with $n=1$ summands, so that
\begin{align}
	\d{\Psi}{\vC}(\lambda_f) &= \d{\Psi}{\lambda_f}(\lambda_f)\d{\lambda_f}{I_4}(I_4)\d{I_4}{\vC}(\vC,\va_0),\label{eq:dpsidC}
\end{align}
with
\begin{align*}
		 \d{\Psi}{\lambda_f}(\lambda_f) &= \frac{b_1(d_1-1)}{d_1}\lambda_f^{d-1} - b_1\lambda_f^{-1},\\
		 \d{\lambda_f}{I_4}(I_4) &= \frac{1}{2\sqrt{I_4}} = \frac{1}{2\lambda_f}.
\end{align*}
With \eqref{def:dI4} and the above we get
\begin{align*}
	2\d{\Psi}{\vC}(\lambda_f) &= 2\left(\frac{b_1(d_1-1)}{d_1}\lambda_f^{d-1} - b_1\lambda_f^{-1}\right) \frac{1}{2\lambda_f}\va_0\otimes\va_0\\
	&= \left(\frac{b_1(d_1-1)}{d_1}\lambda_f^{d-2} - b_1\lambda_f^{-2}\right) \va_0\otimes\va_0,
\end{align*}
leading to
\begin{align}
	\vSa(X,t) &= \left(\frac{b_1(d_1-1)}{d_1}\lambda_f^{d-2}(X,t) - b_1\lambda_f^{-2}(X,t)\right) \va_0(X)\otimes\va_0(X)\\
	\vPa(X,t) &= \left(\frac{b_1(d_1-1)}{d_1}\lambda_f^{d-2}(X,t) - b_1\lambda_f^{-2}(X,t)\right) \vF(X,t)\va_0(X)\otimes\va_0(X)
\end{align}

\subsubsection{Active force tensor $\vSf$}
Here we define
\begin{align*}
	\Psi(\lambda_f) &:= \intl{0}{\lambda_f} p^{act}(s)ds,\\
	p^{act}(\lambda_f) &:= p^{max}\gamma(\alpha,\lambda_f,\lambda_f'),
\end{align*}
with suitable constant $p^{max}$ and function $\gamma$.
Similar to \eqref{eq:dpsidC} we obtain
\begin{align}
	\d{\Psi}{\vC}(\lambda_f) &= \frac{p^{max}}{2\lambda_f}\gamma(\alpha,\lambda_f,\lambda_f')\va_0\otimes\va_0
\end{align}
This gives
\begin{align}
	\vSf(X,t) &= \frac{p^{max}}{\lambda_f(X,t)}\gamma\left(\alpha,\lambda_f(X,t),\d{\lambda_f}{t}(X,t)\right)\va_0(X)\otimes\va_0(X)\\
	\vPf(X,t) &= \frac{p^{max}}{\lambda_f(X,t)}\gamma\left(\alpha,\lambda_f(X,t),\d{\lambda_f}{t}(X,t)\right)\vF(X,t)\va_0(X)\otimes\va_0(X)
\end{align}

\subsubsection{Overall stress tensor}
Adding the different parts together as given in \eqref{def:S_split} now gives
\begin{align}
	\vS(X,t) &= 2(c_{10} + I_1c_{01})\vI - 2c_{01}\vC(X,t)\\
			 &\quad+\Biggl[\left(\frac{b_1(d_1-1)}{d_1}\lambda_f^{d-2}(X,t) - b_1\lambda_f^{-2}(X,t)\right)\nonumber\\
			 &\quad+\frac{p^{max}}{\lambda_f(X,t)}\gamma\left(\alpha,\lambda_f(X,t),\d{\lambda_f}{t}(X,t)\right)\Biggr]\va_0(X)\otimes\va_0(X)\nonumber\\
	g(\lambda_f)&:= \left(\frac{b_1(d_1-1)}{d_1}\lambda_f^{d-2} - b_1\lambda_f^{-2}\right)+\frac{p^{max}}{\lambda_f}\gamma\left(\alpha,\lambda_f,\d{\lambda_f}{t}\right)\\			 
	\vP(X,t) &= \vF(X,t)\vS(X,t)\label{def:completeP}\\
			 &= 2(c_{10} + I_1(\vC(X,t))c_{01})\vF(X,t) - 2c_{01}\vF(X,t)\vC(X,t)\nonumber\\
			 &\quad+g(\lambda_f(X,t))\vF(X,t)\va_0(X)\otimes\va_0(X)\nonumber
\end{align}

% \subsubsection{Detour}
% Prescribing $\vP$ will yield the behaviour of the overall system.
% The equation $\eqref{def:T}$ also holds for values in the current configuration, i.e.
% \[
% 	\vt(x,t,n) = \vsig(x,t)n,
% \]
% where $\vsig$ is the symmetric \e{Cauchy stress tensor}.

\section{FEM discretization}
\subsection{Domain decomposition, basis functions and master reference volume}
We specify a master hexahedron/volume/element $\Om = [-1,1]^3$ with corners
\begin{align*}
	\vX_{master} := \m{-1 & 1 & -1 & 1 & -1 & 1 & -1 &1\\-1 & -1 & 1 & 1 & -1 & -1 & 1 &1\\-1 & -1 & -1 & -1 & 1 & 1 & 1 &1\\}.
\end{align*}
This will be the ``natural'' order when referring to corners of the master volume, and we will stick with that order also for basis function indexing and all derived quantities.
All integrals will be performed on this master element $\Om$, where we apply a Gauss Quadrature with
\begin{align}
	X_1,\ldots,X_G  &\in \Om &&\text{Gauss points},\\
	w_1,\ldots,w_G  &\in \R &&\text{Gauss weights},
\end{align}
so that
\begin{align}
	\intom f(X)dX \approx \sumgp w_pf(X_p)\label{def:gaussintapprox}
\end{align}
for any integrable $f(X)$.
\subsubsection{Master basis functions}
\paragraph{Linear master basis functions}
For $\H^1$ we define the $n_b=8$ linear basis functions
\begin{align*}
	N_i:\Om &\to [0,1]\\
	N_i(X)  &= (1+(-1)^{i}X_1)(1+(-1)^{\ceil{\frac{i}{2}}}X_2)(1+(-1)^{\ceil{\frac{i}{4}}}X_3),\quad i=1\ldots n_b
\end{align*}
which amounts to trilinear functions equal to one in each corner of the hexahedron.
The set of functions has the property
\begin{align}
	\suml{i=1}{n_b}N_i &\equiv 1 & \text{i.e.}\quad \suml{i=1}{n_b}N_i(X) &= 1 \fo X\in\Om,\label{eq:basisfunsum}
\end{align}
which will be needed later.
\paragraph{Quadratic master basis functions}
For quadratic basis functions we need more Degrees of Freedom (DoF) in order to be able to satisfy \eqref{eq:basisfunsum}.
Well-known are $n_b=20$ or $n_b=27$ DoFs using extra edge-midpoint and mid-face locations.
We will use $n_b = 20$ with extra DoFs on the $12$ edge-midpoints.
Those be easily computed by a transformation matrix $T\in\R^{20 \times 8}$: 
\begin{lstlisting}
i = [1 2  2  3  4  4  5  5 6  7  7 8  9  9 10 10 11 11 12 12 13 14 14 15 16 16 17 17 18 19 19 20];  
j = [1 1  2  2  1  3  2  4 3  3  4 4  1  5 2  6  3  7  4  8  5  5  6  6  5  7  6  8  7  7  8  8];
s = [1 .5 .5 1 .5 .5 .5 .5 1 .5 .5 1 .5 .5 .5 .5 .5 .5 .5 .5 1  .5 .5 1  .5 .5 .5 .5 1  .5 .5 1];
T = sparse(i,j,s,20,8);
\end{lstlisting}
Then the extended ``corner'' set for the master hexahedron is $\vX_{master}T' \in\R^{20\times 3}$.
Now, for $\H^1$ we define the quadratic basis functions
\begin{align*}
	N_i:\Om &\to [0,1]\\
	N_1(X)  &= \frac{1}{8}(1-X_1)(1-X_2)(1-X_3)(-X_1-X_2-X_3-2)\\ % C1
    N_2(X)  &= \frac{1}{4}(1-X_1^2)(1-X_2)(1-X_3)\\ % E2
    N_3(X)  &= \frac{1}{8}(1+X_1)(1-X_2)(1-X_3)(X_1-X_2-X_3-2)\\ % C3
    N_4(X) &= \frac{1}{4}(1-X_2^2)(1-X_1)(1-X_3)\\ % E4
    N_5(X) &= \frac{1}{4}(1-X_2^2)(1+X_1)(1-X_3)\\ % E5
    N_6(X) &= \frac{1}{8}(1-X_1)(1+X_2)(1-X_3)(-X_1+X_2-X_3-2)\\ % C6
    N_7(X) &= \frac{1}{4}(1-X_1^2)(1+X_2)(1-X_3)\\ % E7
    N_8(X) &= \frac{1}{8}(1+X_1)(1+X_2)(1-X_3)(X_1+X_2-X_3-2)\\ % C8
    N_9(X) &= \frac{1}{4}(1-X_3^2)(1-X_1)(1-X_2)\\ % E9    
    N_{10}(X) &= \frac{1}{4}(1-X_3^2)(1+X_1)(1-X_2)\\ % E10
    N_{11}(X) &= \frac{1}{4}(1-X_3^2)(1-X_1)(1+X_2)\\ % E11
    N_{12}(X) &= \frac{1}{4}(1-X_3^2)(1+X_1)(1+X_2)\\ % E12
    N_{13}(X) &= \frac{1}{8}(1-X_1)(1-X_2)(1+X_3)(-X_1-X_2+X_3-2)\\ % C13
    N_{14}(X) &= \frac{1}{4}(1-X_1^2)(1-X_2)(1+X_3)\\ % E14
    N_{15}(X) &= \frac{1}{8}(1+X_1)(1-X_2)(1+X_3)(X_1-X_2+X_3-2)\\ %C15
    N_{16}(X) &= \frac{1}{4}(1-X_2^2)(1-X_1)(1+X_3)\\ % E16
    N_{17}(X) &= \frac{1}{4}(1-X_2^2)(1+X_1)(1+X_3)\\ % E17
    N_{18}(X) &= \frac{1}{8}(1-X_1)(1+X_2)(1+X_3)(-X_1+X_2+X_3-2)\\ % C18
    N_{19}(X) &= \frac{1}{4}(1-X_1^2)(1+X_2)(1+X_3)\\ % E19
    N_{20}(X) &= \frac{1}{4}(1+X_1)(1+X_2)(1+X_3)(X_1+X_2+X_3-2)]\\
\end{align*}
which amounts to triquadratic functions equal to one in each corner and edge-midpoints of the hexahedron.

\subsubsection{Domain decomposition}
Let $\Or$ be our domain of interest, decomposed into $\Or = \Omega_1\cup \ldots \cup \Omega_M$, where
$\Omega_m = [\vx^m_1,\ldots,\vx^m_{n_b}]$ is a deformed cube specified by the $n_b$ points $\vx_i^m\in\R^3$.
Let $\Ns := \{\vx_1,\ldots\vx_N\}$ denote the set of $N$ distinct nodes in
$\{\vx_1^1, \ldots, \vx_{n_b}^1, \vx_1^2, \ldots, \vx_{n_b}^{M-1}, \vx_1^M, \ldots, \vx_{n_b}^M\}$, numbered by first occurrence.

Using the notation
\begin{align}
	\vN(X) &:= \m{N_1(X) & \ldots & N_{n_b}(X)}^T\in\R^{n_b\times 1}\\
	\vX^m &:= \m{\vx^m_1 \ldots \vx^m_{n_b}} \in\R^{3\times n_b}
\end{align}
 we can specify an \e{isogeometric mapping} or diffeomorphism
\begin{align}
	\Phi_m : \Om &\to \Omega_m\\
	X &\mapsto \vX^m\vN(X)
\end{align}
which satisfies $\Omega_m = \Phi_m(\Or)$ and has the Jacobian
\begin{align}
	J\Phi_m(X) &= \m{\grad \Phi_{m_1}(X)\\\grad \Phi_{m_2}(X)\\\grad \Phi_{m_3}(X)\\} 
	= \vX^m\m{\grad N_1(X)\\ \vdots\\\grad N_{n_b}(X)} = \vX^m\nabla\vN(X) \in \R^{3\times 3}.\label{def:refbasis_jacobian}
\end{align}

\subsubsection{Node basis functions}
With the volume index set
\begin{align}
	%E_m &:= E(\Omega_m) := \{i\in\{1\ldots N\}~|~ \vx_i\in\Omega_m\}, \quad m=1\ldots M,\\
	V_k &:= \bigl\{i\in\{1\ldots M\}~|~ \vx_k\in\Omega_i\bigr\},&&k=1\ldots N
	%\Nk &:= \{i\in\{1\ldots N\}~|~ \vx_i\in E_m, m\in V_k\} = \{i ~|~ \supp\varphi_i \cap \supp\varphi_k \neq\es\},
\end{align}
we now define basis functions $\varphi_k$ on each node $k$ via
\begin{align}
	\varphi_k(X) &:= \begin{cases}
		N_{l(k,m)}(\Phi^{-1}_m(X)), & X\in\Omega_m, m\in V_k,\\
		0 & \text{else},	
	\end{cases}\label{def:referencebasisfun}\\
	l(k,m) &:= \{i\in\{1\ldots n_b\}~|~ N_i(\Phi_m^{-1}(\vx_k))=1\},\quad m=1\ldots M.
\end{align}
Here $l(k,m)$ refers to the local corner index of the basis function on $\Omega_m$ that equals one at $\vx_k$.


For the gradient of $\varphi_k$ we thus obtain
\begin{align}
	\grad\varphi_k(X) &\re{def:referencebasisfun} \grad (N_{l(k,m)}(\Phi^{-1}_m(X)))
					  = \grad N_{l(k,m)}(\Phi^{-1}_m(X))J\Phi^{-1}_m(X)\nonumber\\
					  &= \grad N_{l(k,m)}(\Phi^{-1}_m(X))(J\Phi_m(X))^{-1}\label{eq:gradvarphi}
\end{align}
for $X\in\Omega_m, m\in V_k$ (zero otherwise).
 
\subsubsection{Discrete integrals with node basis functions}
We define the shorthands
\begin{align}
	\pmp &:= \Phi_m(X_p)\\
	\jmp &:= |\det J\Phi_m(X_p)| = |\det \vX^m\nabla\vN(X_p)|.\label{def:jacshorthand}\\
	\dNkmp &:= \nabla\varphi_k(\pmp) \re{eq:gradvarphi} \left(J\Phi_m(X_p)\right)^{-T}\nabla \Nkmp)\in\R^3\label{}
\end{align}

With the above, and the transformation theorem and the local support of the basis function $\varphi_k$ we now have for any integrable $f(X,t)$ that
\begin{align}
   \intl{\Or}{}f(X,t)\varphi_k(X)dX &= \suml{m=1}{M}\intl{\Omega_m}{}f(X,t)\varphi_k(X)dX = \sumvk\intl{\Omega_m}{}f(X,t)\varphi_k(X)dX\nonumber\\
    &= \sumvk\intl{\Phi_m(\Omega)}{}f(X,t)\varphi_k(X)dX\nonumber\\
	&= \sumvk\intl{\Om}{}f(\Phi_m(X),t)\varphi_k(\Phi_m(X))|\det J\Phi_m(X)|dX\nonumber\\
	& \re{def:gaussintapprox} \sumvk\sumgp w_pf(\pmp,t)\varphi_k(\pmp)|\det J\Phi_m(X_p)|\nonumber\\
	& \re{def:referencebasisfun} \sumvk\sumgp w_pf(\pmp,t)N_{l(k,m)}(\Phi_m^{-1}(\pmp))|\det J\Phi_m(X_p)|\nonumber\\
	& \re{def:jacshorthand} \sumvk\sumgp w_pf(\pmp,t)\Nkmp\jmp\label{eq:fphidx}
\end{align}

Similarly we obtain for the gradient that
\begin{align}
	\intl{\Or}{}f(X,t)\nabla\varphi_k(X)dX &= \sumvk\sumgp w_pf(\pmp,t)\dNkmp\jmp\label{eq:fgradphidx}
\end{align}

\subsubsection{Boundary faces}
On $\Om$ we define the six faces $F_1^R \cup \ldots \cup F_6^R = \partial\Om$ with normals
\begin{align*}
	(\vec{N}^R_1 \ldots \vec{N}^R_6) &:= \m{-1 & 1 & 0 & 0 & 0 & 0\\0 & 0 & -1 & 1 & 0 & 0\\0 & 0 & 0 & 0 & -1 & 1},\\
	\vec{N}^R(X) &:\equiv \vec{N}^R_i && X\in F^R_i, i=1\ldots 6.
\end{align*}

For the given geometry $\Or$ define $B_m := \partial\Omega_m \cap \partial\Or, m=1\ldots M$, i.e. the respective parts of the global boundary on each sub-domain $\Omega_m$.
This corresponds to boundary parts $B^R_m := \Phi_m^{-1}(B_m) \subseteq \partial \Om$ on the master element.
If $I^R_m \subseteq \{1, \ldots, 6\}$ denotes the set of face indices that make up $B^R_m$, we can use the elementary faces to write
\begin{align*}
	B^R_m = \bigcup_{i\in I^R_m}F_i^R.
\end{align*}

The face normals $\vec{N}(X)$ at $X \in \partial \Omega_m$ are given by
\begin{align}
	\vec{N}(X) &= J\Phi_m(\Phi_m^{-1}(X))\vec{N}^R(\Phi_m^{-1}(X)) && X \in\partial\Omega_m\nonumber\\
			     &= J\Phi_m(X^R)\vec{N}^R(X^R) && X^R := \Phi_m^{-1}(X) \in\partial\Om\nonumber\\
				 &= J\Phi_m(X^R)\vec{N}^R_i && X^R \in F^R_i \subset \partial\Om.\label{eq:Normal_trans}
\end{align}

Assuming a suitable set of Gauss points $\{X_1^i,\ldots,X^i_{G_F}\}$ (and weights $w_1,\ldots,w_{G_F}$)
for 2D integration on each face $F^R_i$ of the master element $\Om$, we have for any $f(X,t)$ and $m = 1\ldots M$:
\begin{align*} %\intl{\partial\Phi_m(B^R_m)}{} f(X,t) \vec{N}_m(X)d\vec{N}_m  =
	\intl{B_m}{} f(X,t)\vec{N}(X)d\vec{N} &= \intl{\Phi_m(B^R_m)}{} f(X,t) \vec{N}(X)d\vec{N}\\
	&=\intl{B^R_m}{} f(\Phi_m(X),t) \vec{N}(\Phi_m(X))\left|\det(J{\Phi_m}_{|\partial\Om}(X))\right|d\vec{N}^R\\
	&=\suml{i\in I^R_m}{} \intl{F^R_i}{} f(\Phi_m(X),t) \vec{N}(\Phi_m(X))\left|\det(J{\Phi_m}_{|\partial\Om}(X))\right|d\vec{N}^R\\
	&\re{eq:Normal_trans}\suml{i\in I^R_m}{} \intl{F^R_i}{} f(\Phi_m(X),t) J\Phi_m(X)\vec{N}_i\left|\det(J{\Phi_m}_{|\partial\Om}(X))\right|d\vec{N}^R\\
	&=\suml{i\in I^R_m}{} \sumgp w_p f(\Phi_m(X^i_p),t) J\Phi_m(X^i_p)\vec{N}_i\left|\det(J{\Phi_m}_{|\partial\Om}(X^i_p))\right|
\end{align*}
 
Now, for $f(X,t) = \vP(X,t)\varphi_k(X)$ we obtain:
\begin{align*}
	\intorb \vP(X,t)\varphi_k(X)\vec{N}(X)d\vec{N} &= \suml{m=1}{M} \intl{\partial\Or \cap \partial\Omega_m}{}\vP(X,t)\varphi_k(X) \vec{N}(X)d\vec{N}\\
	&=\suml{m=1}{M} \intl{B_m}{}\vP(X,t)\varphi_k(X) \vec{N}(X)d\vec{N}\\
	&=\sumvk \intl{B_m}{}\vP(X,t)\varphi_k(X) \vec{N}(X)d\vec{N}\\
	&=\sumvk \suml{i\in I^R_m}{}\sumgp w_p \vP(X_p^i,t)\varphi_k(X_p^i) J\Phi_m(X_p^i)\vec{N}_i\left|\det(J{\Phi_m}_{|\partial\Om}(X_p^i))\right|.
\end{align*}


\subsection{Weak form and linear ansatz}
In the following we will use $\varphi/\psi$ to denote node ansatz functions with linear/quadratic master basis functions, respectively.
We define the spaces
\begin{align*}
	\H^1 &:= \langle \psi_1,\ldots,\psi_{N_P} \rangle \subset C^1(\Or)\\
	\H^2 &:= \langle \varphi_1,\ldots,\varphi_N \rangle \subset C^2(\Or)
\end{align*} 
For discretization we will use Taylor-Hood elements, which means $\chi\in\H^2, p\in\H^1$ via
\begin{align}
	\chi(X,t) &= \sumi \vc_i(t)\varphi_i(X), && \vc_i(t):[0,T] \to\R^3.\label{def:chi}\\
	p(X,t) &= \sumiP d_i(t)\psi_i(X), && d_i(t):[0,T] \to\R.
\end{align}
This means we have quadratic ansatz functions for positions, velocity and linear functions for pressure.
With \eqref{def:chi} we deduce the discrete quantities
\begin{align*}
	\vV(X,t) &= \d{\chi}{t}(X,t) = \sumi \vc_i'(t)\varphi_i(X)\\
	\vA(X,t) &= \d{\vV}{t}(X,t) = \sumi \vc_i''(t)\varphi_i(X)\\
	\vF(X,t) &= \d{\chi}{X}(X,t) = \m{\grad \chi_1(X,t)\\ \grad \chi_2(X,t)\\ \grad \chi_3(X,t)}
		   = \sumi\m{c_{i,1}\d{\varphi_i}{X_1} & \dots & c_{i,1}\d{\varphi_i}{X_3}\\
		   				\vdots & \ddots & \vdots\\
		   				c_{i,3}\d{\varphi_i}{X_1} & \dots & c_{i,3}\d{\varphi_i}{X_3}}\\
		   &=\sumi \vc_i(t)\cdot \grad\varphi_i(X) = \sumi \vc_i(t)\otimes \dpi.
\end{align*}

Now, equation \eqref{def:maineq} (without body forces) and the incompressibility constraint give the system 
\begin{align}
	\rho_0(X)\d{\vV}{t}(X,t) &= \divergence\vP(X,t) && \fo X\in\Or,\label{def:maineq_nobodyforce}\\
	J(X,t) &= 1 && \fo X\in\Or,\label{def:maineq_cond}
\end{align}
for which we have the spatial weak form
\begin{align}
	\intor\rho_0(X)\d{\vV}{t}(X,t)\varphi(X)dX &= \intor\divergence\vP(X,t)\varphi(X)dX &&\fo \varphi\in\H^2\\
	\intor (J(X,t)-1)\psi(X)dX &= 0 && \fo \psi\in\H^1
\end{align}
The right hand side of \eqref{def:maineq_nobodyforce} can be transformed as
\begin{align*}
	\intor \div{\vP(X,t)}\varphi(X) dX &= \intor \div{\vP(X,t)\varphi(X)} - \vP(X,t)\divergence\varphi(X) dX\\
		 &= \underbrace{\intorb \vP(X,t)\varphi(X) dN}_{=0} - \intor \vP(X,t)\divergence\varphi(X)dX\\
		 &= \intor -\vP(X,t)\divergence\varphi(X) dX.
\end{align*}
Now, to satisfy equations \eqref{def:maineq_nobodyforce} and \eqref{def:maineq_cond}, they must hold true for every node basis function. 
Consequently, we obtain a system of $3N+N_P$ equations
\begin{align}
	\intor \rho_0(X)\d{\vV}{t}(X,t)\varphi_k(X)dX + \intor\vP(X,t)\dpk dX &= \vnull, && \fo k=1\ldots N,\label{def:weakform}\\
	\intor (J(X,t)-1)\psi_k(X)dX &= 0, && \fo k=1\ldots N_P.\label{def:weakform_nb}
\end{align}

\subsubsection{Integrals of main equation parts}
In the following we fix a $k\in\{1,\ldots, N\}$.
Then, by \eqref{eq:fphidx}, the first summand of \eqref{def:weakform} reads as
\begin{align}
	\intor \rho_0(X)\d{\vV}{t}(X,t)\varphi_k(X)dX\nonumber
		&\re{eq:fphidx} \sumvk\sumgp w_p \rho_0(\pmp)\d{\vV}{t}(\pmp,t)\Nkmp\jmp\nonumber\\
		&= \sumvk\sumgp w_p \rho_0(\pmp)\sumi \vc_i''(t)\varphi_i(\pmp) \Nkmp\jmp\nonumber\\
		&= \sumi \vc_i''(t)\sumvk\sumgp w_p \rho_0(\pmp)N_{l(i,m)}(X_p) \Nkmp\jmp\label{def:discreteVphidX}
\end{align}
Similarly, the second summand of \eqref{def:weakform} writes as
\begin{align}
		\intor\vP(X,t)\dpk dX &\re{eq:fgradphidx} \sumvk\sumgp w_p\vP(\pmp,t)\dNkmp\jmp.\label{def:discretePgradphidX}
\end{align}
Finally, for $k\in\{1\ldots N_P\}$, condition \eqref{def:maineq_cond} formulates to
\begin{align}
	\intor (J(X,t)-1)\psi_k(X)dX &= \sumvk\sumgp w_p (\det\vF(\pmp,t)-1) \Nkmp\jmp\label{eq:weak_incomp_condition}\\
	&= \sumvk\sumgp w_p \left(\det\left(\sumi \vc_i(t)\otimes \divergence\varphi_i(\pmp)\right)-1\right) \Nkmp\jmp.\nonumber
\end{align}

\subsubsection{Local tensor evaluations}
The most elaborative part is to evaluate the stress tensor $\vP$ at all volumes and Gauss points.
We have from \eqref{def:completeP} that
\begin{align*}
		\vP(\pmp,t) &= p(\pmp,t)\vF^{-T}(\pmp,t) + 2(c_{10} + I_1(\vC(\pmp,t))c_{01})\vF(\pmp,t)\\
			 &- 2c_{01}\vF(\pmp,t)\vC(\pmp,t)+g(\lambda_f(\pmp,t))\vF(\pmp,t)\va_0(\pmp)\otimes\va_0(\pmp).
\end{align*}
In the following we derive representations of the various components of $\vP$, where we assume $X\in\Or$.
\begin{align*}
	\vF(\pmp,t) &= \sumi \vc_i(t)\otimes \dNimp,\\
	I_1(\vC(X,t)) &= \tr\vC(X,t) = \tr(\vF(X,t)^T\vF(X,t)) = \vF(X,t) : \vF(X,t)\\
	&= \sumi \vc_i(t)\otimes \dpi : \sumi \vc_i(t)\otimes \dpi\\
	&= \sumi\sumj \left(\vc_i(t)\otimes \dpi\right) : (\vc_j(t)\otimes \dpj)\\
	&= \sumi\sumj (\vc_i(t) \cdot \vc_j(t))(\dpi \cdot \dpj),\\
	\Rightarrow I_1(\vC(\pmp,t)) &= \sumi\sumj (\vc_i(t) \cdot \vc_j(t))(\dNimp \cdot \dNjmp),\\
\end{align*}
\begin{align*}
	\vF(X,t)^T\vF(X,t) &= \sumi\sumj \bigl(\vc_i(t)\otimes\dpi\bigr)^T\vc_j(t)\otimes\dpj\\
	 &= \sumi\sumj \bigl(\dpi\otimes\vc_i(t)\bigr)\vc_j(t)\otimes\dpj\\
	 &=\sumi\sumj (\vc_i(t)\cdot \vc_j(t))\dpi\otimes\dpj,\\
	\lambda_f(X,t)^2 &=I_4(\vC(X,t),\va_0(X)) = \vC(X,t):(\va_0(X)\otimes \va_0(X))\\
			&=  \vF(X,t)^T\vF(X,t) : (\va_0(X)\otimes \va_0(X))\\
			&=  \sumi\sumj (\vc_i(t)\cdot \vc_j(t))\bigl(\dpi\otimes\dpj\bigr) : (\va_0(X)\otimes \va_0(X))\\
			&=  \sumi\sumj (\vc_i(t)\cdot \vc_j(t))\left(\dpi\cdot \va_0(X)\right)(\dpj\cdot \va_0(X)),\\
	\Rightarrow \lambda_f(\pmp,t) &=
    		\left(\sumi\sumj (\vc_i(t)\cdot \vc_j(t))\left(\dNimp\cdot \va_0(\pmp)\right)(\dNjmp\cdot \va_0(\pmp))\right)^\frac{1}{2},\\
\end{align*}
\begin{align*}
    \vF(X,t)(\va_0(X)\otimes\va_0(X)) &= \sumi (\vc_i(t)\otimes \dpi)(\va_0(X)\otimes\va_0(X))\\
    &= \sumi (\vc_i(t)\otimes \va_0(X))(\dpi\cdot\va_0(X)),\\
    \Rightarrow \vF(\pmp,t)(\va_0(\pmp)\otimes\va_0(\pmp)) 
    	&= \sumi (\vc_i(t)\otimes \va_0(\pmp))(\dNimp\cdot\va_0(\pmp)).\\
\end{align*}

\begin{align*}
	\d{\lambda_f}{t}(X,t) &= \frac{1}{2\lambda_f(X,t)} \suml{i,j}{N} (\vc'_i(t)\cdot \vc_j(t) + \vc_i(t)\cdot \vc'_j(t))\left(\dpi\cdot a_0(X)\right)(\dpj\cdot a_0(X))\\
		&= \frac{1}{\lambda_f(X,t)}\suml{i,j}{N} (\vc_i(t)\cdot \vc'_j(t))\left(\dpi\cdot a_0(X)\right)(\dpj\cdot a_0(X))
\end{align*}
%     \vF(X,t)\vF(X,t)^T\vF(X,t) &= \sumk\vc_k(t)\otimes\dpk\sumi\sumj (\vc_i(t)\cdot \vc_j(t))\dpi\otimes\dpj\nonumber\\
%     	&= \suml{i,j,k}{N}(\vc_i(t)\cdot \vc_j(t))(\vc_k(t)\otimes\dpk)\dpi\otimes\dpj\nonumber\\
%     	&= \suml{i,j,k}{N}(\vc_i(t)\cdot \vc_j(t))(\dpk\cdot\dpi)(\vc_k(t)\otimes\dpj)\nonumber\\

% 	\vF(\pmp,t)\vC(\pmp,t) &= \vF(\pmp,t)\vF(\pmp,t)^T\vF(\pmp,t)\\
%     	&= \suml{i,j,k}{N}(\vc_i(t)\cdot \vc_j(t))(\divergence\varphi_k(\pmp)\cdot\divergence\varphi_i(\pmp))(\vc_k(t)\otimes\divergence\varphi_j(\pmp))\\
	%&= \suml{i,j,q\in E_m}{}(\vc_i(t)\cdot \vc_j(t))(\nabla N_{l(q,m)}(X_p)\cdot\nabla N_{l(i,m)}(X_p))(\vc_q(t)\otimes\nabla N_{l(j,m)}(X_p))\\
	%&= \suml{i,j\in E_m}{} (\vc_i(t)\cdot \vc_j(t))\left(\nabla N_{l(i,m)}(X_p)\cdot a_0(\pmp)\right)(\nabla N_{l(j,m)}(X_p))\cdot a_0(\pmp))\nonumber
	%&= \suml{i\in E_m}{} (\vc_i(t)\otimes \va_0(\pmp))(\nabla N_{l(i,m)}(X_p)\cdot\va_0(\pmp))\nonumber

\section{Formulation as System of ODEs}
We now collect the $N$ position coefficient vectors $\vc_i \in\R^3$ and pressure coefficients as
\begin{align*}
	\vu(t) &:= (\vc_1^T(t) \ldots \vc_N^T(t))^T \in\R^{3N},\\
	\vw(t) &:= (d_1(t) \ldots d_{N_P}(t))^T \in\R^{N_P}.
\end{align*}
We introduce the triple index
\begin{align}
	[i] &:= 3(i-1)+(1~2~3)^T, \text{ i.e.}\\
	\vu[i] &:= (\vu_{3(i-1)+1}~~\vu_{3(i-1)+2}~~\vu_{3(i-1)+3})^T\in\R^{3\times 1}
\end{align}
for the $x,y,z$ components of the $i$-th coefficient and hence identify
\[
	\vF(X,t) = \sumi \vc_i(t)\otimes \nabla\varphi_i(X) = \sumi \vu[i](t)\otimes \nabla\varphi_i(X) =: \vF(X,\vu(t)).
\]
also carrying the arguments through to $\vP(X,t) = \vP(X,\vu(t),\vw(t))$.
This allows us to define operators $\vK: \R^{3N\times N_P}\to\R^{3N}, \vg:\R^{3N}\to\R^{N_P}$ (by components) as
\begin{align}
	\vK[k](\vu(t),\vw(t)) &:= \intor\vP(X,\vu(t),\vw(t))\dpk dX\\
	&\re{def:discretePgradphidX} \sumvk\sumgp w_p\vP(\pmp,\vu(t),\vw(t))\dNkmp\jmp\in\R^3, && k=1\ldots N,\\
	\vg_k(\vu(t)) &:= \intor (\det\vF(X,\vu(t))-1)\psi_k(X)dX\\
	&\re{eq:weak_incomp_condition}\sumvk\sumgp w_p (\det\vF(\pmp,\vu(t))-1) \Nkmp\jmp, && k=1\ldots N_P.
\end{align}
Furthermore, from \eqref{def:discreteVphidX} we obtain a mass matrix $\vM\in\R^{3N\times 3N}$ via
\begin{align*}
	\vM\bigl[[i],[k]\bigr] &:= \m{1&0&0\\0&1&0\\0&0&1} \sumvk\sumgp w_p \rho_0(\pmp)N_{l(i,m)}(X_p) \Nkmp\jmp, && i,k=1\ldots N.
\end{align*}

Altogether, the system of equations \eqref{def:mainsys1_weak_kparts},\eqref{def:maineq_cond_weak_kparts} yields the constrained second order differential equation
or differential-algebraic equation (DAE)
\begin{align}
	\vM\vu''(t) + \vK(\vu(t),\vw(t)) &= \vnull,\\
		\text{s.t.}\quad \vg(\vu(t)) &= \vnull,
\end{align}
which can be transformed to a first-order system with the substitution $\vv(t) := \vu'(t)$:
\begin{align}
	\m{\vI & \vnull\\ \vnull & \vM}\m{\vu'(t)\\ \vv'(t)} &= \m{\vv(t) \\ -\vK(\vu(t),\vw(t))},\\
	\text{s.t.}\quad \vg(\vu(t))		&= \vnull.
\end{align}

\subsection{Time-discretization}
For the time-discretization we use the notation $\vu_i := \vu(t_i), t_i := i\Delta t, i=0\ldots N_t\in\N, \Delta t > 0$, similarly for $\vv,\vw$.
Using the standard backward-Euler scheme results in the implicit system
\begin{align}
	\frac{1}{\Delta t}\m{\vI & \vnull\\ \vnull & \vM}\m{\vu_{i+1}-\vu_i\\\vv_{i+1}-\vv_i} &= \m{\vv_{i+1}\\-\vK(\vu_{i+1},\vw_{i+1})},\\
	\vg(\vu_{i+1})		&= \vnull.
\end{align}
Writing the overall system as 
\begin{align}
	\vf([\vu; \vv; \vw]) &:= \m{\frac{1}{\Delta t}\m{\vI & \vnull\\ \vnull & \vM}\m{\vu-\vu_i\\\vv-\vv_i} + \m{-\vv\\\vK(\vu,\vw)}\\
		\vg(\vu)},
\end{align}
we apply the standard Newton-Iteration
\begin{align}
	[\vu; \vv; \vw]^1 &:= [\vu_i; \vv_i; \vw_i],\\
	[\vu; \vv; \vw]^{n+1} &:= [\vu; \vv; \vw]^{n} - \nabla \vf([\vu; \vv; \vw]^n)^{-1}\vf([\vu; \vv; \vw]^n), 
\end{align}
to find $\vu_{i+1}$ etc.
Here we have the Jacobian
\begin{align}
	 \nabla \vf([\vu; \vv; \vw]) &:= \m{\frac{1}{\Delta t} \vI & -\vI & \vnull & \Bigl\rbrace3N\\
	 									\nabla_{\vu}\vK(\vu,\vw) & \frac{1}{\Delta t}\vM & \nabla_{\vw}\vK(\vu,\vw) & \Bigl\rbrace3N\\
	 									\underbrace{\nabla_{\vu}\vg(\vu)}_{3N} & \underbrace{\vnull}_{3N} & \underbrace{\vnull}_{N_P} & \Bigl\rbrace N_P}.
\end{align}

\subsection{Derivation of nonlinearity Jacobian}
Here are some important formulas for matrix derivatives. We first define
\begin{align}
    \vU^j_i(X) & := \m{\ve_i\otimes\dpj}, i=1\ldots3, &
	\vU^j_1(X) &= \m{
            ~   & \dpj  & ~ \\
            0   & 0                 & 0 \\
            0   & 0                 & 0
          },\\
  \vU^j_2(X) &= \m{
            0   & 0                 & 0 \\
            ~   & \dpj  & ~ \\            
            0   & 0                 & 0
          },& 
  \vU^j_3(X) & = \m{
            0   & 0                 & 0 \\
            0   & 0                 & 0 \\
            ~   & \dpj  & ~            
          }.
\end{align}
\begin{align}
	\vF(X,\vu + h\ve[j]_i) &= \vF(X,\vu) + \m{\vnull \\ h\dpj^T\\ \vnull}\leftarrow i\nonumber\\
	&= \vF(X,\vu) + h\ve_i\otimes\dpj\\
	&= \vF(X,\vu) + h\vU^j_i(X),\qquad i=1\ldots3,\ve_i\in\R^3 \text{ unit vec}\nonumber\\
	\Rightarrow \d{}{u[j]_i}\vF(X,\vu) &= \vU^j_i(X),\qquad i=1\ldots3\nonumber\\
	(\vS+\Delta\vU)^{-1} &= \vS^{-1} - \vS^{-1}\Delta\vU\vS^{-1} + \O{|\Delta U|^2}\qquad\cite[p.57]{Bonet2008}\label{eq:matrixinversederivative}\\
	\vF(X,\vu + h\ve[j]_i)^{-1} &= \left(\vF(X,\vu) + h\vU^j_i(X)\right)^{-1}\nonumber\\
		&\re{eq:matrixinversederivative}\vF(X,\vu)^{-1} - h\vF(X,\vu)^{-1}\vU^j_i(X)\vF(X,\vu)^{-1} + \O{h^2}\nonumber\\
	\Rightarrow\d{}{u[j]_i}\vF^{-1}(X,\vu) &= -\vF(X,\vu)^{-1}\vU^j_i(X)\vF(X,\vu)^{-1}\nonumber\\
	\Rightarrow\d{}{u[j]_i}\vF^{-T}(X,\vu) &= -\vF(X,\vu)^{-T}(\vU^j_i(X))^{T}\vF(X,\vu)^{-T}\nonumber
\end{align}

In detail, we have
% \begin{align*}
% 	\d{\vF}{\vu[k]}(X,\vu) &= \nabla_{\vu[k]}\vF(X,\vu)\\
% 			&= \nabla_{\vu[k]}\sumi \vu[i] \otimes \dpi = \bullet\otimes\dpk\\
% 	\nabla_{\vu[k]}(\det\vF(X,\vu)-1) &= \det(\vF(X,\vu))\vF(X,\vu)^{-1}:\nabla_{\vu[k]}\vF(X,\vu)\\
% 				&= \det(\vF(X,\vu))\vF(X,\vu)^{-1}:\bullet\otimes\dpk
% \end{align*}
\subsubsection{Derivation of $\nabla_{\vu}\vK(\vu,\vw)$}
\begin{align*}
	\nabla_{\vu}\vK(\vu,\vw) &= \m{\nabla_{\vu}\vK[1](\vu,\vw)\\ \vdots \\ \nabla_{\vu}\vK[N](\vu,\vw)}
	 = \m{\nabla_{\vu[1]}\vK[1](\vu,\vw) & \ldots & \nabla_{\vu[N]}\vK[1](\vu,\vw)\\
	 	\vdots & \ddots & \vdots\\
	   \nabla_{\vu[1]}\vK[N](\vu,\vw) & \ldots & \nabla_{\vu[N]}\vK[N](\vu,\vw)}\\
	\nabla_{\vu[j]}\vK[k](\vu,\vw) &=\m{\d{\vK[k]}{\vu[j]_1}(\vu,\vw) & \ldots & \d{\vK[k]}{\vu[j]_3}(\vu,\vw)}\in\R^{3\times 3},\quad j,k=1\ldots N\\
	\d{\vK[k]}{\vu[j]_i}(\vu,\vw) &= \d{}{\vu[j]_i} \intor\vP(X,\vu,\vw)\dpk dX\\
		&=  \intor\d{\vP}{\vu[j]_i}(X,\vu,\vw) \dpk dX\\
		&= \sumvk\sumgp w_p \d{\vP}{\vu[j]_i}(\pmp,\vu,\vw) \dNkmp\jmp\in\R^3
\end{align*}
Recall from \eqref{def:completeP} that
\begin{align*}
\vP(X,\vu,\vw) &= p(X,\vw)\vF^{-T}(X,\vu) + 2(c_{10} + I_1(\vC(X,\vu))c_{01})\vF(X,\vu) - 2c_{01}\vF(X,\vu)\vC(X,\vu)\\
	     &\quad+g(\lambda_f(X,\vu))\vF(X,\vu)\va_0(X)\otimes\va_0(X)
\end{align*}
we consider the single summands of $\vP$ to derive $\d{\vP}{\vu[j]_i}(X,\vu,\vw)$:
\begin{align*}
	 \d{}{\vu[j]_i}p(X,\vw)\vF^{-T}(X,\vu) &= -p(X,\vw)\vF(X,\vu)^{-T}(\dpj\otimes\ve_i)\vF(X,\vu)^{-T}\in\R^{3\times 3}
\end{align*}
\begin{align*}
	\nabla_{\vu[j]}I_1(\vC(X,\vu)) &\re{eq:I1_basis} \nabla_{\vu[j]}\suml{i,l}{N} (\vu[i] \cdot \vu[l])(\dpi \cdot \divergence\varphi_l(X))\\
	&= 2\suml{i}{N} \vu[i](\dpi \cdot \dpj) \in\R^3
\end{align*}
\begin{align*}	 
	 &\d{}{\vu[j]_i}2(c_{10} + I_1(\vC(X,\vu))c_{01})\vF(X,\vu)\\
	 =& 2\d{}{\vu[j]_i}I_1(\vC(X,\vu))c_{01}\vF(X,\vu) + 2(c_{10} + I_1(\vC(X,\vu))c_{01})\d{}{\vu[j]_i}\vF(X,\vu)\\
	 =& 2c_{01}\suml{l}{N} \vu[l]_i(\divergence\varphi_l(X) \cdot \dpj)\vF(X,\vu)\\
	 &+ 2(c_{10} + I_1(\vC(X,\vu))c_{01})\vU^j_i(X)\in\R^{3\times 3}
\end{align*}
\begin{align*}
	\d{}{\vu[j]_i}\vC(X,\vu) &= \d{}{\vu[j]_i}\suml{m,l}{N} (\vu[m] \cdot \vu[l])(\divergence\varphi_m(X) \otimes \divergence\varphi_l(X))\\
	&= 2\suml{l}{N} \vu[l]_i(\dpj \otimes \divergence\varphi_l(X))\in\R^{3\times 3}
\end{align*}
\begin{align*}
	&\d{}{\vu[j]_i}2c_{01}\vF(X,\vu)\vC(X,\vu)\\
	=& 2c_{01}\left(\left(\d{}{\vu[j]_i}\vF(X,\vu)\right)\vC(X,\vu) + \vF(X,\vu)\d{}{\vu[j]_i}\vC(X,\vu)\right)\\
	=& 2c_{01}\left(\left(\vU^j_i(X)\right)\vC(X,\vu)
	+ 2\vF(X,\vu)\suml{l}{N} \vu[l]_i(\dpj \otimes \divergence\varphi_l(X))\right)\in\R^{3\times 3}
\end{align*}
\begin{align*}
	\nabla_{\vu[j]}\lambda_f(X,\vu) &= \nabla_{\vu[j]}\sqrt{\suml{m,l}{N} (\vu[m] \cdot \vu[l])\left(\divergence\varphi_m(X)\cdot a_0(X)\right)(\divergence\varphi_l(X)\cdot a_0(X))}\\
	&= \frac{1}{2\lambda_f(X,\vu)} \nabla_{\vu[j]}\suml{m,l}{N} (\vu[m] \cdot \vu[l])\left(\divergence\varphi_m(X)\cdot a_0(X)\right)(\divergence\varphi_l(X)\cdot a_0(X))\\
	&= \frac{1}{2\lambda_f(X,\vu)} 2\suml{l}{N} \vu[l]\left(\dpj\cdot a_0(X)\right)(\divergence\varphi_l(X)\cdot a_0(X))\\
	&= \frac{1}{\lambda_f(X,\vu)} \suml{l}{N} \vu[l]\left(\dpj\cdot a_0(X)\right)(\divergence\varphi_l(X)\cdot a_0(X))
\end{align*}
\begin{align*}
	  & \d{}{\vu[j]_i} g(\lambda_f(X,\vu))\vF(X,\vu)\va_0(X)\otimes\va_0(X)\\
	 =& \left( \d{g}{\lambda_f}(\lambda_f(X,\vu))\d{}{\vu[j]_i}\lambda_f(X,\vu)\vF(X,\vu) + g(\lambda_f(X,\vu))\ve_i\otimes\dpj\right)\va_0(X)\otimes\va_0(X) 
\end{align*}
\begin{align*}
	  f_l'(\lambda) &\re{def:fl} \begin{cases}
		\frac{12.5}{\lfo}\left(1-\frac{\lambda_f}{\lfo}\right) & 0.6 \leq \frac{\lambda_f}{\lfo} \leq 1.4\\ 
		0 & \text{else}
	\end{cases}\\	
	   \d{g}{\lambda_f} &\re{def:g} \d{}{\lambda_f}\left(\frac{b_1}{\lambda_f^2}\left(\lambda_f^{d_1} - 1\right)
		+\frac{p^{max}}{\lambda_f}f_l(\lambda_f)\gamma\left(\alpha,\d{\lambda_f}{t}\right)\right)\\
		&= -\frac{b_1}{\lambda_f^3}\left(\lambda_f^{d_1} - 1\right) + \frac{b_1}{\lambda_f^2}(d_1-1)\lambda_f^{d_1-1}
		+p^{max}\gamma\left(\alpha,\d{\lambda_f}{t}\right) \left(-\frac{1}{\lambda_f^2}f_l(\lambda_f) + \frac{1}{\lambda_f}f_l'(\lambda_f)\right)\\
		&= \frac{b_1}{\lambda_f^3}\left((d_1-1)\lambda_f^{d_1} - (\lambda_f^{d_1} - 1)\right)
		+p^{max}\gamma\left(\alpha,\d{\lambda_f}{t}\right) \left(-\frac{1}{\lambda_f^2}f_l(\lambda_f) + \frac{1}{\lambda_f}f_l'(\lambda_f)\right)\\
		&= \frac{b_1}{\lambda_f^3}\left((d_1-2)\lambda_f^{d_1} + 2\right)
		+p^{max}\gamma\left(\alpha,\d{\lambda_f}{t}\right) \left(-\frac{1}{\lambda_f^2}f_l(\lambda_f) 
		+ \underbrace{\frac{12.5}{\lambda_f\lfo}\left(1-\frac{\lambda_f}{\lfo}\right)}_{\neq 0\text{ if }0.6 \leq \frac{\lambda_f}{\lfo} \leq 1.4}\right)\\
		&= \frac{b_1}{\lambda_f^3}\left((d_1-2)\lambda_f^{d_1} + 2\right)
		+ \frac{p^{max}}{\lambda_f^2}\gamma\left(\alpha,\d{\lambda_f}{t}\right) \left( 
		\underbrace{12.5\frac{\lambda_f}{\lfo}\left(1-\frac{\lambda_f}{\lfo}\right)}_{\neq 0\text{ if }0.6 \leq \frac{\lambda_f}{\lfo} \leq 1.4}-f_l(\lambda_f)\right)
\end{align*}

\subsubsection{Derivation of $\nabla_{\vw}\vK(\vu,\vw)$}
\begin{align*}
	\nabla_{\vw}\vK(\vu,\vw) &= \m{\nabla_{\vw}\vK[1](\vu,\vw)\\ \vdots \\ \nabla_{\vw}\vK[N](\vu,\vw)}
	 = \m{\d{\vK[1]}{\vw_1}(\vu,\vw) & \ldots & \d{\vK[1]}{\vw_N}(\vu,\vw)\\
	 	\vdots & \ddots & \vdots\\
	   \d{\vK[N]}{\vw_1}(\vu,\vw) & \ldots & \d{\vK[N]}{\vw_N}(\vu,\vw)}\\
	\d{\vK[k]}{\vw_i}(\vu,\vw) &= \d{}{\vw_i} \intor\vP(X,\vu,\vw)\dpk dX\\
		&=  \intor\d{}{\vw_i} \left[p\vF^{-T}(X,\vu) +[\ldots] \right] \dpk dX\\
		&=  \intor\d{}{d_i} \left[\sumi d_j\psi_j(X)\vF^{-T}(X,\vu) +[\ldots] \right] \dpk dX\\
		&=  \intor \psi_i(X)\vF^{-T}(X,\vu)\dpk dX\\
		&= \sumvk\sumgp w_p \psi_i(\pmp)\vF^{-T}(\pmp,\vu) \dNkmp\jmp\in\R^3
\end{align*}

\subsubsection{Derivation of $\nabla_{\vu}\vg(\vu)$}
\begin{align*}
	\nabla_{\vu}\vg(\vu) &= \m{\nabla_{\vu}\vg_1(\vu)\\ \vdots \\ \nabla_{\vu}\vg_M(\vu)}
	 = \m{\d{\vg_1}{\vu[1]}(\vu) & \ldots & \d{\vg_1}{\vu[N]}(\vu)\\
	 	\vdots & \ddots & \vdots\\
	   \d{\vg_M}{\vu[1]}(\vu) & \ldots & \d{\vg_M}{\vu[N]}(\vu)} \in\R^{M \times 3N}\\
	\nabla_{\vu[j]}\vg_k(\vu) &= \nabla_{\vu[j]}\intor (\det\vF(X,\vu)-1)\psi_k(X)dX\qquad\in\R^{1\times 3}\\
		&\re{eq:weak_incomp_condition}  \sumvk\sumgp w_p \nabla_{\vu[j]}\det\vF(\pmp,\vu)\Nkmp\jmp\\
		&= \sumvk\sumgp w_p \det\vF(\pmp,\vu)\m{\vF(\pmp,\vu)^{-T}:\vU(\pmp)^j_1\\\vF(\pmp,\vu)^{-T}:\vU(\pmp)^j_2\\\vF(\pmp,\vu)^{-T}:\vU(\pmp)^j_3}^TN_{l(k,m)}(X_p)\jmp\\
		&= \sumvk\sumgp w_p \det\vF(\pmp,\vu)\m{\tr(\vF(\pmp,\vu)^{-1}\vU(\pmp)^j_1)\\\tr(\vF(\pmp,\vu)^{-1}\vU(\pmp)^j_2)\\\tr(\vF(\pmp,\vu)^{-1}\vU(\pmp)^j_3)}^TN_{l(k,m)}(X_p)\jmp\in\R^{1\times 3}
\end{align*}





\newpage
\section{Appendix, additional derivations}

\subsection{Calculation of $\nabla_{\Bc_m}\left\{\det \BF(X,t)\right\}$}
Calculation of the determinant:
\alg{
  \nabla_{\Bc_m}\left\{\det \BF(X,t)\right\}
  &=\nabla_{\Bc_m}\left\{\det \left( \sumlim{i=1}{N}{\Bc_i(t) \otimes \nabla \varphi_i(X)} \right) \right\} \nonumber\\
  &=\nabla_{\Bc_m}\left\{\det \left( \sumlim{i=1}{N}{
      \begin{bmatrix}
        c_{i1} \varphi_{i,1}  & c_{i1} \varphi_{i,2}  & c_{i1} \varphi_{i,3} \\
        c_{i2} \varphi_{i,1}  & c_{i2} \varphi_{i,2}  & c_{i2} \varphi_{i,3} \\
        c_{i3} \varphi_{i,1}  & c_{i3} \varphi_{i,2}  & c_{i3} \varphi_{i,3}
      \end{bmatrix}
    } \right) \right\} \nonumber\\
  &=\nabla_{\Bc_m}\Bigg\{
      \left( \sumlim{i=1}{N}{c_{i1}\varphi_{i,1}}\right) \left[
        \left( \sumlim{i=1}{N}{c_{i2}\varphi_{i,2}}\right) \left( \sumlim{i=1}{N}{c_{i3}\varphi_{i,3}}\right)
        - \left( \sumlim{i=1}{N}{c_{i2}\varphi_{i,3}}\right) \left( \sumlim{i=1}{N}{c_{i3}\varphi_{i,2}}\right)
      \right]
      \nonumber\\
      &\hspace{11.5mm} -\left( \sumlim{i=1}{N}{c_{i1}\varphi_{i,2}}\right) \left[
        \left( \sumlim{i=1}{N}{c_{i2}\varphi_{i,1}}\right) \left( \sumlim{i=1}{N}{c_{i3}\varphi_{i,3}}\right)
        - \left( \sumlim{i=1}{N}{c_{i2}\varphi_{i,3}}\right) \left( \sumlim{i=1}{N}{c_{i3}\varphi_{i,1}}\right)
      \right]
      \nonumber\\
      &\hspace{11.5mm} +\left( \sumlim{i=1}{N}{c_{i1}\varphi_{i,3}}\right) \left[
        \left( \sumlim{i=1}{N}{c_{i2}\varphi_{i,1}}\right) \left( \sumlim{i=1}{N}{c_{i3}\varphi_{i,2}}\right)
        - \left( \sumlim{i=1}{N}{c_{i2}\varphi_{i,2}}\right) \left( \sumlim{i=1}{N}{c_{i3}\varphi_{i,1}}\right)
      \right]
    \Bigg\}\nonumber\\
  &=\nabla_{\Bc_m}\Bigg\{
  \sumlim{i,j,k=1}{N}{c_{i1} \varphi_{i,1} \, c_{j2} \varphi_{j,2} \, c_{k3}\varphi_{k,3}}
    -\sumlim{i,j,k=1}{N}{c_{i1} \varphi_{i,1} \, c_{j2} \varphi_{j,3} \, c_{k3}\varphi_{k,2}}
    \nonumber\\
    &\hspace{11.5mm}-\sumlim{i,j,k=1}{N}{c_{i1} \varphi_{i,2} \, c_{j2} \varphi_{j,1} \, c_{k3}\varphi_{k,3}}
    +\sumlim{i,j,k=1}{N}{c_{i1} \varphi_{i,2} \, c_{j2} \varphi_{j,3} \, c_{k3}\varphi_{k,1}}
    \nonumber\\
    &\hspace{11.5mm}+\sumlim{i,j,k=1}{N}{c_{i1} \varphi_{i,3} \, c_{j2} \varphi_{j,1} \, c_{k3}\varphi_{k,2}}
    -\sumlim{i,j,k=1}{N}{c_{i1} \varphi_{i,3} \, c_{j2} \varphi_{j,2} \, c_{k3}\varphi_{k,1}} 
    \Bigg\}\nonumber\\
  &=\nabla_{\Bc_m}\Bigg\{
    \sumlim{i,j,k=1}{N}{
    c_{i1} c_{j2} c_{k3} (
      \varphi_{i,1} \varphi_{j,2} \varphi_{k,3}-\varphi_{i,1} \varphi_{j,3} \varphi_{k,2}-\varphi_{i,2} \varphi_{j,1} \varphi_{k,3}
      }\nonumber\\
      &\hspace{20mm}+\varphi_{i,2} \varphi_{j,3} \varphi_{k,1}+\varphi_{i,3} \varphi_{j,1} \varphi_{k,2}
      -\varphi_{i,3} \varphi_{j,2} \varphi_{k,1} )
    \Bigg\}\nonumber\\
  &=\nabla_{\Bc_m}\Bigg\{
      \sumlim{i,j,k=1}{N}{
      c_{i1} c_{j2} c_{k3} 
      \det  \begin{bmatrix}
              \varphi_{i,1} & \varphi_{i,2} & \varphi_{i,3} \\
              \varphi_{j,1} & \varphi_{j,2} & \varphi_{j,3} \\
              \varphi_{k,1} & \varphi_{k,2} & \varphi_{k,3}
            \end{bmatrix}
      }
    \Bigg\} \nonumber\\
  &=\nabla_{\Bc_m}\Bigg\{
      \sumlim{i,j,k=1}{N}{
      c_{i1}(t) c_{j2}(t) c_{k3}(t)
      \det \begin{bmatrix}
              \nabla \varphi_{i}(X) \\
              \nabla \varphi_{j}(X) \\
              \nabla \varphi_{k}(X)
            \end{bmatrix}
      }
    \Bigg\}
}
%
%-------
\newpage
%-------
%
Calculation of the gradient:
\alg{
  \nabla_{\Bc_m}\left\{\det \BF(X,t)\right\}
  &=\nabla_{\Bc_m}\Bigg\{
      \sumlim{i,j,k=1}{N}{
      c_{i1}(t) c_{j2}(t) c_{k3}(t)
      \det \begin{bmatrix}
              \nabla \varphi_{i}(X) \\
              \nabla \varphi_{j}(X) \\
              \nabla \varphi_{k}(X)
            \end{bmatrix}
      }
    \Bigg\} \nonumber\\
  &=\sumlim{i,j,k=1}{N}{\left(
      \begin{bmatrix}
        \delby{c_{i1}}{c_{m1}}c_{j2}c_{k3} + c_{i1} \cancel{\delby{c_{j2}}{c_{m1}}} c_{k3} 
          + c_{i1}c_{j2}\cancel{\delby{c_{k3}}{c_{m1}}} \\
        \cancel{\delby{c_{i1}}{c_{m2}}}c_{j2}c_{k3} + c_{i1} \delby{c_{j2}}{c_{m2}} c_{k3} 
          + c_{i1}c_{j2}\cancel{\delby{c_{k3}}{c_{m2}}} \\
        \cancel{\delby{c_{i1}}{c_{m3}}}c_{j2}c_{k3} + c_{i1} \cancel{\delby{c_{j2}}{c_{m3}}} c_{k3} 
        + c_{i1}c_{j2}\delby{c_{k3}}{c_{m3}}
      \end{bmatrix}
      \det \begin{bmatrix}
              \nabla \varphi_{i}(X) \\
              \nabla \varphi_{j}(X) \\
              \nabla \varphi_{k}(X)
            \end{bmatrix}
      \right)}
      \nonumber\\
  &=\sumlim{i,j,k=1}{N}{\left(
      \begin{bmatrix}
        \delta_{im}c_{j2}c_{k3}\\
        c_{i1}\delta_{jm}c_{k3}\\
        c_{i1}c_{j2}\delta_{km}
      \end{bmatrix}
      \det \begin{bmatrix}
              \nabla \varphi_{i}(X) \\
              \nabla \varphi_{j}(X) \\
              \nabla \varphi_{k}(X)
            \end{bmatrix}
      \right)}
      \nonumber\\
  &=\begin{pmatrix}
      \sumlim{j,k=1}{N}{
        c_{j2}c_{k3} \det \begin{bmatrix}
                            \nabla \varphi_{m}(X) \\
                            \nabla \varphi_{j}(X) \\
                            \nabla \varphi_{k}(X)
                          \end{bmatrix}
      } \\
      \sumlim{i,k=1}{N}{
        c_{i1}c_{k3} \det \begin{bmatrix}
                            \nabla \varphi_{i}(X) \\
                            \nabla \varphi_{m}(X) \\
                            \nabla \varphi_{k}(X)
                          \end{bmatrix}
      } \\
      \sumlim{i,j=1}{N}{
        c_{i1}c_{j2} \det \begin{bmatrix}
                            \nabla \varphi_{i}(X) \\
                            \nabla \varphi_{j}(X) \\
                            \nabla \varphi_{m}(X)
                          \end{bmatrix}
      }            
    \end{pmatrix}
    \nonumber\\
  &=\sumlim{i,j=1}{N}{
      \begin{pmatrix}
        c_{i2}c_{j3}\det \begin{bmatrix}
                            \nabla \varphi_{m}(X) \\
                            \nabla \varphi_{i}(X) \\
                            \nabla \varphi_{j}(X)
                          \end{bmatrix}
        \\
        c_{i1}c_{j3}\det \begin{bmatrix}
                            \nabla \varphi_{i}(X) \\
                            \nabla \varphi_{m}(X) \\
                            \nabla \varphi_{j}(X)
                          \end{bmatrix}
        \\
        c_{i1}c_{j2}\det \begin{bmatrix}
                            \nabla \varphi_{i}(X) \\
                            \nabla \varphi_{j}(X) \\
                            \nabla \varphi_{m}(X)
                          \end{bmatrix}
      \end{pmatrix}
    }
    =:\ \begin{pmatrix}
          \nabla_{c_{m1}}\left\{\det \BF(X,t)\right\} \\
          \nabla_{c_{m2}}\left\{\det \BF(X,t)\right\} \\
          \nabla_{c_{m3}}\left\{\det \BF(X,t)\right\}                  
        \end{pmatrix}
}
%
%-------
\newpage
%-------
%
It holds that:
\beq{}
  \boxed{
  \det(\BS+\BU) = \det \BS + \det \BS (\BS^{T-1}:\BU) \,.
  }
\eeq
Furthermore, for the inverse of a matrix/tensor, it holds
\beq{}
  \BA^{-1} = (\det \BA)^{-1} \textnormal{adj}\BA
  \Longrightarrow
  \BA^{T-1} = (\det \BA)^{-1} \textnormal{cof}\BA \,,
\eeq
where
\beq{}
  \textnormal{cof}\BA = \textnormal{cof}
    \begin{pmatrix}
      A_{11}  & A_{12}  & A_{13} \\
      A_{21}  & A_{22}  & A_{23} \\
      A_{31}  & A_{32}  & A_{33}
    \end{pmatrix}
    = \begin{pmatrix}
        A_{22}A_{33}-A_{23}A_{32} & A_{23}A_{31}-A_{21}A_{33} & A_{21}A_{32}-A_{22}A_{31} \\
        A_{13}A_{32}-A_{12}A_{33} & A_{11}A_{33}-A_{13}A_{31} & A_{12}A_{31}-A_{11}A_{32} \\
        A_{12}A_{23}-A_{13}A_{22} & A_{13}A_{21}-A_{11}A_{23} & A_{11}A_{22}-A_{12}A_{21}
      \end{pmatrix} \,.
\eeq
"{\bf \underline{Lemma}}": \par
For $\BS:=\BF$ and $\BU:=\BU^m_l$, $l=1,2,3$, with
\beq{}
  \BU^m_1:= \begin{pmatrix}
            ~   & \nabla \varphi_m  & ~ \\
            0   & 0                 & 0 \\
            0   & 0                 & 0
          \end{pmatrix},\,
  \BU^m_2:= \begin{pmatrix}
            0   & 0                 & 0 \\
            ~   & \nabla \varphi_m  & ~ \\            
            0   & 0                 & 0
          \end{pmatrix},\,
  \BU^m_3:= \begin{pmatrix}
            0   & 0                 & 0 \\
            0   & 0                 & 0 \\
            ~   & \nabla \varphi_m  & ~            
          \end{pmatrix},          
\eeq
it holds that
\beq{}
  \boxed{
  \nabla_{c_{ml}}\left\{\det \BF(X,t)\right\} = \det \BF \left(\BF^{T-1}:\BU^m_l\right) \,.
  }
\eeq
"{\bf \underline{Proof}}": \par
Setting $l:=1$ yields:
\alg{
  \det \BF \left(\BF^{T-1}:\BU^m_1\right)
  &= \cancel{\det \BF (\det \BF)^{-1}} \textnormal{cof}\BF : \BU^m_1 \nonumber\\
  &= \textnormal{cof}\BF 
        \begin{pmatrix}
           ~   & \nabla \varphi_m  & ~ \\
           0   & 0                 & 0 \\
           0   & 0                 & 0
        \end{pmatrix}
    \nonumber\\
  &= \varphi_{m,1} (F_{22}F_{33}-F_{23}F_{32}) +
      \varphi_{m,2} (F_{23}F_{31}-F_{21}F_{33}) +
      \varphi_{m,3} (F_{21}F_{32}-F_{22}F_{31})
    \nonumber\\
  &= \varphi_{m,1} \left(
        \sumlim{i,j=1}{N}{c_{i2}\varphi_{i,2}c_{j3}\varphi_{j,3}}
        - \sumlim{i,j=1}{N}{c_{i2}\varphi_{i,3}c_{j3}\varphi_{j,2}}
      \right) \nonumber\\
    &\hspace{6mm}+\varphi_{m,2} \left(
        \sumlim{i,j=1}{N}{c_{i2}\varphi_{i,3}c_{j3}\varphi_{j,1}}
        - \sumlim{i,j=1}{N}{c_{i2}\varphi_{i,1}c_{j3}\varphi_{j,3}}
      \right)\nonumber\\
    &\hspace{6mm}+\varphi_{m,3} \left(
        \sumlim{i,j=1}{N}{c_{i2}\varphi_{i,1}c_{j3}\varphi_{j,2}}
        - \sumlim{i,j=1}{N}{c_{i2}\varphi_{i,2}c_{j3}\varphi_{j,1}}
      \right) \nonumber\\
  &= \sumlim{i,j=1}{N}{c_{i2}c_{j3}
      (
      \varphi_{m,1}\varphi_{i,2} \varphi_{j,3}-\varphi_{m,1}\varphi_{i,3} \varphi_{j,2}+\varphi_{m,2}\varphi_{i,3} \varphi_{j,1} 
      }\nonumber\\
      &\hspace{11mm}-\varphi_{m,2}\varphi_{i,1} \varphi_{j,3}+\varphi_{m,3}\varphi_{i,1} \varphi_{j,2} - \varphi_{m,3}\varphi_{i,2} \varphi_{j,1}
      ) \nonumber\\
  &= \sumlim{i,j=1}{N}{c_{i2}c_{j3} 
      \det \begin{bmatrix}
              \nabla \varphi_{m} \\
              \nabla \varphi_{i} \\
              \nabla \varphi_{j}           
            \end{bmatrix}
     }
  = \nabla_{c_{m1}}\left\{\det \BF(X,t)\right\} \,.
}



\bibliographystyle{plain}
\bibliography{cbm_library}

\end{document}
