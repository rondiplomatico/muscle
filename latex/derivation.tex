Assuming a mass $\rho_0(X)$ at each point we define the total momentum
\[
 	\vL(t) := \intor \rho_0(X)\vV(X,t) dX,
\] 
for which we postulate the balance equation
\begin{align}
	\force(t) \stackrel{!}{=} \d{\vL}{t}(t) = \vL'(t) = \intor \rho_0(X)\d{\vV}{t}(X,t) dX = \intor \rho_0(X)\vA(X,t) dX,\label{def:bal_eq}
\end{align}
given \e{resultant forces} $\force(t)$.

\subsection{Structure of force}
We assume the forces $\force(t)$ to be composed of two different sources: Forces on the boundary and body forces.
The body forces $\vB(X,t)$ measures the force per unit reference volume on $X$ at time $t$.
These forces are self-weight or gravity, for example.
Further, we assume to have \e{traction vectors} $\vT(X,t,N)$ (first Piola-Kirchhoff traction vector) that indicates the force working per unit surface area with normal $N$ 
at the point $X\in\partial\Or$ at time $t$.
Then the resultant forces are given as
\begin{align}
	\force(t) := \intorb \vT(X,t,N)dN + \intor \vB(X,t)dX \label{def:force}
\end{align}

According to \e{Cauchy's stress theorem}, we can express tractions as tensor product
\begin{align}
	\vT(X,t,N) &= \vP(X,t)N,\label{def:T}
\end{align}
where $\vP$ denotes the \e{first Piola-Kirchhoff stress tensor}.
With this we have, following Gauss integral theorem,
\begin{align}
	\intorb \vT(X,t,N)dN &= \intorb \vP(X,t)NdN = \intor \divergence\vP(X,t)dX.\label{eq:div_form_traction}
\end{align}
Using representation \eqref{eq:div_form_traction} in the force composition 
\eqref{def:force} yields the following form of the balance equation \eqref{def:bal_eq}:
\begin{align}
	\intor \rho_0(X)\d{\vV}{t}(X,t) dX &= \intor \divergence\vP(X,t) + \vB(X,t) dX
\end{align}
As this balance must also be satisfied for each subvolume $\Omega_t\subset\Or$, we actually obtain a pointwise or local form as
\begin{align}
	\rho_0(X)\d{\vV}{t}(X,t) &= \divergence\vP(X,t) + \vB(X,t) \qquad \fo X\in\Or\label{def:maineq}
\end{align}

\subsection{Stress tensor definition}
We will actually assume
\begin{align}
	\vP(X,t) &:= \vF(X,t)\vS(X,t),
\end{align}
where $\vS(X,t)$ stands for the \e{second Piola-Kirchhoff stress tensor}.
Now, most generally one makes the assumption that any stress tensor depends on the location and the current deformation, i.e.
\[
	\vS(X,t) = \tilde{\Sfun}(\vF(X,t),X),\quad \hat{\Sfun}:\R^{3\times 3} \to \R^{3\times 3}.
\]
We assume \e{homogeneous} material, which means that a direct spatial dependence is not given and we thus omit the argument.
Further, due to the principle of material frame-indifference \cite[p.198]{Holzapfel2000}, $\vS$
may actually only depend on the rotation-invariant part of $\vF$.
For any fixed $X,t$ we can decompose $\vF = \vR\vU$ (\cite[p.85]{Holzapfel2000}) with a
\e{rotation} part $\vR$ with $\vR^T\vR = \vI$ and \e{stretch} part $\vU = \vU^T$.
Thus $\vS$ may only depend on $\vU$; since
\[
	\vC = \vF^T\vF = (\vR\vU)^T\vR\vU = \vU^T\vR^T\vR\vU = \vU\vU = \vU^2, 
\]
we will assume that
\[
	\vS(X,t) = \Sfun(\vC(X,t))
\]
for suitable tensor function $\Sfun$.
We also introduce the concept of a \e{fibre direction} $\va_0(X) \in\R^3$, where we will omit the argument $X$ in the following.
Then, for \e{hyperelastic and transversely isotropic} material we have the definition
\[
	\vS(X,t) = \Sfun(\vC(X,t)) := 2\d{\Psi}{\vC}(\vC(X,t),\va_0(X)),
\]
where $\Psi$ is called \e{strain-energy} function, see \cite[p.207]{Holzapfel2000}.

\subsubsection{Invariants}
We further have for any tensor $\vA$ with eigenvalues $\lambda_1,\lambda_2,\lambda_3$ the \e{invariants}
\begin{align}
	I_1(\vA) &:= \tr\vA = \lambda_1 + \lambda_2 + \lambda_3\label{def:I1}\\
	I_2(\vA) &:= \frac{1}{2}\left((\tr \vA)^2 - \tr\vA^2\right) = \lambda_1\lambda_2 + \lambda_1\lambda_3 + \lambda_2\lambda_3\\
	I_3(\vA) &:= \det\vA = \lambda_1\lambda_2\lambda_3\label{def:I3}\\
	I_4(\vA,\vv) &:= \vv\cdot\vA\vv = \vA : (\vv \otimes \vv)\label{def:I4} =: \lambda^2_{\vv}\\
	I_5(\vA,\vv) &:= \vv\cdot\vA^2\vv\label{def:I5}
\end{align}
with derivatives for \e{symmetric} $\vA$
\begin{align}
	\d{I_1}{\vA}(\vA) &:= \d{\tr\vA}{\vA}(\vA) = \d{\vI:\vA}{\vA}(\vA) = \vI\\
	\d{I_2}{\vA}(\vA) &:= I_1(\vA)\vI - \vA\\
	\d{I_3}{\vA}(\vA) &:= I_3\vA^{-1}\\
	\d{I_4}{\vA}(\vA) &:= \vv \otimes \vv\label{def:dI4}\\
	\d{I_5}{\vA}(\vA) &:= \vv \otimes \vA\vv + \vv\vA \otimes \vv,
\end{align}
see \cite[p.216/p.268]{Holzapfel2000}.

Now, $\Psi$ can actually be formulated using the invariants \eqref{def:I1}-\eqref{def:I5} as
\[
	\Psi(\vC) = \Psi(I_1(\vC),I_2(\vC),I_3(\vC),I_4(\vC,\va_0),I_5(\vC,\va_0)).
\]
This gives the general form
\begin{align}
	\vS(X,t) &= 2\d{\Psi}{\vC}(I_1,I_2,I_3,I_4,I_5)\\
		&= \d{\Psi}{I_1}\d{I_1}{\vC} + \d{\Psi}{I_2}\d{I_2}{\vC} + \d{\Psi}{I_3}\d{I_3}{\vC} + \d{\Psi}{I_4}\d{I_4}{\vC} + \d{\Psi}{I_5}\d{I_5}{\vC}\\
		&= 2\Biggl[\left(\d{\Psi}{I_1} + I_1\d{\Psi}{I_2}\right)\vI - \d{\Psi}{I_2}\vC + I_3\d{\Psi}{I_3}\vC^{-1}\\
		&\quad+ \d{\Psi}{I_4}\va_0\otimes\va_0 + \d{\Psi}{I_5}(\va_0\otimes\vA\va_0 + \va_0\vA\otimes\va_0)\Biggr].
\end{align}

Further, we consider \e{incompressible} material, which is expressed by the condition $J(X,t) = 1 \fo X,t$.
This leads to the constant third invariant $I_3 \equiv 1$, where an additional Lagrange-multiplier $p$ is introduced
to satisfy the condition via
\begin{align}
	\vS(X,t) = p\vC^{-1}(X,t) + 2\d{\Psi}{\vC}(\vC(X,t))
\end{align}

In the following, we will specify different $\Psi$ to create an additive split as 
\begin{align}
	\vS(X,t) = p\vC^{-1}(X,t) + \vSi(X,t) + \vSa(X,t) + \vSf(X,t)\label{def:S_split}
\end{align}

\subsubsection{Isotropic stress tensor $\vSi$}
Now, for the isotropic part we use
\[
	\Psi(I_1,I_2,I_3) = c_{10}(I_1-3) + c_{01}(I_2-3),
\]
which gives
\begin{align}
	\vSi(X,t) &= 2(c_{10} + I_1c_{01})\vI - 2c_{01}\vC(X,t)\\
	\vPi(X,t) &= 2(c_{10} + I_1c_{01})\vF(X,t) - 2c_{01}\vF(X,t)\vC(X,t)
\end{align}

\subsubsection{Anisotropic stress tensor $\vSa$}
Further, we introduce the stretch $\lambda_f$ in fibre direction $\va_0(X)$ as
\begin{align*}
	\lambda_f(X,t) &:= \sqrt{I_4(\vC(X,t),\va_0(X))}
\end{align*}
Now, according to \cite{Markert2005}, we employ
\begin{align*}
	\Psi(\lambda_f) := \sumi \frac{b_i}{d_i}(\lambda_f^{d_i} - 1) - b_i\ln\lambda_f
\end{align*}
with $n=1$ summands, so that
\begin{align}
	\d{\Psi}{\vC}(\lambda_f) &= \d{\Psi}{\lambda_f}(\lambda_f)\d{\lambda_f}{I_4}(I_4)\d{I_4}{\vC}(\vC,\va_0),\label{eq:dpsidC}
\end{align}
with
\begin{align*}
		 \d{\Psi}{\lambda_f}(\lambda_f) &= \frac{b_1(d_1-1)}{d_1}\lambda_f^{d-1} - b_1\lambda_f^{-1},\\
		 \d{\lambda_f}{I_4}(I_4) &= \frac{1}{2\sqrt{I_4}} = \frac{1}{2\lambda_f}.
\end{align*}
With \eqref{def:dI4} and the above we get
\begin{align*}
	2\d{\Psi}{\vC}(\lambda_f) &= 2\left(\frac{b_1(d_1-1)}{d_1}\lambda_f^{d-1} - b_1\lambda_f^{-1}\right) \frac{1}{2\lambda_f}\va_0\otimes\va_0\\
	&= \left(\frac{b_1(d_1-1)}{d_1}\lambda_f^{d-2} - b_1\lambda_f^{-2}\right) \va_0\otimes\va_0,
\end{align*}
leading to
\begin{align}
	\vSa(X,t) &= \left(\frac{b_1(d_1-1)}{d_1}\lambda_f^{d-2}(X,t) - b_1\lambda_f^{-2}(X,t)\right) \va_0(X)\otimes\va_0(X)\\
	\vPa(X,t) &= \left(\frac{b_1(d_1-1)}{d_1}\lambda_f^{d-2}(X,t) - b_1\lambda_f^{-2}(X,t)\right) \vF(X,t)\va_0(X)\otimes\va_0(X)
\end{align}

\subsubsection{Active force tensor $\vSf$}
Here we define
\begin{align*}
	\Psi(\lambda_f) &:= \intl{0}{\lambda_f} p^{act}(s)ds,\\
	p^{act}(\lambda_f) &:= p^{max}\gamma(\alpha,\lambda_f,\lambda_f'),
\end{align*}
with suitable constant $p^{max}$ and function $\gamma$.
Similar to \eqref{eq:dpsidC} we obtain
\begin{align}
	\d{\Psi}{\vC}(\lambda_f) &= \frac{p^{max}}{2\lambda_f}\gamma(\alpha,\lambda_f,\lambda_f')\va_0\otimes\va_0
\end{align}
This gives
\begin{align}
	\vSf(X,t) &= \frac{p^{max}}{\lambda_f(X,t)}\gamma\left(\alpha,\lambda_f(X,t),\d{\lambda_f}{t}(X,t)\right)\va_0(X)\otimes\va_0(X)\\
	\vPf(X,t) &= \frac{p^{max}}{\lambda_f(X,t)}\gamma\left(\alpha,\lambda_f(X,t),\d{\lambda_f}{t}(X,t)\right)\vF(X,t)\va_0(X)\otimes\va_0(X)
\end{align}

\subsubsection{Overall stress tensor}
Adding the different parts together as given in \eqref{def:S_split} now gives
\begin{align}
	\vS(X,t) &= 2(c_{10} + I_1c_{01})\vI - 2c_{01}\vC(X,t)\\
			 &\quad+\Biggl[\left(\frac{b_1(d_1-1)}{d_1}\lambda_f^{d-2}(X,t) - b_1\lambda_f^{-2}(X,t)\right)\nonumber\\
			 &\quad+\frac{p^{max}}{\lambda_f(X,t)}\gamma\left(\alpha,\lambda_f(X,t),\d{\lambda_f}{t}(X,t)\right)\Biggr]\va_0(X)\otimes\va_0(X)\nonumber\\
	g(\lambda_f)&:= \left(\frac{b_1(d_1-1)}{d_1}\lambda_f^{d-2} - b_1\lambda_f^{-2}\right)+\frac{p^{max}}{\lambda_f}\gamma\left(\alpha,\lambda_f,\d{\lambda_f}{t}\right)\\			 
	\vP(X,t) &= \vF(X,t)\vS(X,t)\label{def:completeP}\\
			 &= 2(c_{10} + I_1(\vC(X,t))c_{01})\vF(X,t) - 2c_{01}\vF(X,t)\vC(X,t)\nonumber\\
			 &\quad+g(\lambda_f(X,t))\vF(X,t)\va_0(X)\otimes\va_0(X)\nonumber
\end{align}

% \subsubsection{Detour}
% Prescribing $\vP$ will yield the behaviour of the overall system.
% The equation $\eqref{def:T}$ also holds for values in the current configuration, i.e.
% \[
% 	\vt(x,t,n) = \vsig(x,t)n,
% \]
% where $\vsig$ is the symmetric \e{Cauchy stress tensor}.