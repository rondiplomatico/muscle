For discretization we use taylor-hood elements, which means $C^2(\Or)$ test/ansatz functions for the displacement and $C^1(\Or)$ test/ansatz functions for
the pressure.
We fix nodes $\{\vx_1,\ldots,\vx_N\}$, the test space $\H^2$ spanned by $C^2(\Or)$ test functions
$\varphi_1,\ldots,\varphi_N$ and the test space $\H^1$ spanned by $C^1(\Or)$ test functions
$\psi_1,\ldots,\psi_N$. 
We have $\varphi_i(\vx_j) = \delta_{ij}$ and $\varphi_k \equiv 0$ on $\partial\Or$ for all $k$, which holds true for $\psi_i$ as well.
Now, equation \eqref{def:maineq} (without body forces) and the incompressibility constraint give the system 
\begin{align}
	\rho_0(X)\d{\vV}{t}(X,t) &= \divergence\vP(X,t) \qquad \fo X\in\Or,\\
	J(X,t) &= 1 \qquad \fo X\in\Or,
\end{align}
for which we have the spatial weak form
\begin{align}
	\intor\rho_0(X)\d{\vV}{t}(X,t)\varphi(X)dX &= \intor\divergence\vP(X,t)\varphi(X)dX &&\fo \varphi\in\H^2\\
	\intor (J(X,t)-1)\psi(X)dX &= 0 && \fo \psi\in\H^1
\end{align}
The right hand side can be transformed as
\begin{align*}
	\intor \div{\vP(X,t)}\varphi(X) dX &= \intor \div{\vP(X,t)\varphi(X)} - \vP(X,t)\divergence\varphi(X) dX\\
		 &= \underbrace{\intorb \vP(X,t)\varphi(X) dN}_{=0} - \intor \vP(X,t)\divergence\varphi(X)dX\\
		 &= \intor -\vP(X,t)\divergence\varphi(X) dX
\end{align*}
As $\H$ is spanned by linear independent $\varphi_k$,
this gives $3N+N$ equations (for each test function $\varphi_k$, 3D displacement and one constraint) as
\begin{align}
	\intor \rho_0(X)\d{\vV}{t}(X,t)\varphi_k(X)dX + \intor\vP(X,t)\divergence\varphi_k(X) dX &= \vnull && \fo k=1\ldots N\label{def:weakform}\\
	\intor (J(X,t)-1)\psi_k(X)dX &= 0 && \fo k=1\ldots N\label{def:weakform_nb}
\end{align}

Now as solution space we choose the linear ansatz $\chi\in\H^2, p\in\H^1$ as
\begin{align}
	\chi(X,t) &= \sumi \vc_i(t)\varphi_i(X), && \vc_i(t):[0,T] \to\R^3.\\
	p(X,t) &= \sumi d_i(t)\psi_i(X), && d_i(t):[0,T] \to\R.
\end{align}
With this we have
\begin{align}
	\vV(X,t) &= \d{\chi}{t}(X,t) = \sumi \vc_i'(t)\varphi_i(X)\\
	\vA(X,t) &= \d{\vV}{t}(X,t) = \sumi \vc_i''(t)\varphi_i(X)\\
	\vF(X,t) &= \d{\chi}{X}(X,t) = \m{\grad \chi_1(X,t)\\ \grad \chi_2(X,t)\\ \grad \chi_3(X,t)}
		   = \sumi\m{c_{i,1}\d{\varphi_i}{X_1} & \dots & c_{i,1}\d{\varphi_i}{X_3}\\
		   				\vdots & \ddots & \vdots\\
		   				c_{i,3}\d{\varphi_i}{X_1} & \dots & c_{i,3}\d{\varphi_i}{X_3}}\\
		   &=\sumi \vc_i(t)\cdot \grad\varphi_i(X) = \sumi \vc_i(t)\otimes \divergence\varphi_i(X).
\end{align}
This can be used to derive the quantities
\begin{align}
	I_1(\vC(X,t)) &= \tr\vC(X,t) = \tr(\vF(X,t)^T\vF(X,t)) = \vF(X,t) : \vF(X,t)\nonumber\\
	&= \sumi \vc_i(t)\otimes \divergence\varphi_i(X) : \sumi \vc_i(t)\otimes \divergence\varphi_i(X)\nonumber\\
	&= \sumi\sumj \left(\vc_i(t)\otimes \divergence\varphi_i(X)\right) : (\vc_j(t)\otimes \divergence\varphi_j(X))\nonumber\\
	&= \sumi\sumj (\vc_i(t) \cdot \vc_j(t))(\divergence\varphi_i(X) \cdot \divergence\varphi_j(X))\label{eq:I1_basis}\\
	\vF(X,t)^T\vF(X,t) &= \sumi\sumj \bigl(\vc_i(t)\otimes\divergence\varphi_i(X)\bigr)^T\vc_j(t)\otimes\divergence\varphi_j(X)\nonumber\\
	 &= \sumi\sumj \bigl(\divergence\varphi_i(X)\otimes\vc_i(t)\bigr)\vc_j(t)\otimes\divergence\varphi_j(X)\nonumber\\
	 &=\sumi\sumj (\vc_i(t)\cdot \vc_j(t))\divergence\varphi_i(X)\otimes\divergence\varphi_j(X)\nonumber\\
	\lambda_f(X,t)^2 &=I_4(\vC(X,t),a_0(X)) = \vC(X,t):(a_0(X)\otimes a_0(X))\nonumber\\
			&=  \vF(X,t)^T\vF(X,t) : (a_0(X)\otimes a_0(X))\nonumber\\
			&=  \sumi\sumj (\vc_i(t)\cdot \vc_j(t))\bigl(\divergence\varphi_i(X)\otimes\divergence\varphi_j(X)\bigr) : (a_0(X)\otimes a_0(X))\nonumber\\
			&=  \sumi\sumj (\vc_i(t)\cdot \vc_j(t))\left(\divergence\varphi_i(X)\cdot a_0(X)\right)(\divergence\varphi_j(X)\cdot a_0(X))\label{eq:lambdaf_basis}\\
	\d{\lambda_f^2}{t}(X,t) &= 2\lambda_f \suml{i,j}{N} (\vc'_i(t)\cdot \vc_j(t) + \vc_i(t)\cdot \vc'_j(t))\left(\divergence\varphi_i(X)\cdot a_0(X)\right)(\divergence\varphi_j(X)\cdot a_0(X))\nonumber\\
		&= 4\lambda_f \suml{i,j}{N} (\vc_i(t)\cdot \vc'_j(t))\left(\divergence\varphi_i(X)\cdot a_0(X)\right)(\divergence\varphi_j(X)\cdot a_0(X))\nonumber\\
    \vF(X,t)\vF(X,t)^T\vF(X,t) &= \sumk\vc_k(t)\otimes\divergence\varphi_k(X)\sumi\sumj (\vc_i(t)\cdot \vc_j(t))\divergence\varphi_i(X)\otimes\divergence\varphi_j(X)\nonumber\\
    	&= \suml{i,j,k}{N}(\vc_i(t)\cdot \vc_j(t))(\vc_k(t)\otimes\divergence\varphi_k(X))\divergence\varphi_i(X)\otimes\divergence\varphi_j(X)\nonumber\\
    	&= \suml{i,j,k}{N}(\vc_i(t)\cdot \vc_j(t))(\divergence\varphi_k(X)\cdot\divergence\varphi_i(X))(\vc_k(t)\otimes\divergence\varphi_j(X))\nonumber\\
    \vF(X,t)(\va_0(X)\otimes\va_0(X)) &= \sumi \vc_i(t)\otimes \divergence\varphi_i(X)(\va_0(X)\otimes\va_0(X))\nonumber\\
    &= \sumi (\vc_i(t)\otimes \va_0(X))(\divergence\varphi_i(X)\cdot\va_0(X))\nonumber
\end{align}

\subsection{Domain decomposition, basis functions and master reference volume}
We specify a master reference volume $\Om = [-1,1]^3$, and for $\H^1$ the basis functions
\begin{align*}
	N_i:\Om &\to [0,1]\\
	N_i(X)  &= (1+(-1)^{i}X_1)(1+(-1)^{\ceil{\frac{i}{2}}}X_2)(1+(-1)^{\ceil{\frac{i}{4}}}X_3),\quad i=1\ldots 8
\end{align*}
which amounts to trilinear functions equal to one in each corner of the hexahedron having the property
\begin{align}
	\suml{i=1}{8}N_i &\equiv 1 & \text{i.e.}\quad \suml{i=1}{8}N_i(X) &= 1 \fo X\in\Om.
\end{align}

We further specify a decomposition $\Or = \Omega_1\cup \ldots \cup \Omega_M$, where
$\Omega_m = [\vx^m_1,\ldots,\vx^m_8]$ is a deformed cube specified by the eight corner points $\vx_i^m\in\R^3$.
Using the notation
\begin{align}
	\vN(X) &:= \m{N_1(X) & \ldots & N_8(X)}^T\in\R^{8\times 1}\\
	\vX^m &:= \m{\vx^m_1 \ldots \vx^m_8} \in\R^{3\times 8}
\end{align}
 we can specify an \e{isogeometric mapping} or diffeomorphism
\begin{align}
	\Phi_m : \Om &\to \Omega_m\\
	X &\mapsto \vX^m\vN(X)
\end{align}
which satisfies $\Omega_m = \Phi_m(\Or)$ and has the Jacobian
\begin{align}
	\nabla\Phi_m(X) &= \vX^m\nabla\vN(X) \in \R^{3\times 3}.\label{def:refbasis_jacobian}
\end{align}
 
With the edge/volume/neighbor index sets
\begin{align}
	E_m &:= E(\Omega_m) := \{i\in\{1\ldots N\}~|~ \vx_i\in\Omega_m\}, \quad m=1\ldots M,\\
	V_k &:= V(\vx_k) := \{i\in\{1\ldots M\}~|~ \vx_k\in\Omega_i\}, \quad k=1\ldots N,\\
	\Nk &:= \{i\in\{1\ldots N\}~|~ \vx_i\in E_m, m\in V_k\} = \{i ~|~ \supp\varphi_i \cap \supp\varphi_k \neq\es\},
\end{align}
we now define the basis functions $\varphi_k$ via
\begin{align}
	\varphi_k(X) &:= \begin{cases}
		N_{l(k,m)}(\Phi^{-1}_m(X)), & X\in\Omega_m, m\in V_k,\\
		0 & \text{else},	
	\end{cases}\label{def:referencebasisfun}\\
	l(k,m) &:= \{i\in\{1\ldots 8\}~|~ N_i(\Phi_m^{-1}(\vx_k))=1\},\quad m=1\ldots M.
\end{align}
Here $l(k,m)$ refers to the local corner index of the basis function on $\Omega_m$ that equals one at $\vx_k$.

We further apply a Gauss Quadrature on $\Om$ with
\begin{align}
	X_1,\ldots,X_G  &\in \Om &&\text{Gauss points},\\
	w_1,\ldots,w_G  &\in \R &&\text{Gauss weights},
\end{align}
so that
\begin{align}
	\intom f(X,t)dX \approx \sumgp w_pf(X_p).\label{def:gaussintapprox}
\end{align}
With the above and the transformation theorem we now have for any integrable $f(X,t)$ and the shorthand $\pmp := \Phi_m(X_p)$ that
\begin{align}
	\intl{\Omega_m}{}f(X,t)\varphi_k(X)dX &= \intl{\Phi_m(\Omega)}{}f(X,t)\varphi_k(X)dX\label{eq:fphidx}\\
	& = \intl{\Om}{}f(\Phi_m(X),t)\varphi_k(\Phi_m(X))|\det \nabla\Phi(X)|dX\nonumber\\
	& \re{def:gaussintapprox} \sumgp w_pf(\pmp,t)\varphi_k(\pmp)|\det \nabla\Phi(X_p)|\nonumber\\
	& \re{def:referencebasisfun} \sumgp w_pf(\pmp,t)N_{l(k,m)}(\Phi_m^{-1}(\pmp))|\det \nabla\Phi(X_p)|\nonumber\\
	& \re{def:refbasis_jacobian} \sumgp w_pf(\pmp,t)N_{l(k,m)}(X_p)|\det \vX^m\nabla\vN(X_p)|\nonumber\\
	\intl{\Omega_m}{}f(X,t)\nabla\varphi_k(X)dX &= \sumgp w_pf(\pmp,t)\nabla N_{l(k,m)}(X_p)|\det \vX^m\nabla\vN(X_p)|\label{eq:fgradphidx}
\end{align}

\subsection{Application to weak form}
We introduce the shorthand
\begin{align}
	\jmp &:= |\det \vX^m\nabla\vN(X_p)|.\label{def:jacshorthand}
\end{align}
Finally, for a fixed $k\in\{1,\ldots, N\}$ we have for the first term of \eqref{def:weakform}:
\begin{align*}
	\intor \rho_0(X)\d{\vV}{t}(X,t)\varphi_k(X)dX
		&= \suml{m=1}{M}\intl{\Omega_m}{} \rho_0(X)\d{\vV}{t}(X,t)\varphi_k(X)dX\\
		&= \sumvk\intl{\Omega_m}{} \rho_0(X)\d{\vV}{t}(X,t)\varphi_k(X)dX\\  		
		&= \sumvk\intl{\Omega_m}{} \rho_0(X)\sumi \vc_i''(t)\varphi_i(X)\varphi_k(X)dX\\
		&\re{eq:fphidx} \sumvk\sumgp w_p \rho_0(\pmp)\sumi \vc_i''(t)\varphi_i(\pmp) N_{l(k,m)}(X_p)\jmp\\
		&= \sumvk\sumnk \vc_i''(t)\sumgp w_p \rho_0(\pmp)N_{l(i,m)}(X_p) N_{l(k,m)}(X_p)\jmp
\end{align*}

The second term of \eqref{def:weakform}, with \eqref{def:completeP}, reads as
\begin{align*}
		&\intor\vP(X,t)\divergence\varphi_k(X) dX\\
		=& \suml{m=1}{M}\intl{\Omega_m}{}\vP(X,t)\divergence\varphi_k(X) dX
			= \sumvk\intl{\Omega_m}{}\vP(X,t)\divergence\varphi_k(X) dX\\
		\re{eq:fgradphidx}&\sumvk\sumgp w_p\vP(\pmp,t)\nabla N_{l(k,m)}(X_p)|\det \vX^m\nabla\vN(X_p)|\\
		\re{def:jacshorthand}&\sumvk\sumgp w_p\vP(\pmp,t)\nabla N_{l(k,m)}(X_p)\jmp\\
		\re{def:completeP}&\sumvk\sumgp w_p\Bigl[p(\pmp,t)\vF^{-T}(\pmp,t) + 2(c_{10} + I_1(\vC(\pmp,t))c_{01})\vF(\pmp,t)\\
			 &- 2c_{01}\vF(\pmp,t)\vC(\pmp,t)+g(\lambda_f(\pmp,t))\vF(\pmp,t)\va_0(\pmp)\otimes\va_0(\pmp)\Bigr]\nabla N_{l(k,m)}(X_p)\jmp\nonumber.
\end{align*}
In detail we have
\begin{align*}
	I_1(\vC(\pmp,t)) &\re{eq:I1_basis} \sumi\sumj (\vc_i(t) \cdot \vc_j(t))(\divergence\varphi_i(\pmp) \cdot \divergence\varphi_j(\pmp))\\
	&\re{def:referencebasisfun} \suml{i,j\in E_m}{} (\vc_i(t) \cdot \vc_j(t))(\divergence\varphi_i(\pmp) \cdot \divergence\varphi_j(\pmp))\\
	&\re{def:referencebasisfun} \suml{i,j\in E_m}{} (\vc_i(t) \cdot \vc_j(t))(\nabla N_{l(i,m)}(X_p) \cdot \nabla N_{l(j,m)}(X_p))\\
	\vF(\pmp,t)\vC(\pmp,t) &= \vF(\pmp,t)\vF(\pmp,t)^T\vF(\pmp,t)\\
    	&= \suml{i,j,k}{N}(\vc_i(t)\cdot \vc_j(t))(\divergence\varphi_k(\pmp)\cdot\divergence\varphi_i(\pmp))(\vc_k(t)\otimes\divergence\varphi_j(\pmp))\\
    	&= \suml{i,j,q\in E_m}{}(\vc_i(t)\cdot \vc_j(t))(\nabla N_{l(q,m)}(X_p)\cdot\nabla N_{l(i,m)}(X_p))(\vc_q(t)\otimes\nabla N_{l(j,m)}(X_p))\\
    \lambda_f(\pmp,t)^2 &\re{eq:lambdaf_basis}
    		\sumi\sumj (\vc_i(t)\cdot \vc_j(t))\left(\divergence\varphi_i(\pmp)\cdot a_0(\pmp)\right)(\divergence\varphi_j(\pmp)\cdot a_0(\pmp))\nonumber\\
    	&= \suml{i,j\in E_m}{} (\vc_i(t)\cdot \vc_j(t))\left(\nabla N_{l(i,m)}(X_p)\cdot a_0(\pmp)\right)(\nabla N_{l(j,m)}(X_p))\cdot a_0(\pmp))\nonumber
\end{align*} 
\begin{align*}  	
    \vF(\pmp,t)(\va_0(\pmp)\otimes\va_0(\pmp)) 
    	&= \sumi (\vc_i(t)\otimes \va_0(\pmp))(\divergence\varphi_i(\pmp)\cdot\va_0(\pmp))\nonumber\\
    	&= \suml{i\in E_m}{} (\vc_i(t)\otimes \va_0(\pmp))(\nabla N_{l(i,m)}(X_p)\cdot\va_0(\pmp))\nonumber
\end{align*}
By the same arguments, condition/equation \eqref{def:weakform_nb} now reads as
\begin{align*}
	\intor (J(X,t)-1)\varphi_k(X)dX &= \sumvk\sumgp w_p (\det\vF(\pmp,t)-1) N_{l(k,m)}(X_p)\jmp\\
	&= \sumvk\sumgp w_p \left(\det\left(\sumi \vc_i(t)\otimes \divergence\varphi_i(\pmp)\right)-1\right) N_{l(k,m)}(X_p)\jmp\\
	&= \sumvk\sumgp w_p \left(\det\left(\sum_{i\in E_m} \vc_i(t)\otimes \divergence\varphi_i(\pmp)\right)-1\right) N_{l(k,m)}(X_p)\jmp.\\
\end{align*}
